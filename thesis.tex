% **************************************************
% Document Class Definition
% **************************************************
\documentclass[%
    paper=A4,               % paper size --> A4 is default in Germany
    parskip=half,           % spacing value / method for paragraphs
    chapterprefix=true,     % prefix for chapter marks
    11pt,                   % font size
    headings=normal,        % size of headings
    bibliography=totoc,     % include bib in toc
    listof=totoc,           % include listof entries in toc
    titlepage=on,           % own page for each title page
    captions=tableabove,    % display table captions above the float env
    chapterprefix=false,    % do not display a prefix for chapters
    appendixprefix=false,    % but display a prefix for appendix chapter
    draft=false,            % value for draft version
]{scrreprt}%


% ******************************************************************************************
% Added correct packages for character coding and fonts for the document when using PDFLaTeX
% ******************************************************************************************
\RequirePackage[utf8]{inputenc}   	% UTF-8 Support
\RequirePackage[T1]{fontenc}		% T1 Fonts for font encoding -> also set in cleanthesis.sty
%
\RequirePackage[T1]{url} 			% Web addresses with T1 encoding
\urlstyle{tt} 						% Web addresses in tt-style
%

% **************************************************
% Set extra options for later-on loaded packages
% **************************************************
\PassOptionsToPackage{defernumbers=true}{biblatex}	% important for split-bibliographies like cleanthesis.sty uses

% **************************************************
% Additional useful little LaTeX helper packages
% **************************************************
\usepackage[section]{placeins}	% Insert float barriers before every section
\usepackage{flafter}			% Never insert figures before the actual reference

% **************************************************
% Setup YOUR thesis document in this file !
% **************************************************
% !TEX root = thesis.tex


% **************************************************
% Files' Character Encoding
% **************************************************
\RequirePackage[utf8]{inputenc}   	% UTF-8 Support
\RequirePackage[T1]{url} 			% Web addresses with T1 encoding
\urlstyle{tt} 						% Web addresses in tt-style

% **************************************************
% Information and Commands for Reuse
% **************************************************
\newcommand{\thesisTitle}{A Practical Analysis of UEFI Threats Against Windows~11}
\newcommand{\thesisName}{Joshua Machauer}
\newcommand{\thesisSubject}{Bachelor's Thesis}
\newcommand{\thesisDate}{December 25, 2022}
\newcommand{\thesisDateGerman}{25. Dezember 2022}
\newcommand{\thesisVersion}{Draft 1.0}

\newcommand{\thesisFirstReviewer}{Prof. Dr. Jean-Pierre Seifert}
\newcommand{\thesisFirstReviewerUniversity}{\protect{Technische Universität Berlin}}
\newcommand{\thesisFirstReviewerDepartment}{Electrical Engineering and Computer Science}

\newcommand{\thesisSecondReviewer}{Prof. Dr. Stefan Schmid}
\newcommand{\thesisSecondReviewerUniversity}{\protect{Technische Universität Berlin}}
\newcommand{\thesisSecondReviewerDepartment}{Electrical Engineering and Computer Science}

\newcommand{\thesisFirstSupervisor}{Hans Niklas Jacob}
\newcommand{\thesisSecondSupervisor}{Christian Werling}

\newcommand{\thesisUniversity}{\protect{Technische Universität Berlin}}
\newcommand{\thesisUniversityDepartment}{Electrical Engineering and Computer Science}
\newcommand{\thesisUniversityInstitute}{Institute of Software Engineering and Theoretical Computer Science}
\newcommand{\thesisUniversityGroup}{Security in Telecommunications (SecT)}
\newcommand{\thesisUniversityCity}{Berlin}
\newcommand{\thesisUniversityStreetAddress}{Ernst-Reuter-Platz 7}
\newcommand{\thesisUniversityPostalCode}{10587}



% **************************************************
% Debug LaTeX Information
% **************************************************
%\listfiles


% **************************************************
% Load and Configure Packages
% **************************************************
\usepackage[english]{babel} % babel system, adjust the language of the content
\usepackage[printonlyused]{acronym}
\PassOptionsToPackage{% setup clean thesis style
    figuresep=colon,%
    hangfigurecaption=false,%
    hangsection=true,%
    hangsubsection=true,%
    sansserif=false,%
    configurelistings=true,%
    colorize=full,%
    colortheme=bluemagenta,%
    configurebiblatex=true,%
    bibsys=biber,%
    bibfile=bib-refs,%
    %bibstyle=alphabetic,%			-> citations with alphabetic enumeration
    bibstyle=numeric-comp,%		-> citations with numeric compressed numeric sorting
    % bibstyle=ieee-alphabetic,%		-> citations with IEEE transactions-like style with numeric labels
    %bibsorting=nty,% 
    bibsorting=none,%				-> sorted in the order of the citations from start to finish
}{cleanthesis}
\usepackage{cleanthesis}

\hypersetup{% setup the hyperref-package options
    pdftitle={\thesisTitle},    %   - title (PDF meta)
    pdfsubject={\thesisSubject},%   - subject (PDF meta)
    pdfauthor={\thesisName},    %   - author (PDF meta)
    plainpages=false,           %   -
    colorlinks=false,           %   - colorize links?
    pdfborder={0 0 0},          %   -
    breaklinks=true,            %   - allow line break inside links
    bookmarksnumbered=true,     %
    bookmarksopen=true          %
}


\graphicspath{{./images/}{./figures/}}
\usepackage{svg}

\usepackage{multirow}
\usepackage[shortcuts]{extdash}

\usepackage{textcmds}
\usepackage{booktabs}
\usepackage{soul}
\usepackage{colortbl}
\usepackage{nameref}


% **************************************************
% Document CONTENT
% **************************************************
\begin{document}

% uncomment the following command to fill up pages with
% whitespace instead of aligning the first and last lines
% of a page (see \raggedbottom vs. \flushbottom)
%\raggedbottom

% --------------------------
% rename document parts
% --------------------------
\renewcaptionname{english}{\figurename}{Fig.}
\renewcaptionname{english}{\tablename}{Tab.}
\renewcommand{\lstlistlistingname}{List of Source Code Listings}
\renewcommand{\lstlistingname}{Listing}


% ---------------------------------------------------------------------------------------------------
% Newly defined LaTeX commands for typographic use (e.g., emphasizing and highlighting special terms) 
% ---------------------------------------------------------------------------------------------------
% WARNING: in all four commands below you will need to escape special LaTeX characters & % $ # _ { } ~ ^ \
% escape them via \&  \%  \$  \#  \_  \{  \}  \textasciitilde{} or \~{}  \textasciicircum{}  \textbackslash{}
% refer to https://en.wikibooks.org/wiki/LaTeX/Special_Characters#Other_symbols for more example
%
% to emphasize all source-code-related information outside a listing
% for example: ``The \code{copy(\&buffer)} function''
\newcommand{\code}[1]{\texttt{#1}\xspace}
%
% to emphasize all non-source-code-related information
% for example: ``The program uses the \cursive{Mersenne Twister} function ''
\newcommand{\cursive}[1]{\textit{#1}\xspace}
%
% to emphasize program and tool names
% for example: ``The \program{opp\_runall} tool''
\newcommand{\program}[1]{\textsf{#1}\xspace}
%
% to emphasize file names
% for example: ``The \filename{result\_scenario1.sca} file''
\newcommand{\filename}[1]{\textsl{#1}\xspace}
%
% ---------------------------------------------------------------------------------------------------
% NOTE: the alternative four commands below can be used just like the ones above WITHOUT the need to escape special LaTeX characters
% BUT:  these command variants cannot be used inside other commands or macros (e.g., \caption{})
%
%\usepackage{xparse}		% package to support parsing of customized commands and macros
%
% to emphasize all source-code-related information outside a listing without escaping LaTeX characters
% for example: ``The \Code{copy(&buffer)} function''
%\DeclareDocumentCommand\Code{ v }{%
%{\texttt{#1}\xspace}%
%}%
%
% to emphasize all non-source-code-related information
% for example: ``The program uses the a \Cursive{Mersenne_Twister} function ''
%\DeclareDocumentCommand\Cursive{ v }{%
%{\textit{#1}\xspace}%
%}%
%
% to emphasize program and tool names without escaping LaTeX characters
% for example: ``The \Program{opp_runall} tool''
%\DeclareDocumentCommand\Program{ v }{%
%{\textbf{#1}\xspace}%
%}%
%
% to emphasize file names without escaping LaTeX characters
% for example: ``The \Filename{result#01.res} file''
%\DeclareDocumentCommand\Filename{ v }{%
%{\textsl{#1}\xspace}%
%}%
%

% --------------------------pdfbookmark
% Front matter
% --------------------------
\pagenumbering{Alph}
\pagestyle{empty}				% no header or footers
\input{content/titlepages}		% INCLUDE: all titlepages
\clearpage

% !TEX root = ../thesis.tex
%
%************************************************
% Declaration
%************************************************
\pdfbookmark[0]{Selbst\"andigkeitserkl\"arung}{Selbst\"andigkeitserkl\"arung}
\chapter*{Selbst\"andigkeitserkl\"arung}
\label{sec:declaration}
\thispagestyle{empty}

Hiermit erkl\"are ich, dass ich die vorliegende Arbeit selbstst\"andig und eigenh\"andig sowie ohne
unerlaubte fremde Hilfe und ausschlie\ss{}lich unter Verwendung der aufgef\"uhrten Quellen und
Hilfsmittel angefertigt habe.

\bigskip

\noindent\textit{\thesisUniversityCity, den \thesisDateGerman}

\smallskip

\begin{flushright}
	\begin{minipage}{5cm}
		\rule{\textwidth}{1pt}
		\centering\thesisName
	\end{minipage}
\end{flushright}

%*****************************************
%*****************************************
\clearpage

% !TEX root = ../thesis.tex

% http://williamstallings.com/Extras/Abstract.html
% https://www.enago.com/academy/abstract-versus-introduction-difference/

\pdfbookmark[0]{Abstract}{Abstract}
\chapter*{Abstract}
\label{sec:abstract}
\thispagestyle{empty}

% motivation
In Computer Security firmware attacks are one of the most feared security threats, executing during the boot process, they can already have full control over the system before an operating system and accompanying antivirus programs are even loaded.
With widespread adaption of standardized \acs{UEFI} firmware these threats have become less machine dependent, and able to target a host of systems at once.
% problem statement
Their appearances in the wild are rare as they are stealthy by nature. We categorize past analyses of \acs{UEFI} threats (against Windows) by their attack vector and perform our own.
% approach
With a deep-dive into the \acs{UEFI} environment we learn hands on about encountered security mechanisms targeting pre-boot attacks, setting our focus on Secure Boot and \acs{TPM}-assisted BitLocker.
% results
We were able to achieve system level privileged execution on Windows 11 by exploiting unrestricted hard drive access to deploy our payload and modify the Windows Registry. With BitLocker enabled, our \emph{BitLogger} was able to decrypt and mount the drive using a keylogged Recovery Key, or when part of the chain of trust using a \acs{VMK} sniffed from \acs{TPM} communication.
% conclusions
\acs{UEFI} threats are very powerful and discredit all system integrity, making it impossible to put any further trust into the system.

\acresetall

\pdfbookmark[0]{Zusammenfassung}{Zusammenfassung}
\chapter*{Zusammenfassung}
\thispagestyle{empty}
\label{sec:zusammenfassung}

\blindtext

\acresetall		% INCLUDE: the abstracts (english and german)
\clearpage

% !TeX root = ../thesis.tex
%
\pdfbookmark[0]{Acknowledgement}{Acknowledgement}
\addchap*{Acknowledgement}
\label{sec:acknowledgement}

 % INCLUDE: acknowledgement
\clearpage
%
\pagenumbering{Roman}
\currentpdfbookmark{\contentsname}{toc}
\setcounter{secnumdepth}{3}
\setcounter{tocdepth}{3}		% define depth of toc
\pagestyle{plain}				% no header or footers
\microtypesetup{protrusion=false}
\tableofcontents				% display table of contents
\microtypesetup{protrusion=true}
\clearpage

% --------------------------
% Body matter
% --------------------------
\pagenumbering{arabic}			% arabic page numbering
\setcounter{page}{1}			% set page counter
\pagestyle{scrheadings}			% header and footer style

% !TEX root = ../thesis.tex

% https://www.enago.com/academy/abstract-versus-introduction-difference/
% https://www.student.unsw.edu.au/introductions
% https://www.scribbr.com/dissertation/introduction-structure/

\chapter{Introduction}

% background 
% definition of (uefi/firmware) rootkit
\cite{microsoft-secure-windows-boot-process, microsoft-rootkits, veracode, kaspersky}
persistence
% motiviation
% problem statement
goals
% structure of the thesis
1/3 - 1/2 pages

% state the general topic and give some background
% provide a review of the literature related to the topic
% define the terms and scope of the topic
% outline the current situation
% evaluate the current situation (advantages/ disadvantages) and identify the gap
% identify the importance of the proposed research
% state the research problem/ questions
% state the research aims and/or research objectives
% state the hypotheses
% outline the order of information in the thesis
% outline the methodology
% !TeX root = ../../thesis.tex

\chapter{\acs{UEFI}/\acs{PI}}

\textcquote{uefi-spec-overview}{The \ac{UEFI} specifications define a new model for the interface between personal-computer \ac{OS} and \ac{PF}. \textelp{} Together, these provide a standard environment for booting an \ac{OS} and running pre-boot applications}.
The specifications making up this model are:

\begin{itemize}
    \item \acs{ACPI} Specification
    \item \acs{UEFI} Specification
    \item \acs{UEFI} Shell Specification
    \item \acs{UEFI} \acs{PI} Specification
    \item \acs{UEFI} \acs{PI} Distribution Packaging Specification
    \item \acs{TCG} \acs{EFI} Platform Specification
    \item \acs{TCG} \acs{EFI} Protocol Specification
\end{itemize}

The \ac{ACPI} and \ac{UEFI} \ac{PI} Distribution Packaging Specification are not required to be able to follow this thesis.
As for the other specifications, we briefly summarize the relevant content.


\clearpage
% !TeX root = ../../thesis.tex

\section{\acf{UEFI}}

The \ac{UEFI} specification itself is a pure interface specification, describing the programmatic interface for interaction with the \ac{PF}, merely stating what interfaces and structures a \ac{PF} has to offer and what an \ac{OS} may use \cite{beyond-bios}.

during boot system resources are owned by the firmware

protected mode, takes care of memory management and mapping with a one to one mapping of physical to virtual


% what are its concrete goals
- complete solution describing all features and capabilities
- abstract interfaces to support a range of processors without the need for knowledge about underlying hardware for the bootloader
- sharable persistent storage for platform support code
security
\TODO{MEMORY LAYOUT no memory protection, RWE everywhere}

It was designed to replace the legacy \acl{BF} \ac{BIOS} \TODO{which wasnt very standardized}, while also providing backwards compatibility by defining the \acf{CSM} allowing \ac{UEFI} firmware to boot legacy \ac{BIOS} applications.

system table with boot- and runttime service functions for the bootloader and os to call
datatables containing platform-related information

\cite{beyond-bios}

\subsection{\acf{GUID}}

The \ac{UEFI} environment depends on \acp{GUID}, also known \acp{UUID} to uniquely identify a variety of things, such as protocols, files, hard drive paritions.
\acp{GUID} are 128-bit long, statistically unique identifiers and can be generated on demand and without a centralized authority, statistically guaranteeing that there will be no duplicates on a system that combines hard and software from multiple vendors \cite{rfc-4122}.

\subsection{\acf{GPT}}

Partitions allow a disk to be distinctly separated into logical disks, allowing for each to be formatted with a different file systems.
Prior to \ac{UEFI} disks have been paritioned using the \ac{MBR} parition table, supporting up to 4 different partitions.
The \ac{MBR} is stored within the first sector, also optionally containing 424 bytes of bootable code through which the \ac{BIOS} boots \cite[Section 13.3.1]{uefi-spec}.
\ac{UEFI} is still backwards compatible with \ac{MBR} partitioned disks and contained on each disk, but \ac{UEFI} does not execute the boot code.
The \ac{MBR} is used in two different ways by the \ac{UEFI} environment, either as a legacy \ac{MBR} or a protective \ac{MBR}.
With the legacy \ac{MBR}, \ac{UEFI} uses the partitions defined in the \ac{MBR} parition table, where as the protective \ac{MBR} only has one partition spanning the entire disk.
The protective partition is for legacy devices and in reality \ac{GPT} partitioning is used to separate the disk.
For this \ac{UEFI} defines two \ac{OS} types used in \ac{MBR} parition entries.
One identifies the \ac{ESP}, the parition \ac{UEFI} boots from, within the legacy \ac{MBR} partition table and the other indicates that a protective parition is used \cite[Section 5]{uefi-spec}.
\cite[Section 5]{uefi-spec} defines the \ac{GPT} disk layout, with the \ac{GPT} format \ac{LBA} are 64 bit instead of 32 bit, allowing to support drives with up to 9400000000 \ac{TB} of storage, where as \ac{MBR} is limited to 2 \ac{TB}.
This is accompanied by allowing many more than 4 partitions, with Windows supporting up to 128 \cite{microsoft-windows-and-gpt-faq}.
\ac{GUID} are used to identify paritions and parition types, but also offering a human readable parition name.
\ac{GPT} also has a primary and a backup parition table for redundancy pruposes, the primary table follows the \ac{MBR} sector and the backup is at the end of the disk.

\subsection{\acf{ESP}}

The \ac{ESP} can reside any media that is supported by the \ac{UEFI} firmware and has to be \ac{FAT}32 formatted \cite[Section 13.3]{uefi-spec}.
It must contain an \lstinline{EFI} root directory \cite[Section 13.3.1.3]{uefi-spec} and all \ac{UEFI} applications, that are to be launched directly by the \ac{UEFI} firmware have to be located in subdirectories below the \lstinline{EFI} driectory \cite[Section 13.3.1.3]{uefi-spec}. Drivers and indirectly loaded applcations have no storage restrictions. Vendors are to use vendor\-/specifically named subdirectories within the \lstinline{EFI} directory. Fixed disks have no restrictions on the amount of \acp{ESP} present, whereas removable media is only allowed to have one \ac{ESP}, so that boot behavior is deterministic. In general the \ac{ESP} is identified by a specific \ac{GUID}, but implementations are allowed to support accordingly structured \ac{FAT} partitions. Since there is no limitation on the amount of \acp{ESP}, boot applications can share the drive with their \ac{OS}, or can be accumulated in a single system\-/wide \ac{ESP} \cite[Section 13.3.3]{uefi-spec}.


\subsection{\acs{UEFI} Images}

\ac{UEFI} Images are files containting executable code, they use a subset of the \ac{PE32}+ file format with a modified header signature.
The format comes with relocation tables, making it possible for the images to be executed in place or to be loaded at non pre\-/determined memory addresses.
They support multiple CPU architectures such as IA, ARM, RISC-V and x86.
There are three different subtypes of executables: applications, boot and runtime drivers. They mainly differ by their memory type and how it behaves.
Loading and transferring execution are two separate steps, so that security policies can be applied before executing a loaded image \cite[Section 2.1.1]{uefi-spec}.

Applications are always unloaded when they return execution, while drivers are only unloaded when they return an error code. This allows drivers to install their offered functionality upon intial executions and later calls to these functions jump back into the driver's image which is still loaded.
Boot drivers are unloaded when an \ac{OS} loader application transitions to runtime by taking over the memory management through the call of the boot service function \lstinline{ExitBootServices}, while runtime drivers remain loaded and are translated into the virtual memory mapping. \ac{OS} loaders only return execution in error cases.


\subsection{Protocols and Handles}

When \ac{UEFI} binaries are loaded only the entry point is \emph{linked}, the rest of the communication has to be programmatically discovered through protocol interfaces.
Protocols are created dynamically and provide a mechanism to allow extension of firmware capabilities over time \cite[Section 3.6]{tianocore-edk2-driver-writer-s-guide}.
They are C structures and may contain services, in the form of function pointers, or other data structures, they are identified by \acp{GUID} and stored in a single global database implemented by the firmware \cite{beyond-bios}.
This database is called the handle database, handles describe a logical grouping of one or more protocols \cite[Section 3.6]{tianocore-edk2-driver-writer-s-guide}.
Handles are unique per session and should not be saved across reboots \cite{beyond-bios}.
Multiple instances of a protocol identified by the same \ac{GUID} can exist on different handles, offering the same service on different devices.

\begin{figure}[htb]%
    \centering%
    \includegraphics[width=0.7\textwidth]{uefi/handle_types.jpg}%
    \caption{Handle types (taken from \cite[Figure 3]{tianocore-edk2-driver-writer-s-guide})}%
    \label{fig:handle-types}%
\end{figure}

\cite{tianocore-edk2-driver-writer-s-guide} explains the categories of handles that are formed by the type of protocols that are grouped. \autoref{fig:handle-types} shows these categories.

\begin{description}
    \item[Image handles] are handles of \ac{UEFI} images loaded into memory, as they support the \hyperref[lst:loaded-image-protocol]{Loaded Image Protocol}, giving access to information about the image in memory. This includes the image's address, size, memory type, origin and optional load options.
    \item[Driver handles] are handles that group the \ac{UEFI} Driver Model related protocols (Driver Binding Protocol, the two Component Name Protocols and the two Driver Diagnostics Protocols)
    \item[Driver image handles] are \ac{UEFI} Driver Model related protocols installed onto images loaded in memory.
    \item[Agent handles] is a term used in the \ac{UEFI} Driver Model, they describe tracked consumers of other protocols.
    \item[Controller/Device handles] are interchangably used to refer to physical and virtual devices that offer \ac{I/O} abstraction protocols.
        Physical device handles support the Device Path Protocol for generic path/location information \cite[Section 10.2]{uefi-spec}.
    \item[Service handles] are used for generic hardware unrelated abstractions.
\end{description}

\subsection{\acs{UEFI} Driver Model}

\cite[Section 2.5.2]{uefi-spec} describes the \acs{UEFI} Driver Model, it simplifies the design of device drivers by moving implementation of the device mangement and discovery into the firmware, leaving drivers with only the responsibility to offer interfaces for installation and removal.

We will focus on device drivers, these do not add any new device handles but instead offer protocol abstractions build upon already existing \ac{I/O} abstractions offered by bus drivers.
A driver following the \ac{UEFI} Driver Model is not allowed to interact with any hardware in its entry point and is instead required to install an instace of the Driver Binding Protocol on its own image handle.
The Loaded Image Protocol also offers a field where a driver can provide a function to unload itself.
It may also additionally install the confirguration or diagnostic related protocols.
Runtime drivers usually register a notification function that is triggered when an \ac{OS} loader calls \code{ExitBootServices}, this allows them to translate any allocated memory to their virtual addresses.

The firmware will try to connect device drivers to a controller by iterating over all instances of the Driver Binding Protocol in the handle database and calling the \code{Supported} function of the Driver Binding Protocol on a controller. The device driver then checks whether it supports the controller by for example looking for specific \ac{I/O} abstraction protocols, that it will want to laters use and further abstract.
If the driver supports the device the firmware will call the \code{Start} function of the Driver Binding Protocol to have the driver install its offered protocols on the controller handle.
This is done recursively as the newly installed device driver might now fullfill the requirements for another driver.
The firmware can also call the \code{Stop} of the Driver Binding Protocol function if it wants a driver to uninstall its protocol instance from a controller, an example for this would be another device driver wanting to exclusively manage a controller. This is done by tracking agents of protocols, in other words the drivers who consume a protocol.

\subsection{Systemtable}

The UEFI System Table is an important data structure, it provides access to system configuration information, generic boot and runtime services \cite[Section 3.3]{tianocore-edk2-driver-writer-s-guide}.
It also serves as the entrance in to the \ac{UEFI} environment, as a loaded images receives a only pointer to the system table as well as its image handle through its entry point. Although the Loaded Image Protocol provides and interface to hand optional load options to the image \cite{beyond-bios}.

\TODO{during boot boot and runtime services are available}


\subsubsection{Boot Services}

\ac{UEFI} applcations must use boot services functions to access devices and allocate memory. They are available until an \ac{OS} loader takes control over the system via a call to the boot service \code{ExitBootServices()}, from which on only runtime services are available. \cite[Section 7]{uefi-spec} splits the boot services into five categories:

\begin{description}
    \item [Event, Timer, and Task Priority Services] used to create, close, signal, wait for and check events. Setting timers and raising or restoring task priority levels.
    \item [Memory Allocation Services] to allocate and free pools or whole pages of memory, as well as retrieve the \ac{UEFI} managed memory map.
    \item [Protocol Handler Services] used to install, uninstall and retrieve protocol instance as well as abstractions related to the \ac{UEFI} Driver Model.
    \item [Image Services] to load, unload and start images. Images can also use these to transfer execution back to the firmware or with \code{ExitBootServices()} assume control over the system
    \item [Miscellaneous Services] offer basic memory manipulation, checksum calculation, watchdog timers and monotonic counters.
\end{description}

\subsubsection{Runtime Services}

\TODO{me}

\begin{description}
    \item [Variable Services] used to query, get and set \hyperref[sec:uefi-pi:uefi:variables]{variables}.
    \item [Time Services] used to get and set time as well as a system wakeup timer.
    \item [Virtual Memory Services] relate to enabling virtual memory and translating memory addresses.
    \item [Image Services] to load, unload and start images. Images can also use these to transfer execution back to the firmware or with \code{ExitBootServices()} assume control over the system
    \item [Miscellaneous Runtime Services] offer system reset, a monotonic counter and capsule services. Capsules allow the \ac{OS} to pass data to the firmware, this includes firmware managment related data.
\end{description}

\subsection{Variables}
\label{sec:uefi-pi:uefi:variables}

\ac{UEFI} variables are key/value pairs used to store arbitrary data passed between the \ac{UEFI} firmware and \ac{UEFI} applications.
The data type has to be known beforehand and as such is specified for variables defined in \ac{UEFI}.
The Storage implementation is not specified by \ac{UEFI}, but it must support non\-/volatility, to retain after reboots, or temper resistance if demanded.
Variables are defined by a vendor \ac{GUID}, a name and attributes.
Attributes include their scope (boot, runtime, non-volatile), whether writes require authentication or result in appending data instead of overwriting \cite[Section 8.2]{uefi-spec}.
Architecturally defined \ac{UEFI} variables are called Globally Defined Variables where the vendor \ac{GUID} has the value \code{EFI\_GLOBAL\_VARIABLE} \cite[Section 3.3]{uefi-spec}.

\subsection{Boot Manager}
\label{sec:uefi-pi:uefi:boot-manager}

The \ac{UEFI} boot manager is a firmware component executed after the platform is completely initialized, it decides which \ac{UEFI} drivers or applcations are loaded and when.
The boot behavior is configured through architecturally defined \ac{NVRAM} global variables\cite[Section 3.1]{uefi-spec}.
Each load option entry for a driver or application resides in a variable following the naming scheme of \code{Driver\#\#\#\#} or \code{Boot\#\#\#\#} respectively. Where \code{\#} stands for a hexadecimal digit forming a 4 digit number, requiring leading zeros.
If a firmware implementation allows for the creation of new load options they can then be added to the ordered lists \code{DriverOrder} and \code{BootOrder}, they reference load options and dictate the order in which they are processed.
Driver load options are processed before the boot load options, there also exists the \code{BootNext} variable to override the boot options once.
A general depiction of the \ac{UEFI} boot flow can be seen in \autoref{fig:uefi-boot-sequence}.
Implementations usually allow for an interactive menu, where users can modify the order or boot entries manually \cite[Section 3.1.1]{uefi-spec}.
Boot options are generally first attempted to be loaded through the \code{LoadImage} boot service.
If the device path of a boot option only points to a device instead to the file on a device, it attempts to load a default boot application with the \hyperref[lst:simple-file-system-protocol]{Simple File System Protocol}\cite[Section 3.1.2]{uefi-spec}, for x64 it uses the default path \code{\textbackslash EFI\textbackslash BOOT\textbackslash BOOTX64.EFI} \cite[Section 3.5]{uefi-spec}.




\begin{figure}[htb]%
    \centering%
    \includegraphics[width=0.8\textwidth]{uefi_boot_sequence}%
    \caption{Booting Sequence (taken from \cite[Figure 2-1]{uefi-spec})}%
    \label{fig:uefi-boot-sequence}%
\end{figure}


% !TeX root = ../../thesis.tex

\subsection{Security}

\ac{UEFI} offers security mechanisms restricting what is allowed to modify and be modified on a system.
This involves authentication of ownership over the platform.

\subsubsection{Secure Boot}
\label{sec:uefi-pi:uefi:secure-boot}

Secure Boot provides a secure hand-off from the firmware to 3rd party applications used during the boot process, located on insecure media \cite{tianocore-understanding-uefi-secure-boot-chain} \cite[Sections 32.2 and 32.5.1]{uefi-spec}.
It assumes the firmware to be a trusted entity and all 3rd party software to be untrusted, this includes images from hardware vendors in the \ac{PCI} option \acp{ROM}, bootloader from \ac{OS} vendors, and tools such as the \ac{UEFI} shell \cite{tianocore-understanding-uefi-secure-boot-chain}.
Digital signatures, embedded within the \ac{UEFI} images, can be used to authenticate origin and/or integrity \cite[Section 32.2]{uefi-spec}.
This is done through asymmetric signing. Component providers must sign their executables with their private key and publish the public key.
The public keys are stored in a signature \ac{DB} and the signed executable can be verified against the database before execution.
Multiple signatures can be embedded within the same image \cite[Section 32.2.2]{uefi-spec}.
The signatures are created by first calculating a hash over select parts of the executable and then signing it with a private key.
The output of this hashing is called a digest and the algorithm for obtaining the digest is defined in \cite{microsoft-pe-signature-format}.
Secure Boot also disallows legacy booting through the \ac{CSM}.

Secure Boot is managed through three components: a \ac{PK}, one or more \ac{KEK}, and the signature \acp{DB}.

\begin{description}
    \item[\acl{PK}]
        The \ac{PK} establishes a trust relationship between platform owner and firmware, the public half is enrolled into the firmware.
        The private half represents platform ownership, as it can be used to change or delete the \ac{PK} as well as enroll or modify \acp{KEK}.
    \item[\acl{KEK}]
        The \ac{KEK} establishes a trust relationship between \ac{OS} and firmware, as its private half is used to modify the signature \acfp{DB}.
    \item[Signature \aclp{DB}]
        Signature \acp{DB} contain image hashes and certificates, to either allow or deny execution of associated images.
\end{description}

Internally, these are all implemented by authenticated variables, residing in tamper resistant non-volatile storage \cite[Section 32.3]{uefi-spec}.
The \ac{PK} is a simple variable where the \ac{KEK} and \ac{DB} are implemented through signature list data structures \cite[Section 32.4.1]{uefi-spec}.
The variable services can be used to append entries or to read and write the list as a whole \cite[Sections 32.3.5 and 32.5.3]{uefi-spec}.
The variables are part of the \hyperref[sec:uefi-pi:uefi:variables]{Globally Defined Variables}, for each variable there also exists a variant reserved for default entries.
These can be used by an \ac{OEM} to supply platform\-/defined values, used for example during Secure Boot initialization by a user.
Their contents can be copied to their live versions, to then be used during Secure Boot operation.
The current state of Secure Boot is communicated with a secure variable, which the \ac{OS} can probe \cite[Section 3.3]{uefi-spec}.

Users, who are physically present, may disable Secure Boot as well as enroll default or custom keys via an interactive menu \cite[Section 3.3]{uefi-spec}.

\subsubsection{Firmware Management}

The \nameref{lst:firmware-management-protocol} provides a boot abstraction for authenticated updating and management of the \ac{PF} \cite[Section 23]{uefi-spec}.
The runtime services \code{QueryCapsuleCapabilities()} and \code{CapsuleUpdate()} may be used by the \ac{OS} to pass updates to the firmware in a persistent manner, so that they can be processed on subsequent boots \cite[Section 23.3]{uefi-spec}.
\acp{OEM} also often provide their own ways to process firmware updates, for example via dedicated \ac{USB} ports, which allow to process firmware updates upon boot.
As these are entirely dictated by the platform designer, it is not possible to make vendor independent assessments about their security.

\subsubsection{User Identity Policies}

\ac{UEFI} enables a system to have multiple users with varying levels of privileges.
This may restrict their ability to enroll other users or to boot from select drives \cite[Section 36.1.2]{uefi-spec}.
A trusted environment must be maintained for the integrity of the security identification, by restricting which drivers are loaded and securing the storage of drivers \cite[Section 36.1.4]{uefi-spec}.
% !TeX root = ../../thesis.tex

\subsection{Security}

\ac{UEFI} offers security mechanisms restricting what is allowed to modify and be modified on a system.
This involves authentication of ownership over the platform.

\subsubsection{Secure Boot}
\label{sec:uefi-pi:uefi:secure-boot}

Secure Boot provides a secure hand-off from the firmware to 3rd party applications used during the boot process, located on insecure media \cite{tianocore-understanding-uefi-secure-boot-chain} \cite[Sections 32.2 and 32.5.1]{uefi-spec}.
It assumes the firmware to be a trusted entity and all 3rd party software to be untrusted, this includes images from hardware vendors in the \ac{PCI} option \acp{ROM}, bootloader from \ac{OS} vendors, and tools such as the \ac{UEFI} shell \cite{tianocore-understanding-uefi-secure-boot-chain}.
Digital signatures, embedded within the \ac{UEFI} images, can be used to authenticate origin and/or integrity \cite[Section 32.2]{uefi-spec}.
This is done through asymmetric signing. Component providers must sign their executables with their private key and publish the public key.
The public keys are stored in a signature \ac{DB} and the signed executable can be verified against the database before execution.
Multiple signatures can be embedded within the same image \cite[Section 32.2.2]{uefi-spec}.
The signatures are created by first calculating a hash over select parts of the executable and then signing it with a private key.
The output of this hashing is called a digest and the algorithm for obtaining the digest is defined in \cite{microsoft-pe-signature-format}.
Secure Boot also disallows legacy booting through the \ac{CSM}.

Secure Boot is managed through three components: a \ac{PK}, one or more \ac{KEK}, and the signature \acp{DB}.

\begin{description}
    \item[\acl{PK}]
        The \ac{PK} establishes a trust relationship between platform owner and firmware, the public half is enrolled into the firmware.
        The private half represents platform ownership, as it can be used to change or delete the \ac{PK} as well as enroll or modify \acp{KEK}.
    \item[\acl{KEK}]
        The \ac{KEK} establishes a trust relationship between \ac{OS} and firmware, as its private half is used to modify the signature \acfp{DB}.
    \item[Signature \aclp{DB}]
        Signature \acp{DB} contain image hashes and certificates, to either allow or deny execution of associated images.
\end{description}

Internally, these are all implemented by authenticated variables, residing in tamper resistant non-volatile storage \cite[Section 32.3]{uefi-spec}.
The \ac{PK} is a simple variable where the \ac{KEK} and \ac{DB} are implemented through signature list data structures \cite[Section 32.4.1]{uefi-spec}.
The variable services can be used to append entries or to read and write the list as a whole \cite[Sections 32.3.5 and 32.5.3]{uefi-spec}.
The variables are part of the \hyperref[sec:uefi-pi:uefi:variables]{Globally Defined Variables}, for each variable there also exists a variant reserved for default entries.
These can be used by an \ac{OEM} to supply platform\-/defined values, used for example during Secure Boot initialization by a user.
Their contents can be copied to their live versions, to then be used during Secure Boot operation.
The current state of Secure Boot is communicated with a secure variable, which the \ac{OS} can probe \cite[Section 3.3]{uefi-spec}.

Users, who are physically present, may disable Secure Boot as well as enroll default or custom keys via an interactive menu \cite[Section 3.3]{uefi-spec}.

\subsubsection{Firmware Management}

The \nameref{lst:firmware-management-protocol} provides a boot abstraction for authenticated updating and management of the \ac{PF} \cite[Section 23]{uefi-spec}.
The runtime services \code{QueryCapsuleCapabilities()} and \code{CapsuleUpdate()} may be used by the \ac{OS} to pass updates to the firmware in a persistent manner, so that they can be processed on subsequent boots \cite[Section 23.3]{uefi-spec}.
\acp{OEM} also often provide their own ways to process firmware updates, for example via dedicated \ac{USB} ports, which allow to process firmware updates upon boot.
As these are entirely dictated by the platform designer, it is not possible to make vendor independent assessments about their security.

\subsubsection{User Identity Policies}

\ac{UEFI} enables a system to have multiple users with varying levels of privileges.
This may restrict their ability to enroll other users or to boot from select drives \cite[Section 36.1.2]{uefi-spec}.
A trusted environment must be maintained for the integrity of the security identification, by restricting which drivers are loaded and securing the storage of drivers \cite[Section 36.1.4]{uefi-spec}.
\clearpage
% !TeX root = ../thesis.tex

\section{\acf{PI}}

\subsection{Boot Sequence}

% https://edk2-docs.gitbook.io/edk-ii-build-specification/2_design_discussion/23_boot_sequence

focus will be on dxe and transient system load
\autoref{fig:pi-phases}



\begin{figure}[htb]%
    \centering%
    \includegraphics[width=\textwidth]{pi_boot_sequence}%
    \caption{\ac{PI} Architecture Firmware Phases \cite[Figure 2-1]{pi-spec}}%
    \label{fig:pi-phases}%
\end{figure}


\begin{enumerate}
    \item{\acf{SEC}}
    The Security phase is the first code executed by the CPU, it is uncompressed and executed directly from flash. It consists of platform specific assembly.

    \begin{itemize}
        \item  Handles all platform restart events (power on, wakup from sleep, etc)
        \item  Creates a temporary memory state by configuring the CPU Cache as RAM (CAR) "no evictions mode"
        \item  Serves as the root of trust in the system
        \item  Passes handoff information to the Pre-EFI Initialization (PEI) Foundation
    \end{itemize}


    Since the CPU doesn't know about UEFI or BIOS the initial step is exactly the same, it starts in 16-bit real mode and fetches it's first instruction from `CS = 0xF000` and `IP = 0xFFF0` but instead of shifting `CS` left by four bits and adding `IP`, the `CS` base register is initialized to `0xFFFF'0000`. So the first instruction is fetched from the physical address `0xFFFF'FFF0` (`0xFFFF'0000 + 0xFFF0`). The CS base address remains at this initial value until the CS selector register is loaded by software (e.g. far jump or call instruction)

    \begin{itemize}
        \item Populates Reset Vector Data structure
        \item Saves Built-in self-test (BIST) status
        \item Enables protected mode (16 bit -> 32 bit)
        \item Configures temporary RAM (not only limited in processor cache) by using MTRR to configure CAR.
    \end{itemize}

    Passing of handoff information to the PEI phase:

    \lstinline{typedef VOID EFIAPI (*EFI_PEI_CORE_ENTRY_POINT)(IN CONST EFI_SEC_PEI_HAND_OFF *SecCoreData, IN CONST EFI_PEI_PPI_DESCRIPTOR *PpiList);}

    SEC Core Data:
    \begin{itemize}
        \item Points to a data structure containing information about the operating environment:
        \item Location and size of the temporary RAM
        \item Location of the stack (in temporary RAM)
        \item Location of the Boot Firmware Volume (BFV)
    \end{itemize}


    PPI list:
    \begin{itemize}

        \item Temporary RAM support PPI

              An optional service that moves temporary RAM contents to permanent RAM.

        \item SEC platform information PPI

              An optional service that abstracts platform-specific information to locate the PEIM dispatch order and maximum stack capabilities.
    \end{itemize}


    % The SEC phase is the first phase in the UEFI boot process. It is specified in \cite[Vol 1, Section 13 ]{pi-spec}. Under its responsibilities fall setting up temporary memory used for the stack and the establishment of the system's root of trust which is a foundation for all secure operations. Inductive security designs rely on this root of trust to build a chain of trust by having a module verify the integrity of its subsequent module.
    ref to PSP

    inductive security design
    integrity of next module checked by the previous module

    handles all platform restart events
    applying power to system from unpowered state
    restarting from active state
    receiving exception conditions

    creates temporary memory store
    possibly CPU \ac{CAR}
    cache behaves as linear store of memory
    no evictions mode
    every memory access is a hit
    eviction not supported as main memory is not set up yet and would lead to platform failure


    final step
    Pass handoff information to the \ac{PEI} Foundation
    % what is the PEI foundation
    \begin{itemize}
        \item state of platform
        \item location and size of the \ac{BFV}
        \item location and size of the temporary RAM
        \item location and size of the stack
        \item optionally one or more \acp{HOB} via the \ac{SEC} \ac{HOB} Data \ac{PPI}
    \end{itemize}


    Part of this process is a so called \ac{HOB} with a function pointer to a procedure to verify PE modules.

    SEC Platform Information PPI
    information about the health of the processor

    SEC HOB Data PPI

    \item{\acf{PEI}}
    Configures a system meeting the minimum prerequisites for the Driver Execution (DXE) phase, which is generally a linear array of RAM large enough for successful execution.

    PEI provides a framework allowing vendors to supply initialization modules for each functionally distinct piece of system hardware which must be initialized before the DXE phase.

    PEI design goals of the PI architecture:
    \begin{itemize}
        \item Maintenance of the "chain of trust", includes protection and authorization of PEI modules
        \item Provide a core PEI module
        \item Independent developement of intialization modules
    \end{itemize}
    The PEI phase consists of the PEI Foundation core and specialized plug-ins known as Pre-EFI Initialization Modules (PEIMs).

    Since the PEI phase is very early in the boot process it can't assume reasonable amounts of RAM so the features are limited:
    \begin{itemize}
        \item Locating, validating and dispatching PEIMs
        \item Communication between PEIMs
        \item Providing Hand-Off Data for DXE phase
        \item Initializing some permanent memory complement
        \item Describing the memory in Hand-Off Blocks (HOBs)
        \item Describing the firmware volume locations in HOBs
        \item Passing control into the Driver Execution Environment (DXE) phase
        \item Discover boot mode and possibly resume from sleep state
    \end{itemize}
    PEI Service Table visible to all PEIMs in the system, a pointer to this table is passed as an argument via the PEIM entry point, it is also part of each PEIM-to-PEIM Interface (PPI).


    % | Service                  | Description                                                                                                                                 |
    % | ------------------------ | ------------------------------------------------------------------------------------------------------------------------------------------- |
    % | PPI Services             | Manages PPIs to facilitate intermodule calls between PEIMs. Interfaces are installed and tracked on a database maintained in temporary RAM. |
    % | Boot Mode Services       | Manages the boot mode (S3, S5, normal boot, diagnostics, etc.) of the system                                                                |
    % | HOB Services             | Creates data structures called Hand-Off Blocks (HOBs) that are used to pass information to the next phase of the PI Architecture.           |
    % | Firmware Volume Services | Finds PEIMs and other firmware files in the firmware volumes                                                                                |
    % | PEI Memory Services      | Provides a collection of memory management services for use both before and after permanent memory has been discovered                      |
    % | Status Code Services     | Provides common progress and error code reporting services (for example, port 080h or a serial port for simple text output for debug).      |
    % | Reset Services           | Provides a common means by which to initiate a warm or cold restart of the system.                                                          |

    % #### PEI Foundation/Core

    PEI Foundation code is portable across all platforms of a given instruction-set. The set of exposed services is the same across different microarchitextures and allows PEIMs to be written in C.

    - Dispatches PEIMs
    - Maintains boot mode
    - Initializes permanent memory
    - Invokes DXE loader

    % #### PEI Dispatcher

    The PEI Dispatcher evaluates dependencies of PEIMs in the firmware volume, these dependencies are PPIs. The Dispatcher holds internal state machines to check dependencies of PEIMs, it starts executing PEIMs whose dependencies are statisfied to build up dependencies of other PEIMs, this is done until the dispatcher cannot invoke any more PEIMs. Then the DXE Initial Program Loader (IPL) PPI is invoked to pass control to the DXE phase.

    % #### Pre-EFI Initialization Modules (PEIMs)
    PEIMs are specialized drivers that personalize the PEI Foundation to the platform. They are analogus to DXE driver and generally correspond to the components being initialized. It is strongly recommended that PEIMS do only the minimum necessary work to initialize the system to a state that meets the prerequisites of the DXE phase. PEIMs reside in firmware volumes (FVs).

    % #### PEIM-to-PEIM Interfaces (PPIs)
    PEIMs communicate with each other using a structure called PPI. A PPI is a GUID pointer pair. The GUID is used to identifiy a certain service and the pointer provides access to data structures and services of the PPI.

    % There are two kinds of PPIs:
    % - Architectural PPIs
    % - Additional PPIs

    An architectural PPI is described in the PEI Core Interface Specification (CIS) and the GUID is known to the PEI Foundation. They typically provide a common interface to the PEI Foundation to a service with platform specific implementation.

    An additional PPI is important for interoperability but isn't required by the PEI Foundation, they can be classified as mandatory or optional.


    \begin{itemize}
        \item init permanent memory
        \item describe memory in \acp{HOB}
        \item describe \ac{FV} in \acp{HOB}
        \item pass control to \ac{DXE}
    \end{itemize}

    crisis recovery (what is this?)
    resuming from S3 sleep state
    linear array of RAM
    \ac{PEIM} provides a framework to allow vendors to supply separate initialization modules for
    each functionally distinct piece of system hardware that must be initialized prior to the DXE phase \cite{pi-spec}

    % design goals
    maintenance of chain of trust, protection against unauthorized updates to the PEI phase or modules
    authentication of the PEI Foundation and its modules
    provide core PEI module (PEI foundation) processor architecture independent, supports add-in moudles from vendors for processors, chipsets, RAM

    % what it does
    Locating, validating, and dispatching PEIMs
    Facilitating communication between PEIMs
    Providing handoff data to subsequent phases

    \item{\acf{DXE}}


    % - DXE Foundation/Core
    % - DXE Dispatcher
    % - DXE Drivers

    % #### DXE Foundation

    The DXE Foundation produces a set of Boot, Runtime and DXE Services and exposes them through handle databases in the EFI System Table. It is designed to be completely portable, independent of processor, chipset and platform. The only dependent of the Hand-Off Blocks from the PEI phase, after these are processed the all prior phases can be unloaded.

    % #### DXE Dispatcher

    The DXE Dispatcher discovers DXE drivers within the Firmware Volume (FV) and executes them in the correct order, respecting their dependencies towards each other. The Firmware Volume file format allows the DXE driver images to be packaged with expressions about their dependencies. Since the DXE Drivers are PE/COFF images the dispatcher comes with an apropriate loader to load and execute the image format.

    % #### DXE Drivers

    % - Drivers that execute very early in the DXE phase
    % - Drivers that comply with the UEFI Driver Model

    The DXE Drivers are responsible for initializing the processor, chipset,
    and platform components as well as providing software abstractions for console and
    boot devices in the form of services.

    dxe core/foundation
    platform independent
    is implementation of UEFI
    UEFI Boot Services
    UEFI Runtime Services
    DXE Services

    dxe dispatcher
    discover drivers stored in firmware volumes and execute in proper order
    apriori file optionally in FV or depex of driver
    after dispatching all drivers in the dispatch queue hands control over to BDS

    dxe drivers
    init processor, chipset and platform
    produce arichtectural protocols and \ac{I/O} abstractions for consoles and boot devices

    % responsibilities
    initializing the processor, chipset, and platform components
    providing software abstractions for system services, console devices, and boot devices.

    \item{\acf{BDS}}
    The DXE Foundation will hand control to the BDS Architectural Protocol after all of the DXE drivers whose dependencies have been satisfied have been loaded and executed by the DXE Dispatcher.

    % - Initializing console devices based on the ConIn, ConOut, and StdErr environment variables
    % - Loading device drivers listed in the DriverOrder environment variables
    % - Attempting to load and execute boot selections list from the BootOrder environment variables

    During the BDS phase new Firmware Volumes (FV) might be discovered and control is once again handed to the DXE Dispatcher to load drivers found on these additional volumes.


    DXE arichtectural protocol
    one function entry
    platform boot

    attempts to connect boot devices required to load the os
    discovers volumes containing new drivers
    calls DXE dispatcher
    doesnt return when successfully booting OS

    UEFI itself only specifies the NVRAM variables used in selecting boot options
    leaves the implementation of the menu system as value added implementation space \cite{uefi-spec}

    \cite{pi-spec}

    \begin{itemize}
        \item Initializing console devices
        \item Loading device drivers
        \item Attempting to load and execute boot selections
    \end{itemize}

    \item{\acf{TSL}}

    The Transient System Load (TSL) is primarily the OS vendor provided boot loader. Both the TSL and the Runtime Services (RT) phases may allow access to persistent content, via UEFI drivers and UEFI applications. Drivers in this category include PCI Option ROMs.

    This phase ends when an OS boot loader calls 'ExitBootServices()'.

    boottime and runtime services/driver
    bootloader
    \cite[13.3 System Partition]{uefi-spec}
    \cite[3.5.1.1]{uefi-spec}

    ExitBootServices()

    \item{\acf{RT}}
    Boot service drivers have been unloaded and only runtime services are accessible.


    runtime services/driver

    \item{\acf{AL}}
    The After Life (AL) phase consists of persistent UEFI drivers used for storing the state of the system during the OS orderly shutdown, sleep, hibernate or restart processes.

    hibernation
    sleep

\end{enumerate}

\subsection{\acs{UEFI}/\acs{PI} Firmware Images}

Firmware Images are stored in Flash Devices (FD), a Firmware Volume (FV) serves as file level interface. Usually multiple FVs are present in a single FD but a signle FV can also be distributed via multiple FDs.
A FV is formatted with a binary file system, typically with Firmware File System (FFS).

In a FFS modules are stored as files, they can be executed at the fixed address from Read Only Memory (ROM) or through relocation in loaded memory. Within a file are multiple sections which then contain the "leaf" images. These are for example PE32 images.

% https://edk2-docs.gitbook.io/edk-ii-build-specification/2_design_discussion/22_uefipi_firmware_images
\cite[Volume 3, 2.1]{pi-spec}

\includegraphics[width=\textwidth]{flash_device}
\ac{FD}
persistent
physical device
contains firmware code and/or data
typically flash
may be divided into smaller pieces to form multiple logical firmware devices
multiple physical firmware devices may be aggregated into one larger logical firmware device

\acf{FV}
logical device
organized into a file system
attributes such as
- size
- formatting
- read/write access

\acf{FFS}
organization of files and free space
no dierectory hierarchy
all files flat in root dir
parsing requires walking for beginning to end

firmware files
types
% PEI_CORE
% PEIM
% DXE_CORE
% DRIVER
% FIRMWARE_VOLUME_IMAGE
% FREEFORM

some file types are sub-divided in file sections

file sections can be either
encapsulation or leaf
leaf sections such as
PE32
% DXE_DEPEX
% PEI_DEPEX
RAW
VERSION
TE

dxe drivers files
contain one PE32 executable section
may contain version section
may contain dxe depex section

freeform files
can contain any combination of sections

PEI phase Service Table
FfsFindNextFile, FfsFindFileByName and FfsGetFileInfo

DXE phase
% EFI_FIRMWARE_VOLUME2_PROTOCOL

depex

\cite{tianocore-edk2-build-spec}
% !TeX root = ../../thesis.tex

\subsection{Security}

\subsubsection{Hardware Validated Boot}
Secure Boot relies for the firmware as its root of trust, hardware validated boot shifts this trusts out of the firmware image into hardware.
amd
% https://ebrary.net/24869/computer_science/secure_technology
% https://www.amd.com/system/files/documents/amd-security-white-paper.pdf
intel
% https://edk2-docs.gitbook.io/understanding-the-uefi-secure-boot-chain/secure_boot_chain_in_uefi/intel_boot_guard

with pi offering security PPI and dxe protocols for this

PEI Guided Section Extraction PPI
Security PPI

Guided Section Extraction Protocol
Security Architecture Protocol
Security2 Architecture Protocol


\subsubsection{Firmware Protection}

The \ac{DXE} phase also offers drivers to register notification
% End of Dxe Event
% From SEC through the signaling of this event, all of the components should be under the authority of
% the platform manufacturer and not have to worry about interaction or corruption by 3rd party
% extensible modules such as UEFI drivers and UEFI applications.

% Platform may choose to lock certain resources or disable certain interfaces prior to executing third
% party extensible modules. Transition from the environment where all of the components are under
% the authority of the platform manufacturer to the environment where third party modules are
% executed is a two-step process:

% 1. End of DXE Event is signaled. This event presents the last opportunity to use resources or
% interfaces that are going to be locked or disabled in anticipation of the invocation of 3rd party
% extensible modules.
% 2. DXE SMM Ready to Lock Protocol is installed. PI modules that need to lock or protect their
% resources in anticipation of the invocation of 3rd party extensible modules should register for
% notification on installation of this protocol and effect the appropriate protections in their
% notification handlers


% https://papers.vx-underground.org/papers/Other/Advanced%20Malware/UEFI%20Secure%20Boot%20in%20Modern%20Computer%20Security%20Solutions.pdf
% NIST 800-147 BIOS Protection Guidelines [15]
% NIST 800-147B BIOS Protection Guidelines for Servers [16]
% NIST 800-155 BIOS Integrity Measurement Guidelines [17]


% https://eclypsium.com/2019/10/23/protecting-system-firmware-storage/

DXE SMM Ready to Lock Vol4




flash device security







\subsubsection{TPM}
\subsubsection{TPM measurements}
% https://tianocore-docs.github.io/edk2-TrustedBootChain/release-1.00/
% https://tianocore-docs.github.io/edk2-TrustedBootChain/release-1.00/3_TCG_Trusted_Boot_Chain_in_EDKII.html
% https://tianocore-docs.github.io/edk2-TrustedBootChain/release-1.00/6_Checklist_for_Platform_Developers.html
% https://learn.microsoft.com/en-us/windows/security/information-protection/tpm/tpm-fundamentals
% https://learn.microsoft.com/en-us/windows/security/information-protection/tpm/trusted-platform-module-overview
% what is TPM
A \acf{TPM} is a system component which enables trust in computing platforms
helps verify if the Trusted Computing Base has been compromised
securely storing passwords, certificates and encryption keys in separate state to host
only communicating through a well defined interface.
store platform measurements that help ensure that the platform remains trustworthy
authentication
attestation
hardware and software implementations
software special mode shielding TPM resources from normal execution
\cite{tcg-tpm-summary}
\cite{tcg-tpm-library-part1-architecture}

% what is done with the measurements
% https://learn.microsoft.com/en-us/windows/security/information-protection/bitlocker/bitlocker-overview
how are they used
works with bitlocker to protect user data
ensure computer has not been tampered with while offline

% what is measured
statically configured, unchangeable data
not dynamic and changeable across the boot,
\cite{tianocore-trusted-boot-chain}

\begin{table}
    \centering
    \begin{tabular}{ c|p{30em}  }
        \acs{PCR} Index & \multicolumn{1}{c}{\acs{PCR} Usage}                                                                      \\
        \hline
        0               & SRTM, BIOS, Host Platform Extensions, Embedded Option ROMs and PI Drivers                                \\
        \hline
        1               & Host Platform Configuration                                                                              \\
        \hline
        2               & UEFI driver and application Code                                                                         \\
        \hline
        3               & UEFI driver and application Configuration and Data                                                       \\
        \hline
        4               & UEFI Boot Manager Code (usually the MBR) and Boot Attempts                                               \\
        \hline
        5               & Boot Manager Code Configuration and Data (for use by the Boot Manager Code) and \ac{GPT}/Partition Table \\
        \hline
        6               & Host Platform Manufacturer Specific                                                                      \\
        \hline
        7               & Secure Boot Policy                                                                                       \\
        \hline
        8               & First \ac{NTFS} boot sector (volume boot record)                                                         \\
        \hline
        9               & Remaining \ac{NTFS} boot sectors (volume boot record)                                                    \\
        \hline
        10              & Boot Manager                                                                                             \\
        \hline
        11              & BitLocker Access Control                                                                                 \\
    \end{tabular}

    % \label{}
    \caption{\cite{tcg-pc-client-platform-firmware-profile-spec, windows-internals-6-part2}}
\end{table}


% https://tianocore-docs.github.io/edk2-TrustedBootChain/release-1.00/media/image2.png

% https://tianocore-docs.github.io/edk2-TrustedBootChain/release-1.00/3_TCG_Trusted_Boot_Chain_in_EDKII.html
% Table 2 PCR usage (simple rules)

% https://tianocore-docs.github.io/edk2-TrustedBootChain/release-1.00/media/image3.png

% when is it measured
\cite{tianocore-trusted-boot-chain}

% where is it measured
\ac{TCG}2 Protocol
Trusted Computing Group 2 (TCG2) Protocol \cite[Section 6.7.3]{tcg-efi-protocol-spec}

% secret storage, seal and unseal

\clearpage
% !TeX root = ../../thesis.tex

\section{TPM}

\TODO{me}
% TPM itself not part ot the uefi specifications but its interaction in the \cite

% https://tianocore-docs.github.io/edk2-TrustedBootChain/release-1.00/
% https://tianocore-docs.github.io/edk2-TrustedBootChain/release-1.00/3_TCG_Trusted_Boot_Chain_in_EDKII.html
% https://tianocore-docs.github.io/edk2-TrustedBootChain/release-1.00/6_Checklist_for_Platform_Developers.html
% https://learn.microsoft.com/en-us/windows/security/information-protection/tpm/tpm-fundamentals
% https://learn.microsoft.com/en-us/windows/security/information-protection/tpm/trusted-platform-module-overview
% what is TPM
A \acf{TPM} is a system component which enables trust in computing platforms
helps verify if the Trusted Computing Base has been compromised
securely storing passwords, certificates and encryption keys in separate state to host
only communicating through a well defined interface.
store platform measurements that help ensure that the platform remains trustworthy
authentication
attestation
hardware and software implementations
software special mode shielding TPM resources from normal execution
\cite{tcg-tpm-summary}
\cite{tcg-tpm-library-part1-architecture}

% what is done with the measurements
% https://learn.microsoft.com/en-us/windows/security/information-protection/bitlocker/bitlocker-overview
how are they used
works with bitlocker to protect user data
ensure computer has not been tampered with while offline

% what is measured
statically configured, unchangeable data
not dynamic and changeable across the boot,
\cite{tianocore-trusted-boot-chain}

\begin{table}
    \centering
    \begin{tabular}{ c|p{30em}  }
        \acs{PCR} Index & \multicolumn{1}{c}{\acs{PCR} Usage}                                                                      \\
        \hline
        0               & SRTM, BIOS, Host Platform Extensions, Embedded Option ROMs and PI Drivers                                \\
        \hline
        1               & Host Platform Configuration                                                                              \\
        \hline
        2               & UEFI driver and application Code                                                                         \\
        \hline
        3               & UEFI driver and application Configuration and Data                                                       \\
        \hline
        4               & UEFI Boot Manager Code (usually the MBR) and Boot Attempts                                               \\
        \hline
        5               & Boot Manager Code Configuration and Data (for use by the Boot Manager Code) and \ac{GPT}/Partition Table \\
        \hline
        6               & Host Platform Manufacturer Specific                                                                      \\
        \hline
        7               & Secure Boot Policy                                                                                       \\
        \hline
        8               & First \ac{NTFS} boot sector (volume boot record)                                                         \\
        \hline
        9               & Remaining \ac{NTFS} boot sectors (volume boot record)                                                    \\
        \hline
        10              & Boot Manager                                                                                             \\
        \hline
        11              & BitLocker Access Control                                                                                 \\
    \end{tabular}

    % \label{}
    \caption{\cite{tcg-pc-client-platform-firmware-profile-spec, windows-internals-6-part2}}
\end{table}

\subsection{Measurements}

\TODO{me}


% https://tianocore-docs.github.io/edk2-TrustedBootChain/release-1.00/media/image2.png

% https://tianocore-docs.github.io/edk2-TrustedBootChain/release-1.00/3_TCG_Trusted_Boot_Chain_in_EDKII.html
% Table 2 PCR usage (simple rules)

% https://tianocore-docs.github.io/edk2-TrustedBootChain/release-1.00/media/image3.png

% when is it measured
\cite{tianocore-trusted-boot-chain}

% where is it measured
\ac{TCG}2 Protocol
Trusted Computing Group 2 (TCG2) Protocol \cite[Section 6.7.3]{tcg-efi-protocol-spec}

% secret storage, seal and unseal

\clearpage
% !TeX root = ../../thesis.tex

\section{\acs{UEFI} Shell}

Part of the family of \ac{UEFI} specifications is a shell specification which defines a feature rich \ac{UEFI} shell application to interact with the \ac{UEFI} environment \cite[Section 1.1]{uefi-shell-spec}.
It offers commands relating to boot and general configuration, device and driver management, file system access, networking \cite[Section 5.1]{uefi-shell-spec} and scripting \cite[Section 4]{uefi-shell-spec}.
The \ac{UEFI} shell may already be part of the boot options but can always be supplied on removable media in the default boot path.
During initialization the shell produces default mappings for file systems and block devices, this defines names that can be used interchangably with their device paths when issuing shell commands \cite[Section 3.7.2]{uefi-shell-spec}.
These mappings are designed to be consistent across reboots as long as the hardware configuration stays the same, they are comparable to Windows partition letters \cite[Appendix A]{uefi-shell-spec}.
It also produces the initial output of what is equivalent to the invocation of the commands \program{ver} and \program{map} \cite[Section 3.3]{uefi-shell-spec}.
\program{ver} displays the version of the \ac{UEFI} specification the firmware conforms to, while map shows the current device mapping \cite[Section 5.3]{uefi-shell-spec}.
\autoref{fig:uefi-shell} depicts an exemplary output of the \ac{EDK} II \ac{UEFI} shell.

The \ac{UEFI} shell is a great tool to visualize the \ac{UEFI} environment.
With the \program{devtree} command we can see all handles of devices complying with the \ac{UEFI} driver model.
It is also a great reference to understand device paths.


When we inspect the mapping table we can see \lstinline{FSx:} and \lstinline{BLKx:} aliases, \lstinline{FSx:} maps to file systems and \lstinline{BLKx:} to block devices.
This identification is performed via instances of the \hyperref[lst:simple-file-system-protocol]{Simple File System Protocol} and \TODO{double check} Block \ac{I/O} Protocol.
% explain Simple File System Protocol
The \hyperref[lst:simple-file-system-protocol]{Simple File System Protocol} \cite[Section 13.4]{uefi-spec} provides, together with the \hyperref[lst:simple-file-system-protocol]{File Protocol}, file-type access to the device it is installed on \cite[Section 13.5]{uefi-spec}.
The two protocols are independent of the underlying file system the media is formatted with.



\TODO{me}

\begin{figure}[htb]
    \centering
    \includegraphics[width=1.0\textwidth]{uefi_shell.png}
    \caption{\ac{UEFI} command prompt}
    \label{fig:uefi-shell}
\end{figure}


\section{\acs{EDK} II}

\ac{EDK} II, maintained by \emph{TianoCore}, is an open source implementation of \ac{UEFI}, offering a modern, feature-rich, cross-platform firmware development environment for the \ac{UEFI} and \ac{PI} specifications \cite{tianocore}.
It can be used to build modules of all types defined in the \ac{PI} and \ac{UEFI} specification and supports the generation of \ac{PF} images, option\-/\acp{ROM} and bootable media.
On top of implementing the \ac{PI} and \ac{UEFI} specifications, it defines a lot of helpful libraries and protocols that can be used to simplify the development process.
The build process is flexible as it can use different compilers such as \program{GCC} and \program{MSVC}.

\emph{TianoCore} also offers material to learn about developing applications, drivers, firmware, and a general understanding about the \ac{UEFI} ecosystem.
% !TeX root = ../thesis.tex

\chapter{\acf{TPM}}

With the \ac{TPM} the \ac{TCG} specifies a system component designated for security\-/related functions, providing the ability to establish trust in a system \cite{microsoft-windows-trusted-platform-module-overview}.
Its implementation can be accomplished through dedicated hardware or by using the \ac{CPU}'s isolated \ac{SMM} \cite[Section 9.3]{tcg-tpm-library-part1-architecture}.
Asides the generation and secure storage of cryptographic keys, the \ac{TPM} can be used for system integrity measurements.
Boot code is measured into the \ac{TPM} by the \ac{PF}, providing evidence over the initialization process and making it possible to detect deviations \cite{microsoft-windows-trusted-platform-module-overview}.

\section{\acfp{PCR}}
\label{sec:tpm:pcr}

\cite{tcg-tpm-library-part1-architecture} defines \acp{PCR} to store the system integrity measurements.
The registers can only be modified in two ways, either through a complete \ac{TPM} reset or by extending their values.
Extending a \ac{PCR} is done by concatenating the hashed measurements together with the current \ac{PCR} values to form the new contents.
This creates a chain of measurements where from one diverging measurement on, all subsequent \ac{PCR} extensions result in entirely differnt values.

\begin{equation}
    PCR[i]_{(new)}\:=\:Hash(PCR[i]_{(old)}\:||\:Hash(Measurement))
\end{equation}

The \ac{TPM} is a passive system components, relying on the host processor to perform measurements and extend the \acp{PCR}.
The measurement chain starts with a point called the \ac{CRTM}, consisting of the first instructions executed to establish a chain of trust.
A Root of trust in a system is an element that must be trusted as its behavior is non\-/verifiable \cite{tcg-tpm-library-part1-architecture}.
\cite[3.2.2]{tcg-pc-client-platform-firmware-profile-spec} requires this chain to start in an immutable portion of the \ac{PI} process, the \ac{SRTM}.
\cite[3.2.3.1]{tcg-pc-client-platform-firmware-profile-spec} defines the \ac{PF} to be composed of a Boot Block and the \ac{UEFI} firmware, the Boot Block consists of the \ac{SEC} and \ac{PEI} phase as well as the \ac{IBB}.
The Boot Block forms the \ac{SRTM}, while the \ac{UEFI} firmware is only part of the chain of trust by being measured from the \ac{SRTM}.
The \ac{RTM} can either start with the \ac{SRTM} measuring itself or a \ac{H-CRTM} measuring the \ac{SRTM}.
It falls under the responsibility of the \ac{PF} to perform the integrity measurements \cite{tcg-pc-client-platform-firmware-profile-spec}, different parts of the boot process are measured into separate \acp{PCR}.
\autoref{fig:pf-measurement-flow} shows a high level measurement flow.
\autoref{tab:pcr-usage} shows \ac{PCR} indexes and their type of content, that is measured, relevant for this thesis.

Interaction with the \ac{TPM} is done via a well\-/defined interface, for external chips this is done over hardware busses such as \ac{LPC} or \ac{SPI}.
\ac{TCG} specifies the \hyperref[lst:tcg2-protocol]{\ac{TCG}2 Protocol} for the \ac{UEFI} environment, providing an abstracted communication interface indepdenent of the underlying implementation.

\begin{figure}[htb]
    \centering
    \includegraphics[width=1.0\textwidth]{tpm/tpm_measurements}
    \caption{\ac{PF} Measurement Flow (taken from \cite[Figure 3]{tianocore-trusted-boot-chain})}
    \label{fig:pf-measurement-flow}
\end{figure}



\section{Sealing/Unsealing}

The chain of measurements can be used to attest for a trusted system state.
The \ac{TPM} can be given data, such as a cryptographic key, in a state that is assumed to be trusted.
This state is reflected by the current \acp{PCR}, consisting of the collection of lead\-/up measurements.
The \ac{TPM} then seals the data with a policy that describes which \ac{PCR} indexes to use and/or proof of authentication through a \ac{PIN} or passphrase.
The data can then only be unsealed on subsequent boots when the system is in the same trusted state that it was in, when the data was sealed.
Any modification of the boot code will be reflected in a deviation of \ac{PCR} values leaving the \ac{TPM} unable to unseal the data.
\TODO{CITE}

\begin{table}
    \centering
    \begin{tabular}{cp{30em}}
        \toprule
        \multicolumn{1}{c}{{\bfseries \acs{PCR} Index}} & \multicolumn{1}{c}{{\bfseries Measurements}}                                                             \\
        \cmidrule[0.4pt](r){1-1}
        \cmidrule[0.4pt](l){2-2}
        \multirow{2}{*}{\textbf{0}}                     & SRTM, BIOS, Host Platform Extensions, Embedded Option ROMs and PI Drivers                                \\
        \cmidrule[0.4pt](r){1-1}
        \cmidrule[0.4pt](l){2-2}
        \textbf{1}                                      & Host Platform Configuration                                                                              \\
        \cmidrule[0.4pt](r){1-1}
        \cmidrule[0.4pt](l){2-2}
        \textbf{2}                                      & UEFI driver and application Code                                                                         \\
        \cmidrule[0.4pt](r){1-1}
        \cmidrule[0.4pt](l){2-2}
        \textbf{3}                                      & UEFI driver and application Configuration and Data                                                       \\
        \cmidrule[0.4pt](r){1-1}
        \cmidrule[0.4pt](l){2-2}
        \textbf{4}                                      & UEFI Boot Manager Code (usually the MBR) and Boot Attempts                                               \\
        \cmidrule[0.4pt](r){1-1}
        \cmidrule[0.4pt](l){2-2}
        \multirow{2}{*}{\textbf{5}}                     & Boot Manager Code Configuration and Data (for use by the Boot Manager Code) and \ac{GPT}/Partition Table \\
        \cmidrule[0.4pt](r){1-1}
        \cmidrule[0.4pt](l){2-2}
        \textbf{6}                                      & Host Platform Manufacturer Specific                                                                      \\
        \cmidrule[0.4pt](r){1-1}
        \cmidrule[0.4pt](l){2-2}
        \textbf{7}                                      & Secure Boot Policy                                                                                       \\
        \cmidrule[0.4pt](r){1-1}
        \cmidrule[0.4pt](l){2-2}
        \multirow{2}{*}{\textbf{8}}                     & First \ac{NTFS} boot sector (volume boot record)                                                         \\
        \cmidrule[0.4pt](r){1-1}
        \cmidrule[0.4pt](l){2-2}
        \textbf{9}                                      & Remaining \ac{NTFS} boot sectors (volume boot record)                                                    \\
        \cmidrule[0.4pt](r){1-1}
        \cmidrule[0.4pt](l){2-2}
        \textbf{10}                                     & Boot Manager                                                                                             \\
        \cmidrule[0.4pt](r){1-1}
        \cmidrule[0.4pt](l){2-2}
        \textbf{11}                                     & BitLocker Access Control                                                                                 \\
        \bottomrule
    \end{tabular}%
    \caption{\ac{PCR} Usage (taken from \cite[Table 1]{tcg-pc-client-platform-firmware-profile-spec} and \cite[Table 9-2]{windows-internals-6-part2})}%
    \label{tab:pcr-usage}%
\end{table}
% !TeX root = ../../thesis.tex


\chapter{Windows 11}
\label{sec:windows}

Windows 11 is, at the time of writing, the latest iteration in the line of desktop \acp{OS} from Microsoft.
It is built upon the foundation of Windows 10 and, as such, shares many security settings and policies with its predecessor \cite{microsoft-windows-11-overview}.
Simultaneously, it raises the requirements of hardware\-/related security features.
It does not support being booted from legacy \ac{BIOS} anymore and requires the \ac{PF} to conform to at least \ac{UEFI} version 2.3.1 and to be capable of Secure Boot.
It also requires the presence of a \ac{TPM} version 2.0 \cite{microsoft-windows-minimum-hardware-requirements-overview}.

\section{\acs{UEFI}}

To be able to analyze \ac{UEFI} threats against Windows 11 it is important to understand how Windows interacts with the \ac{UEFI} environment.

\subsection{Installation}

The interaction with \ac{UEFI} begins with the installation process and the partitioning of the hard drive Windows is installed onto.
% creates at least four partitions
When the Windows Installer is launched, it creates at least four \ac{GPT} partitions on the target hard drive: the \acf{ESP}, a recovery partition, a partition reserved for temporary storage, and the boot partition containing the system files.
Two copies of the \emph{Windows Boot Manager} \program{bootmgfw.efi} are placed on the \ac{ESP}, one under \program{EFI\brackslash Boot\brackslash bootx64.efi} for the default boot behavior (i.e., booting from the installed hard drive) and one under \program{EFI\brackslash Microsoft\brackslash Boot\brackslash bootmgfw.efi} alongside boot resources such as the \ac{BCD}.
The path of the latter boot manager is saved in a boot load option variable entry \code{Boot\#\#\#\#}, which is then added to the \code{BootOrder} list variable.
The boot load option contains optional data consisting of a \ac{GUID} identifying the Windows Boot Manager entry in the \ac{BCD}.
The \ac{BCD}, as its name suggests, contains arguments used to configure various steps of the boot process \cite[Section 12]{windows-internals-7-part2}.
The boot partition is the primary Windows partition.
It is formatted with the \ac{NTFS} file system containing the Windows installation.
This is also the location of the final step of the Windows \ac{UEFI} boot process, \program{Windload.efi}, the application responsible for loading the kernel into memory \cite[Section 12]{windows-internals-7-part2}.

\subsection{Boot}

Now that we have established the basic structure of the Windows \ac{UEFI} boot environment, we can discuss the boot process.
The Windows boot process begins after the \ac{UEFI} Boot Manager launches the Windows Boot Manager, which starts by retrieving its executable path and the \ac{BCD} entry \ac{GUID} from the boot load options.
Then it loads the \ac{BCD} and accesses its entry.
If not disabled in the \ac{BCD}, it loads its own executable into memory for integrity verification \cite[Section 12]{windows-internals-7-part2}.
Depending on what hibernation status is set within the \ac{BCD}, it may launch the \program{Winresume.efi} application, which reads the hibernation file and resumes the kernel execution \cite[Section 12]{windows-internals-7-part2}.
On a full boot, it checks the \ac{BCD} for boot entries.
If the entry points to a BitLocker encrypted drive, it attempts decryption.
If this fails a recovery prompt is displayed, otherwise the system proceeds to load the \ac{OS} loader \program{Windload.efi} which maps the kernel image \program{ntoskernl.exe} into memory. After a call to \code{ExitBootServices()} it transfers execution to the kernel \cite[Section 12]{windows-internals-7-part2}.

\subsection{Runtime}

During runtime, Windows uses the variable services to communicate with the \ac{PF} and even exposes these to application developers.
It also supports firmware and option\-/\ac{ROM} updates via the capsule delivery services.

\section{Registry}

A crucial part of the whole Windows ecosystem is the \emph{Registry}.
It is a system database containing the information required to boot, what drivers to load, general system\-/wide configuration as well as application configuration \cite[Section 1]{windows-internals-7-part1}.
The Registry is a hierarchical database containing keys and values, whereas keys can contain other keys or values, forming a tree\-/like structure.
Values store data through various data types.
It is comparable to a file system structure with keys behaving like directories and values behaving like files \cite[Section 10]{windows-internals-7-part2}.
At the top level it has 9 different keys \cite[Section 10]{windows-internals-7-part2}.
Normally Windows users are not required to change Registry values directly and instead interact with it through applications providing setting abstractions.
Although some more advanced options may not be exposed and can be accessed through the \program{regedit.exe} application which provides a graphical user interface to traverse and modify the Registry \cite[Section 10]{windows-internals-7-part2}.
It also supports importing and exporting registry keys along their subkeys and contained values.
Internally, the registry is not a single large file but instead a set of files called \emph{hives}.
Each hive contains one tree that is mapped into the Registry as a whole.
There is no one\-/to\-/one mapping between registry root keys and hive files, the \ac{BCD} file, for example, is also a hive file and is mapped into the Registry under \program{HKEY\underbreak LOCAL\underbreak MACHINE\brackslash BCD00000000} \cite[Section 10]{windows-internals-7-part2}.
Some hives even reside entirely in memory as a means of offering hardware configuration through the Registry \ac{API}.

% !TeX root = ../../thesis.tex

\section{Security}

\autoref{fig:windows-startup-process} gives an overview over the security within the Windows startup process.
With Secure Boot starting the process and Trusted Boot enventually taking over.
We do not cover Measured Boot in this thesis.

\begin{figure}[phtb]
    \centering
    \includegraphics[width=1.0\textwidth]{windows/windows_startup_process.png}
    \caption{Windows startup process (taken from \cite{microsoft-secure-the-windows-boot-process})}
    \label{fig:windows-startup-process}
\end{figure}

\subsection{Secure Boot}

Devices shipping with Windows 11 must have Secure Boot enabled by default \cite{microsoft-windows-minimum-hardware-requirements-overview}.
Windows certified devices generally must allow users to enroll custom keys and signature \acp{DB} (to allow executioon of non\-/Windows bootloaders), aditionally a user should be able to completely disable secure boot.
Windows offers two signature \acp{DB} the \emph{Microsoft Windows Production PCA 2011} required for the Windows boot process and \emph{Microsoft Corporation \ac{UEFI} \ac{CA} 2011}, which is reserved for third party executables signed at Microsoft's discretion after manual review \cite{microsoft-uefi-signing}.
Microsoft advises to only allow other third party \ac{UEFI} applications if necessary and even mandates the exclusion of \acp{DB} other than \emph{Microsoft Windows Production PCA 2011} on Secured\-/core \acp{PC} \cite{microsoft-secure-the-windows-boot-process}.

\subsection{Trusted Boot}

Trusted Boot picks up where Secure Boot left off and maintains the code integrity chain through the kernel into the Windows startup process.
\ac{KMCI} verifies the digital signature of Windows boot components, including boot drivers, startup files and the \ac{ELAM} driver \cite{microsoft-trusted-boot}.
\ac{ELAM} provides antimalware software developers an interface to be intialized early in the boot process, before other third\-/party components, to monitor the subsequent boot process \cite{micosoft-windows-elam}.
\cite{understanding-windows-trusted-boot} gives a detailed walkthrough of the trusted boot process.

Microsoft can also leverage hardware virtualization features called \ac{VSM} for \ac{VBS}.
This allows for \ac{HVCI} where the \ac{CI} checks are taken out of the kernel environment and are now be performed from within the isolated hypervisor\-/based security environment \cite{micosoft-windows-oem-vbs}.

Formerly the term \emph{Device Guard} was used to promote the two security related features \ac{HVCI} and \ac{WDAC} (restricts execution of user level applcations).
Microsoft has since retired the term to prevent confusion, as there is no direct dependency between the two \cite{microsoft-windows-no-longer-device-guard}.

\subsection{\acf{BDE}}
\label{sec:windows:security:bde}

Windows is only able to enforce security policies when it is active, leaving the system vulnerable when accessed from outside of the \ac{OS} \cite[Section 9]{windows-internals-6-part2}.
Windows uses BitLocker, an integrated \ac{FVE}, aimed to protect system files and data from unauthorized accecss while at rest \cite{microsoft-bitlocker-overview}.
It also serves as a mechanism to verify boot integrity when used with in combination of a \ac{TPM} \cite[Section 9]{windows-internals-6-part2}.
The en- and decryption of the volume is done by a filter driver beneath the \ac{NTFS} driver as shown in \autoref{fig:bitlocker-volume-access-driver-stack}.
The \ac{NTFS} driver translates file and directory access into block-wise operations on the volume.
The filter driver then receives these block operations, encrypting blocks on write and decrypting blocks on read, while they pass through it.
This results in en- and decryption, that is entirely transparent to the \ac{NTFS} driver, making the underlying volume appear decrypted \cite[Section 9]{windows-internals-6-part2}.
The encryption of each block is done using a modified version of the \ac{AES}128 and \ac{AES}256 cypher \cite[Section 9]{windows-internals-6-part2}.
A \ac{FVEK} is used in combination with the block index as input for the algorithm, resulting in an entirely different output for two blocks with identical data \cite[Section 9]{windows-internals-6-part2}.
The \ac{FVEK} is encrypted with a \ac{VMK} which is in turn encrypted with multiple protectors.
These encrypted versions of the \ac{VMK} are stored together with the encrypted \ac{FVEK} in an unencrypted meta data portion at the beginning of the BitLocker protected volume \cite[Section 9]{windows-internals-6-part2}.
The \ac{VMK} can be encrypted by the following protectors:

\begin{figure}[htb]%
    \centering
    \includesvg[width=0.5\textwidth]{bitlocker_volume_access_driver_stack.drawio.svg}
    \caption{BitLocker Volume Access Driver Stack (inspired by \cite[Figure 9-24]{windows-internals-6-part2})}%
    \label{fig:bitlocker-volume-access-driver-stack}%
\end{figure}

\begin{description}
    \item[Startup key] The Startup key can be storead on removable media such as a \ac{USB} stick and serves as physical proof of ownership.
        The removable media contains a \program{.bek} file, that is named with a \ac{GUID} corresponding to BitLocker meta data entry, as it it possible to have multiple start up keys for the same volume \cite[Section 2.6]{bde-format-spec}\cite{microsoft-windows-prepare-your-org}.

    \item[TPM]
        When a \ac{TPM} is used to seal the \ac{VMK}, BitLocker can ensure integrity of early boot components and boot configuration, as the unseal operation fails upon modifcation of the boot flow.
        The \ac{TPM} can, in combination to the \ac{PCR} values, use a startup key, a pin or both to seal the \ac{VMK} \cite{microsoft-bitlocker-countermeasures}.
        The collection of \ac{PCR} indexes used to instruct the \ac{TPM} when initially sealing the \ac{VMK} is called a validation profile.
        The validation profile is configurable through Windows group policy settings, and\ac{PCR}11 is always required as its contents measure the BitLocker Access Control.
        Windows configures BitLocker with a default validation profile of \code{\{0, 2, 4, 11\}} \cite{microsoft-windows-bitlocker-group-policy-settings}.
        When Secure Boot is enabled and correctly configured, Microsoft defines this as only using their \emph{Microsoft Windows Production PCA 2011} signature \ac{DB}, BitLocker defaults to a validation profile of \code{\{7, 11\}} \cite{microsoft-pcr7-binding}.
        The measurements that extend \ac{PCR}7 are defined in \cite{microsoft-trusted-execution-environment}.
        The content reflects the current state and configuration of Secure Boot, including trusted keys.
        BitLocker then makes use of the Secure Boot measurements for integrity validation instead of the \ac{PCR} values containing the early boot components \cite{microsoft-windows-bitlocker-group-policy-settings}.

    \item[Recovery key]
        When BitLocker is activated it always generates a recovery key serving as an additional protector to whatever primary method is selected.
        This allows users to recover their data when for example the \ac{TPM} fails to unseal the \ac{VMK} or the startup key is lost.
        The recovery key can be added to the users Microsoft account or saved to unencrypted media.
        It consists 48 digits divided into 8 blocks, each block is converted into a 16-bit value making up a 128-bit key \cite[Section 2.4]{bde-format-spec}.

    \item[User key] When BitLocker is used without a \ac{TPM} the \ac{VMK} can be encrypted with the use of a user supplied password with the maximum length of 49 characters. \cite[Section 2.7]{bde-format-spec}

    \item[Clear key] BitLocker can also be suspended when a user for example wants to update their \ac{PF}.
        The \ac{VMK} is then encrypted with an unprotected 256-bit key stored on the volume \cite[Section 2.5]{bde-format-spec}.

\end{description}
% !TEX root = ../thesis.tex

\chapter{Past Threats}
\label{sec:past-threats}

% https://www.blackhat.com/docs/asia-17/materials/asia-17-Matrosov-The-UEFI-Firmware-Rootkits-Myths-And-Reality.pdf
% https://www.researchgate.net/profile/Anton-Sergeev-2/publication/269310822_Too_young_to_be_secure_Analysis_of_UEFI_threats_and_vulnerabilities/links/551306b50cf23203199aa237/Too-young-to-be-secure-Analysis-of-UEFI-threats-and-vulnerabilities.pdf

Before we implement our own \ac{UEFI} attacks, we first take a look at how past \ac{UEFI} threats are structured.
The threats discussed range from actual attacks found in the wild and analyzed by security researchers, over attacks that have been implemented for similar research purposes, to tools that enable system owners a more advanced control over their systems.


\begin{center}
    \begin{tabular}{lll}
        \toprule
        \textbf{Approach}                  & \textbf{Bootkit} & \textbf{Rootkit} \\
        \arrayrulecolor{gray}
        \cmidrule[0.4pt](r){1-1}
        \cmidrule[0.4pt](lr){2-2}
        \cmidrule[0.4pt](l){3-3}
        \multirow{4}{4em}{Storage\-/based} & \textbf{ours}    & VectorEDK        \\
                                           &                  & Mosaicregressor  \\
                                           &                  & LoJax            \\
                                           &                  & \textbf{ours}    \\
        \cmidrule[0.4pt](r){1-1}
        \cmidrule[0.4pt](lr){2-2}
        \cmidrule[0.4pt](l){3-3}
        \multirow{3}{4em}{Memory\-/based}  & Efiguard         & MoonBounce       \\
                                           & ESPecter         & CosmicStrand     \\
                                           & Dreamboot        &                  \\
                                           & FinSpy           &                  \\
        \arrayrulecolor{black}
        \bottomrule
    \end{tabular}
\end{center}

\section{Infection}
\vspace{-0.5em}

The infection is the most important part of an attack, as it dictates when, in what environment, and with what privileges the \ac{UEFI} payload is executed.

\vspace{-0.5em}
\subsection{Bootkit}
\vspace{-0.5em}

Bootkits use the \ac{UEFI} Boot manager to gain execution on a system.
There are a variety of methods using different mechanisms of the boot process.
FinSpy backs up and replaces the Windows Boot Manager \program{bootmgfw.efi} on the \ac{ESP} \cite{finspy}.
ESPecter patches the entry point of the Windows Boot Manager \program{bootmgfw.efi} and its copy \program{bootx64.efi} in the default boot path, so it executes malicious code upon launch \cite{especter}.
Dreamboot and EFIGuard are more proof-of-concept than real attacks and suggest being booted into using removable media, but they are also able to be added to the default boot path on an \ac{ESP}, or generally added as their own new boot entry \cite{efiguard}.
They are both applications, which launch the Windows Boot Manager through its boot entry upon execution \cite{dreamboot, efiguard}.

\subsection{Rootkit}

Firmware rootkits have been rarer and how exactly the firmware images were infected is not often known.
VectorEDK uses \ac{OEM}'s software tooling to generate a firmware update utility on a bootable \ac{USB} stick that can then be inserted with physical access to the system \cite{mosaicregressor}.

LoJax infection method comes with the signed kernel driver from the program \program{RWEverything}.
\program{RWEverything} is a legitimate tool that can be used to query hardware\-/related information on a system.
LoJax uses the driver to read and write to memory mapped \ac{I/O}, as well as \ac{PCI} configuration registers.
It leverages this to find the \ac{SPI} flash mapping, to dump the firmware image.
It then removes previously packaged \ac{NTFS} drivers and adds its payload.
Reflashing the modified image relies on the platform to be either misconfigured or of an older kind that has a race condition exploit.
The \ac{SPI} flash is secure by a \ac{BIOS} control register, which has a \ac{BIOS} Write Enable bit and \ac{BIOS} Lock Enable bit.
The locking mechanism has to be correctly implemented by firmware designers through \ac{SMM} interrupts.
When writing to the \ac{BIOS} Write Enable bit, while the \ac{BIOS} is enabled, the operation initially succeeds but is then reverted by a \ac{SMM} interrupt routine.
This could either be incorrectly implemented or exploited through race conditions using multi\-/processing or multi\-/threading existing on older hardware.
When one thread constantly sets the Write Enable bit to 1 and the other tries to perform write operations on the \ac{SPI} flash, the firmware image will eventually be overwritten \cite{lojax}.

MosaicRegressor and LoJax add their payload in the form of \ac{DXE} drivers to a firmware volume \cite{mosaicregressor-technical-details,lojax}, as these are automatically executed by the \ac{DXE} dispatcher.
MoonBounce and CosmicStrand instead patch existing files in the firmware image.
MoonBounce patches the \ac{DXE} Core \cite{moonbounce}, while CosmicStrand patches an existing \ac{DXE} driver \cite{cosmicstrand}.
While both approaches could fundamentally be done in the form of an added \ac{DXE} driver, it does make the detection harder.

\section{Approach}

We can categorize the threats by their attack vector.
Rootkits and bootkits do not seem to have distinct approaches, as they both start their execution in the \ac{UEFI} environment prior to the Windows boot process.
We found that their approach can mainly be divided into storage\-/based and memory\-/based attacks.
Storage\-/based attacks mostly gain execution in the operating system environment by writing their payload into the Windows installation and modifying configuration data on the disk.
These attacks are often performed offline before any parts of the operating system are executed.
Memory\-/based attacks instead hook into the operating system's boot process to execute malicious code alongside the operating system in memory.
For storage\-/based attacks, we were only able to find examples of rootkits \cite{vector-edk,mosaicregressor-technical-details,lojax}, whereas memory\-/based attacks were performed by both root- and bootkits \cite{dreamboot,efiguard,especter,finspy,moonbounce,cosmicstrand}.
There is no technical limitation as we show in \autoref{sec:attacks:neither:bootkit} when we implement our own storage\-/based bootkit, but more likely a general preference for memory\-/based attacks, as they are more sophisticated.
Storage\-/based attacks face more restrictions such as BitLocker and code integrity checks.

\subsection{Storage-based}

Storage\-/based attacks need file\-/based access to the Windows installation to modify its content.
The primary partition is \ac{NTFS} formatted and, due to the \ac{UEFI} specification only mandating compliant firmware to support \ac{FAT}12, \ac{FAT}16 and \ac{FAT}32 \cite[Section 13.3.1.1]{uefi-spec}, \ac{NTFS} drivers are delivered as part of the attack.
MosaicRegressor and Lojax seem to use VectorEDK's leaked \ac{NTFS} driver \cite{mosaicregressor-technical-details, lojax}.
LoJax deploys its payload under the file path \program{C:\textbackslash Windows\textbackslash SysWOW64\textbackslash autoche.exe} and then modifies the registry entry \program{HKEY\_LOCAL\_MACHINE\textbackslash SYSTEM\textbackslash CurrentControlSet\textbackslash Control\textbackslash Session Manager\textbackslash BootExecute}, so that their payload is executed instead of the original executable \cite{lojax}.
MosaicRegressor simply deploys its payload in the Windows startup folder \cite{mosaicregressor-technical-details}, whose contents, as its name suggests, are executed upon Windows startup.

\subsection{Memory-based}

It seems to be unique to ESPecter to patch out the integrity self-check of the Windows Boot Manager, as it is the only bootkit to change the bootloader on disk instead of in\-/memory \cite{especter}.
FinSpy and Dreamboot when executed, load \program{bootmgfw.efi} into memory and apply patches before transferring execution \cite{finspy, dreamboot}.
EFIGuard loads an additional \ac{UEFI} driver which is able to hook the boot service \code{LoadImage()}.
When the function is called to load \program{bootmgfw.efi}, it patches the bootloader in memory \cite{efiguard}.
MoonBounce applies its patches from within an \code{ExitBootServices()} hook \cite{moonbounce}.

The general approach is the same for all memory\-/based attacks.
They propagate malicious code execution further up in the boot chain by hooking each image of the boot process as it is loaded into memory, i.e., from \program{bootmgfw.efi} to \program{Winload.efi} to \program{ntoskernel.exe}, the kernel image.

EFIGuard and ESPecter patch the kernel to disable Windows Driver signing, allowing them to install further kernel drivers \cite{efiguard,especter}.
While FinSpy and Dreamboot deploy payloads executed with elevated privileges \cite{finspy,dreamboot}.
MoonBounce and CosmicStrand map code directly into the kernel space \cite{moonbounce,cosmicstrand}.
% !TeX root = ../thesis.tex

\chapter{Test Setup}

\TODO{describe test setup}
qemu + swtpm

fresh Windows 11 installation
% !TEX root = ../thesis.tex

\chapter{Attacks}
% chapter summary, briefly introduce the three attacks
Our different attacks face three escalating levels of security mechanisms. The first is with Secure Boot and Bitlocker disabled, the second is just Secure Boot enabled and the third is both Secure Boot and Bitlocker enabled with the focus of the study on Bitlocker.
% common assumption/requirement across the attacks
All attacks share the requirement of being able to add DXE Drivers to the DXE Volume.
% how to achieve assumption/requirement
This can be achieved by having read/write access to the SPI flash or using the Signed Capsule Update. Gaining read/write access to the SPI Flash is possible either through physical access to the device by using an SPI clamp on the chip itself or through exploits like for example the
% see SMM multi threaded exploit
. Signed Capsule Updates can be leveraged with access to private vendor information by signing the payload to make it appear legitimate or by intercepting the distribution process and employing infected firmware.
% ref lenovo vantage for distributed example
% network boot maybe

\section{Test Setup}
\TODO{describe test setup}
swtpm

% !TeX root = ../../thesis.tex

\section{Neither Secure Boot nor BitLocker}

We start by implementing a baseline attack, that we can use to test against Secure Boot and BitLocker.
We implement it in the form of a bootkit and a rootkit, with both sharing the same approach and core functionality.
The general approach of our attack is to access the hard drive containing the Windows installation from within \ac{UEFI} environment.
We gain then code execution with elevated privileges in the Windows environment by modifying its content.

\subsection{Bootkit}
\label{sec:attacks:neither:bootkit}

We start with the bootkit.

\subsubsection{Infection}

We have two ways to infect a system, we can either use a bootable medium such as a CD-ROM or \ac{USB} stick with a \ac{UEFI} application installing the bootkit or using a Windows executable.
Booting into the installer application requires either the firmware implementation or the boot order to prefer booting from the removable media over Windows.
This can be forced when booting accessing the interactive firmware menu at startup, given that it is not password protected.
Installation from Windows requires admin privileges to mount and modify the \ac{ESP}.

The installation process is identical for both options, we access the \ac{ESP} and create a copy of the Windows Boot Manager located under \lstinline{EFI\\Microsoft\\Boot\\bootmgfw.efi}.
We then replace the original with our bootkit as well as dropping all resources required by the bootkit on the \ac{ESP}.
Now that our bootkit is in place of the Windows Boot Manager, when the \ac{UEFI} Boot Manager selects the boot load option for the Windows Boot Manager, it will cause our bootkit to be executed.
\autoref{fig:windows-boot-entry} shows a dump of the Windows boot entry using the \ac{UEFI} shell command \program{bcfg}.
The entry contains the device path including the file path and optional data.

\begin{figure}[htb]
    \centering
    \includegraphics[width=1.0\textwidth]{attacks/neither/windows_boot_entry.png}
    \caption{Windows boot entry, part of the \ac{UEFI} shell output of \program{bcfg}}
    \label{fig:windows-boot-entry}
\end{figure}

\subsubsection{File access}

The most important step for a storage-based approach is gaining access to the Windows installation from within the \ac{UEFI} environment.
Since \ac{UEFI} does not require the firmware to come with an \ac{NTFS} driver, our attack has to come with its own.
\ac{EDK} II does not provide one, but we can use the open source \ac{NTFS} driver \lstinline{ntfs-3g} from Tuxera \cite{ntfs-3g}.
It was ported to the \ac{UEFI} environment by \emph{pbatard} \cite{ntfs-3g-uefi}.
Using \ac{EDK} II to compile it, we receive a \program{.efi} executable image.

We can use the \ac{UEFI} shell and its file system related commands to test the \ac{NTFS} driver's capabilities.
When booting into the \ac{UEFI} shell we are greeted with a screen displaying the \ac{UEFI} specification version the firmware supports and a list of default mappings for file systems and block devices.
These mappings are created by the shell to provide a short name that can be used interchangably with longer device path when issuing commands \cite[Section 3.7.2]{uefi-shell-spec}.
They are designed to be consistent across reboots as long as the hardware configuration stays the same and are comparable to Windows partition letters \cite[Appendix A]{uefi-shell-spec}.
\autoref{fig:mapping} shows the mapping of a partition containing a Windows installation.
As there is no \ac{NTFS} driver present yet, it is only displayed as a block device.

\begin{figure}[htb]
    \centering
    \includegraphics[width=0.8\textwidth]{attacks/neither/mapping}
    \caption{Mapping of the Windows parition}
    \label{fig:mapping}
\end{figure}

We can enter the mapping name of the file system containing our \ac{NTFS} driver to use it as our current working directory and load the driver using the \program{load} command.
The command is executed successfully and can the driver is now listed when querying the currently loaded drivers with the command \program{drivers}.
We can now instruct the default mappings to be reset with the \program{map -r} command, to receive an updated list including the file systems now provided through the \ac{NTFS} driver.
\autoref{fig:mapping-ntfs} also shows us that the new file system now sit on top of a device which previously was only listed as a block device.

\begin{figure}[htb]
    \centering
    \includegraphics[width=0.8\textwidth]{attacks/neither/mapping_ntfs}
    \caption{Mapping of the Windows parition after loading the \ac{NTFS} driver}
    \label{fig:mapping-ntfs}
\end{figure}

As done before we now type the mapping name of the new file systems, we check the root directories' contents with \program{ls} until we find the partition containing the Windows installation and then enter \program{vol} to check the access rights.
This reveals that the file system is currently read\-/only.
Upon debugging the \ac{NTFS} driver it appears to be that the driver falls back to read\-/only when it encounters a file that indicates that the Windows system is in hibernation mode.
We can remove this fallback from the \ac{NTFS} driver's code and recompile.

On our hardware setups we noticed that the firmware can already ship with an \ac{NTFS} driver included, which is read\-/only.
In the case of our rootkit we would be able to remove this driver by modifying the firmware iamge, but we can implement a solution that applies to both types of \ac{UEFI} payload.
We can change the \ac{NTFS} driver to install its \hyperref[lst:simple-file-system-protocol]{Simple File System Protocol} under a different \ac{GUID} instead of \code{gEfiSimpleFileSystemProtocolGuid}.
This makes it possible to install two instances of the \hyperref[lst:simple-file-system-protocol]{Simple File System Protocol} alongside each other on the same controller.
The alternative \ac{GUID} can then be used in our root- and bookit, to retrieve our specific protocol instance with write access.
The driver also has to open the protocols it uses without demanding exclusive ownership over them.
This also prevents the \ac{NTFS} driver from being when trying to open a protocols that is already in exclusive ownership \cite[Section 7.3]{uefi-spec}, which would be a likely scenario as filesystem drivers are encouraged to get exclusive control over their block device \cite[Section 13.5]{uefi-spec}.

We now know that provided we get to load the \ac{NTFS} driver we can access the data contained within a Windows installation with read and write.
\TODO{WEITERMACHEN} Since our rootkit will not use the UEFI shell we need to have the \ac{NTFS} driver load as part of the boot process.

The next step is for our bootkit to use the \ac{NTFS} driver to gain file system access and write our payload to the Windows installation.
During our bootkit infection process we place the \ac{NTFS} driver on the \ac{ESP}, so that our bootkit can load it.
In our bootkit, we can use the Loaded Image Protocol, that is installed to the handle of the bootkit's image in memory to retrieve the handle of the device our bootkit was loaded from \cite[Section 9.1]{uefi-spec}.
This handle can then be used to call the Boot Services \lstinline{LoadImage} and \lstinline{StartImage} to load and execute the NTFS driver.
Since the driver conforms to the UEFI Driver Model, we need to also reconnect all controllers recursively, so it can assume controller over the NTFS formatted volumes, by installing the \hyperref[lst:simple-file-system-protocol]{Simple File System Protocol} on their handles.
Loading the payload and other non-executable files into memory is done differently, here we use the handle from the Loaded Image Protocol to open the \hyperref[lst:simple-file-system-protocol]{Simple File System Protocol} installed onto the \ac{ESP}, we can then call the \lstinline{OpenVolume} resulting in an instance of the \hyperref[lst:simple-file-system-protocol]{File Protocol} representing the root folder of the volume \cite[Section 13.4]{uefi-spec}.
This instance can then be used to open and read our payload with the absolute path on the \ac{ESP} into memory.

\subsubsection{Payload deployment}

To perform the write operation we now need a handle we did not yet interact with, at least directly.
We can use the Boot Service \lstinline{LocateHandleBuffer} to receive an array of all handles that support the \hyperref[lst:simple-file-system-protocol]{Simple File System Protocol}, this includes volumes such as the \ac{ESP} but also the Windows recovery partition.
We can iterate over all handles to open the volume and attempting to create a new file with a file path that's inside of the Windows installation.
This operation fails on volumes not containing a Windows installation which we can just skip.
Eventually the volume containing Windows is found and the file is created and opened successfully, we can then write our payload, that we read into memory earlier, onto the disk and close the file again.

Now the question arises as to where to write our payload to, we want automatic and elevated execution.
Earlier we discovered that the \ac{NTFS} \ac{DXE} driver disregards the file access permission model \TODO{Windows File Permissions} so we are not restricted in the same way an unprivileged user would be when accessing the disk.
\emph{MosaicRegressor} writes its payload to the Windows startup folder, a folder whose contents are automatically executed at system startup.
The programs within the startup folder are unfortunately not automatically run at an elevated level, so this isn't a suitable target location.

\TODO{DLL proxy loading}
\TODO{modifying Windows Executables KMCI}

% Task Scheduler
The Task Scheduler is a Windows service responsible for managing the automatic execution of background tasks \cite[Section 10]{windows-internals-7-part2}.
Tasks are performed on certain triggers, which may be time-based (periodically or on a specific time) or event-based, for example on user logon or system boot \cite{microsoft-task-scheduler-triggers}.
A task can perform various actions upon invocation \cite{microsoft-task-scheduler-actions}, but we will focus on command execution.
Most tasks will simply execute other programs as their action, this execution is performed under specified a security context \cite{microsoft-task-scheduler-security-contexts}.
The idea of our attack is to have a task, that performs its action with a high privilege level, execute our payload.
The task of our choosing is called \lstinline{Autochk\Proxy}, that performs the command

\begin{lstlisting}
%windir%\system32\rundll32.exe /d acproxy.dll,PerformAutochkOperations
\end{lstlisting}

30 minutes after system boot, the executable \lstinline{rundll32.exe} loads the \ac{DLL} \lstinline{acproxy.dll} and invokes the exported function \lstinline{PerformAutochkOperations} \cite{microsoft-rundll32}.
The function name as well as the task name suggest the performed action relates to the Windows utility \emph{autochk} which verifies the integrity of \ac{NTFS} file systems \cite{microsoft-autochk}.
The Task Scheduler keeps book of its active tasks in the registry under \lstinline{HKLM\SOFTWARE\Microsoft\Windows NT\CurrentVersion\Schedule\TaskCache}, grouped by four subkeys Boot, logon, plain and Maintenance.
These entries consist only of a \ac{GUID} that is used to look up the task descriptor saved under their respective task master (registry) keys, these task master keys are located under \lstinline{HKLM\SOFTWARE\Microsoft\Windows NT\CurrentVersion\Schedule\TaskCache\Tasks} \cite[Section 10]{windows-internals-7-part2}.
There also exist a secondary copy of the task descriptors, on the regular file system under \lstinline{%windir%\system32\Task}, stored as \ac{XML} files.

We can use the Task Scheduler Configuration Tool to modify the target task on a system under our control, we change the executable path as well as remove the configured delay.
We then use the Windows registry editor \lstinline{reged.exe} to navigate to the task descriptor store, there we search for the task master key belonging to our task and export this key.

\includegraphics[width=\textwidth]{attacks/neither/05_taskscheduler_autochk.png}

% edit with start cmd.exe and trigger manually
% whoami
% We modify the task's action, to run our payload instead.
To verify the privileges our payload is executed with, we can save the output of \lstinline{whoami /all} into a file.
The \lstinline{whoami} command shows the current user and privileges \cite{microsoft-whoami}.
After manually triggering the task through the configuration tool, we see that our payload was run from the \lstinline{nt authority\system} user account, which is the most privileged system account \cite{microsoft-localsystem-account}.

\TODO{whoami /all snippet}

% chntpw and reged
% edit Task in machine under Control
We can use this exported key and import it on our victim's system as part of our attack.
This way, instead of modifying a single value of the registry key, the victim's key maintains its integrity as we also overwrite the hash value with correct data.
To import the key on an offline system, we can use a Linux utility called \lstinline{chntpw} whose primary purpose it is to reset the password of local Windows user accounts \cite{chntpw}.
The library does this by editing the registry of a Windows installation and as such the author also offers a standalone registry editor called \lstinline{reged}.
% test
% dual boot
We can test the Linux tool when dual-booting a Linux and a Windows installation.
We place our payload in the Windows installation and then boot into Linux, where we can open the \lstinline{HKEY_LOCAL_MACHINE/SOFTWARE} hive in \lstinline{reged} and import our modified registry key.
% import and overwrite registry key on target machine
This overwrite the task descriptor and when booting into Windows our payload is executed.

% port to uefi
The next step is to port the \lstinline{reged} utility so that it works in the UEFI environment, so we can use it as part of our bootkit.
% most stdlib stuff is just mapping to UEFIlib stuff with equivalents or using gcc implementations
The porting process boils down to providing semantically equivalent definitions of external function calls, such as c standard library and Linux kernel functions, to link against.
Declarations and macros are still supplied by the local compiler's system headers.
Function definitions can often be translated to \ac{UEFI} equivalents, \ac{EDK} II has libraries offering implementations of commonly used abstraction.
% memory allocation (malloc, calloc, realloc), memory manipulation (memset, memcpy) string manipulation (sscanf, strtol), stdout (printf), abort, exit
Memory allocation maps to the MemoryAllocationLib, memory manipulation to BaseMemoryLib, basic string manipulation to BaseLib, stdout to PrintLib (only relevant for print debugging).
% cstdio is non trivial and has to be implemented by calling protocols on the right volume
Function calls related to standard input and output such as opening, reading and writing a file, namely the hive file, are more complex and have to be mapped to the \ac{UEFI} protocols \hyperref[lst:simple-file-system-protocol]{Simple File System Protocol} and \hyperref[lst:simple-file-system-protocol]{File Protocol}.
Luckily the author of \lstinline{reged} used distinct functions to access the hive file and registry file, making it possible to keep the original source code unmodified, except for a change in the import behavior.
The name of a task master key is the task's \ac{GUID}, which may differ from device to device, thus we cannot import a key into its exact path, we instead iterate over the subkeys of the target's parent key.
We then match for the name value of the key.

Now that we modified the Windows installation to execute our payload upon boot, we need to transfer execution from the bootkit to the original Windows Boot Manager.
Loading the original application is inspired by how the UEFI Boot Manger loads boot options, this includes relaying the \lstinline{LoadOptions} and \lstinline{ParentHandle} of the \emph{\ac{EFI} Loaded Image Protocol} \cite[Section 9.1]{uefi-spec} instance installed to our bootkit to the Windows Boot Manager.


\subsection{Rootkit}

Performing the same attack in the form of a rootkit is very similar and mainly differs in the infection process.
The \ac{UEFI} payload is now compiled as a \ac{DXE} driver instead of an application.
When placed in the \ac{DXE} volume it is automatically loaded by the \ac{DXE} Dispatcher iterating over the \ac{FV}, loading  drivers whose dependencies are resolved.
The core functionality of our \ac{UEFI} payload is identical with the exception that we don't have to manually load the \ac{NTFS} driver anymore and accessing the Windows payload is now done with the \emph{Firmware Volume2 Protocol} defined in the \cite[Section 3.4.1]{pi-spec}, instead of \emph{Simple Filesystem Protocol}. There are no traditional file names on a firmware volume, and we have to search for files using the module \acp{GUID}.

\subsubsection{Infection}

Infection with the rootkit is has a much higher barrier of entry, as it requires read and write access to the firmware image, which often requires physical access.
\autoref{sec:test-setup} potentially exploit \ac{OEM} specific flash mechanism, signing with stolen private key, part of the supply chain, might also be physical \TODO{LIST ALL OPTIONS}

We have to retrieve the image, insert our payload into a \ac{DXE} volume and deploy the modified image.
In UEFITool we navigate to the \ac{DXE} Volume containing the \ac{DXE} Core and \ac{DXE} drivers.
% since files are part of file sections we cant drop in the .efi
We cannot directly drop our \ac{UEFI} payload in form of \lstinline{.efi} files with UEFITool, because \ac{DXE} drivers have three mandatory sections: the \ac{PE32} executable section, composed of the \lstinline{.efi} file content, a version section and the \ac{DEPEX} section \cite[Vol. 3, 2.1.4.1.4]{pi-spec}.
% compile dxe driver within ovmf
% generate unused volume to receive .ffs file with version, depex, user interface and pe section
For our \ac{UEFI} payload to be generated as a sectioned \ac{FFS} file we add our files to the build process of \ac{OVMF} package in \ac{EDK} II. When part of the \ac{FDF} which is used to generate a firmware image file, the intermediary \lstinline{.ffs} files from the build process are of much value for us.
% pack executable binary as uefi module
% EDK II produces freeform image with one raw section
For our Windows payload we can use a special \ac{EDK} II module type which takes binary files as input, resulting in a sectioned file of type \lstinline{EFI_FV_FILETYPE_FREEFORM}, with no restrictions on the contained file sections \cite[Vol. 3, 2.1.4.1.7]{pi-spec}.
The output contains only one file section of type \lstinline{EFI_SECTION_RAW} consisting of the binary payload.
This use of this special module has the benefit that its \ac{GUID} is used to attribute the sectioned file when being placed in the firmware volume.
Not that we have \lstinline{.ffs} files corresponding to all our resources used in the attack we can import these into the target image with UEFITool.

\TODO{this}
overwrite the SPI flash with modified image by using the programmer again.

\clearpage

% !TeX root = ../../thesis.tex

\section{Secure Boot Enabled}
\label{sec:attacks:secure-boot}

Our second attack is performed with Secure Boot enabled.
We assume that the signature \acp{DB} of allowed images does not contain our image's hashes and that the interactive \ac{UEFI} setup menu is password protected.
Otherwise, we could simply turn off Secure Boot.

\subsection{Bootkit}

The interactive menu being password\-/protected makes the likelihood of infection via booting into our installer smaller.
We now solely rely on the boot order/firmware policy to prefer removable media.
Even if this was to be the case, we promptly see that Secure Boot already denies the execution of the installer when trying to boot it.
When using our Windows installer we observe the same denial for the bootkit itself.
The Windows Boot Manager boot option pointing to our bootkit is now denied execution.
If we were to have overwritten the standard boot entry of the hard drive \program{EFI\brackslash Boot\brackslash bootx64.efi}, a copy of the Windows Boot Manager, Windows would now be rendered unbootable.

\subsection{Rootkit}

In \autoref{sec:uefi-pi:pi:security} we discussed how the \ac{PI} specification defines the usage of its two security architectural protocols, with them being required to be invoked on every call to \code{LoadImage()}, and that the \nameref{lst:security2-architectural-protocol} is responsible for the implementation of Secure Boot authentication.
As \code{LoadImage()} is used internally within the \ac{DXE} dispatcher the security protocol invocations also apply to our rootkit's \ac{DXE} drivers when being loaded.
We also discussed in \autoref{sec:uefi-pi:uefi:secure-boot} that Secure Boot relies on the firmware image as its root of trust, where Secure Boot is inherently unable to verify the behavior of the \ac{PI} process.

Now, these two seem to be conflicting, but when we deploy our rootkit it is unaffected by Secure Boot and executes just like before.
When we look at the reference implementation in \ac{EDK} II, we can see why: \autoref{lst:dxe-image-verification-handler} shows a snippet of the function that is used to implement the \nameref{lst:security2-architectural-protocol}.
It shows that the image origin dictates which policy is being applied.
The standard policy for images from a \acf{FV} (\code{IMAGE\underbreak FROM\underbreak FV}) is to always allow execution.
This aligns with what the \ac{UEFI} specification says about the Secure Boot Firmware Policy:
\textcquote[32.5.3.2]{uefi-spec}{The firmware may approve \ac{UEFI} images for other reasons than those specified here.
    For example: whether the image is in the system flash \textelp{}}.
This behavior was reproducible on all our test setups.
Even if the \ac{PF} were to apply Secure Boot authentication to \ac{DXE} drivers, as long as the root of trust of authentication is established within the firmware image it can be patched as all code within the firmware image is modifiable.

\vspace{1em}

\lstinputlisting[language=C,caption={Policy Selection in DxeImageVerificationHandler (\ac{EDK} II reference implementation of \nameref{lst:security2-architectural-protocol})},captionpos=b,label=lst:dxe-image-verification-handler]{code/dxe_image_verification_handler.c}

\clearpage
\section{Secure Boot and Bitlocker}
assumptions:
secure boot or not
bitlocker enabled with TPM auto decryption

observation:
boot execution differs from executing rootkit
tpm values different
bitlocker auto decryption fails
recovery key prompt

what is the reaction of the average user
(ask admin for recovery password)
type in recovery password
alternative would be to remove drive and insert into safe device

% ref to background os loader
prompt is done by the OS Loader
ergo still during transient system load phase
required to use protocol services
therefor uses uefi services for IO
such as SimpleTextInputEx Protocol
go over the two different input protocols
find out which one is used

explain more in depth how protocols are returned to the end user
one instance per controller/handle

explain basic hooking
explain how we retain information of the hook in question
map protocol pointer to hook information
keylog recovery key
key input advancment is weird and makes tracking tricky

alternatively screen shot
still need hook to find when enter is pressed
explain how screenshotting works
some basic compression
wait for recovery key
send recovery key on enter press

on real hardware
network stack wasn't installed onto handles when boot over ip was disabled
compared loaded dxe drivers between both configurations with efi shell
Realtek Family driver not loaded
load manually
reinstall all handle to controllers to enable network stack regardless

sending key out is only good for physical access attack vector
dislocker linux utility
mount encrypted drive with decryption mean
read and write access
dual boot in vm
enter password and it works
port to uefi
bitlocker encrypts block-wise
uefi protocol stack
hook block io
again hook data mapping
dislocker validate block
solves recovery key advancement issue

hook ExitBootServices
enable hook
write payload
import registry key
disable hook

next boot would require to input tpm values again
update tpm values in payload
caveat pin? look into this

% reference to rootkit definition
persistence when part of root of trust
fresh install / tpm update values
% paper von betreuern
hook Trusted Copmuting Group 2 (TCG2) Protocol
TPM communication
receive bitlocker vmk key and send to dislocker
% !TeX root = ../thesis.tex

\chapter{Results}
% !TeX root = ../thesis.tex

% https://www.scribbr.com/research-paper/discussion/

meaning, importance, and relevance of your results
explaining and evaluating what you found, showing how it relates to your literature review

\chapter{Discussion}

we achieved a boot and rootkit with unrestricted disk access which results in elevated execution on the target OS
persistence with rootkit/none with bootkit
bootkit delivery:
usb stick, from windows
rootkit delivery:
spi clamp, firmware delivery process, maybe windows with exploit

bootkit vs rootkit
bootkit:
installation is much easier:
windows installer
physical presence with bootable usb stick
defeated by secure boot
in case of physical presence it may require to change boot order
bios password mitigates that
if no password present we can disable secure boot
not entirely persistent
fresh reinstallation with partition removal and general hard drive replacements defeat it

rootkit:
barrier of entry is higher
physical access is more difficult than just booting from a usb stick
exploit to override spi flash or be delivered with supply chain difficult
but high payoff
persistence across reinstallations or hard drive replacements
can prevent further bios updates and be unremovable
secure boot does not include internal DXE drivers
option ROM rootkit is defeated by secure boot
spi reflash may disable secure boot by changing variable anyways
SMM rootkit very powerful, complete control over the system
% \cite{}  https://pdfs.semanticscholar.org/68e7/42523f493b78111031a5a221a8cf767064f4.pdf


we didnt try to be undetectable
windows is very vulnerable with unrestricted disk access
secure boot is very limited
secure boot can easily be disabled without bios password
TPM does its job in detecting PCR change
% attack assumption reflected to real world aplicability
bitlocker reocvery prompt can raise suspicion
very effective if part of the delivery process or in general present before os installation
BitLogger somewhat last resort
social engineering aspect
you can change recovery message and URL in BCD hive


% https://learn.microsoft.com/en-us/windows/security/information-protection/bitlocker/bitlocker-countermeasures#bootkits-and-rootkits

boottime vs runtime rootkit

\section{Rootkit classification}

statisken zu bilocker und secureboot auf systemen

industrie standard zur system security in firmen

\section{Mitigations}

bios password against secure boot removal

windows cant assume what the implementation of ReadKeyStrokeEx looks like (normally function patching might have a jump etc, which we dont even have here)

hardware validated boot

inaccessible spi flash

tpm + pin/usb detectability

\subsection{User awareness}


% https://learn.microsoft.com/en-us/windows/security/information-protection/bitlocker/bitlocker-recovery-guide-plan

recovery guide

what causes bitlocker recovery
- password wrong too often
- TPM 1.2, changing the BIOS or firmware boot device order
- Having the CD or DVD drive before the hard drive in the BIOS boot order and then inserting or removing a CD or DVD
- Failing to boot from a network drive before booting from the hard drive.
- Docking or undocking a portable computer
- Changes to the NTFS partition table on the disk including creating, deleting, or resizing a primary partition.
- Entering the personal identification number (PIN) incorrectly too many times
- Upgrading critical early startup components, such as a BIOS or UEFI firmware upgrade
- Updating option ROM firmware graphics card
- Adding or removing hardware
- REMOVING, INSERTING, OR COMPLETELY DEPLETING THE CHARGE ON A SMART BATTERY ON A PORTABLE COMPUTER
- Pressing the F8 or F10 key during the boot process
what does the recovery screen say

% https://learn.microsoft.com/en-us/windows/security/information-protection/bitlocker/bitlocker-device-encryption-overview-windows-10
% https://learn.microsoft.com/en-us/mem/configmgr/protect/deploy-use/bitlocker/helpdesk-portal?source=recommendations
% https://learn.microsoft.com/en-us/microsoft-desktop-optimization-pack/mbam-v25/
Enables end users to recover encrypted devices independently by using the Self-Service Portal

googeln wie legitime recovery key prompt reaktion aussieht

enterprise policy recovery key einschraenkbar?

enterprise policy on recovery key loss

vermitteln was das prompt bedeuten koennte

aber kann man einfach nicht anzeigen lassen

Security Flaw of entering a Recovery Password in an inheritly unsafe System

enterprise doesnt hand out recovery keys and instead receives hard drive


!!!!!!!!!!!!!!!!!!!!!!!!!
without hardware chain of trust a compromised system can patch/change any software and fixes are impossible

phishing prompts on their own
% !TeX root = ../thesis.tex

\chapter{Conclusion}

Our practical analysis of \ac{UEFI} threats against Windows 11 showed that enabling Secure Boot when using BitLocker comes with a hidden hit in security.
Microsoft missuses Secure Boot in an attempt to provide platform firmware integrity validation, where the \ac{TPM} already offered a perfectly fine solution.
While also having set a dangerous precedence by offering user the immediate ability to override integrity violations in an inherently untrustworthy system.
Trade\-/off


\section{Achieved Goals}

when we are already in the image we can gain full control over the system
system cant be trusted anymore e.g. uefi services
full file access
escalate it to local system level execution
bitlocker has the flaw of allowing to enter criticial information into an inherently untrustable system
on the other hand one could force such a prompt themselves
mere exisitence of a recovery key is a security flaw

\section{Future Work}

tpm and pin
capsule update
exploit in tpm measruement chain that results in not being measured
can exploit the tg2 hook directly to retrieve the vmk
look into memory based rootkit with bitlocker and tpm and also what the hypervisor kernel security does to them
look further into \ac{SMM}

In \autoref{sec:tpm:pcr} we briefly mentioned that the \ac{RTM} is established by either a \ac{H-CRTM} measing the \ac{SRTM} or the \ac{SRTM} measuring.
The latter opens up area for further investigations.


% --------------------------
% Back matter
% --------------------------
%
{%
    \setstretch{1.1}
    \renewcommand{\bibfont}{\normalfont\small}
    \setlength{\biblabelsep}{0pt}
    \setlength{\bibitemsep}{0.5\baselineskip plus 0.5\baselineskip}
    \printbibliography[nottype=online]
    \newrefcontext[labelprefix={@}]
    \printbibliography[heading=subbibliography,title={Webpages},type=online]
}
\clearpage

\microtypesetup{protrusion=false}

\pdfbookmark[0]{List of Figures}{List of Figures}
\listoffigures
\clearpage

\pdfbookmark[0]{List of Tables}{List of Tables}
\listoftables
\clearpage

\pdfbookmark[0]{List of Source Code Listings}{List of Source Code Listings}
\lstlistoflistings
\clearpage

\pdfbookmark[0]{List of Acronyms}{List of Acronyms}
% !TEX root = ../thesis.tex

\chapter{Acronyms}
\label{sec:acronyms}

\begin{acronym}
    \acro{AL}{Afterlife}%
    \acro{BDS}{Boot Device Selection}%
    \acro{BF}{Boot Firmware}%
    \acro{BFV}{\acl{BF} Volume}%
    \acro{BIOS}{Basic Input/Ouput System}%
    \acro{CAR}{Cache as RAM}%
    \acro{CSM}{Compatibility Support Module}%
    \acro{DXE}{Driver Execution Environment}%
    \acro{EFI}{Extensible Firmware Interface}%
    \acro{HOB}{Hand-off Block}%
    \acro{OS}{Operating System}%
    \acro{PEI}{Pre-\acs{EFI} Initialization}%
    \acro{PEIM}{\acl{PEI} Module}%
    \acro{PF}{Platform Firmware}%
    \acro{PI}{Platform Initialization}%
    \acro{PPI}{\acs{PEIM}-to-\acs{PEIM} Interface}%
    \acro{RT}{Runtime}%
    \acro{SEC}{Security}%
    \acro{TSL}{Transient System Load}%
    \acro{UEFI}{Unified \acl{EFI}}%
\end{acronym}
\clearpage
\microtypesetup{protrusion=true}

\appendix\clearpage
% !TEX root = ../thesis.tex

\chapter{Appendix}

\section{Protocols}

\lstinputlisting[language=C,linerange={192-198,298-300, 304-308},caption={Simple Text Input Ex Protocol},captionpos=b,label=lst:simple-text-input-ex-protocol]{deps/edk2/MdePkg/Include/Protocol/SimpleTextInEx.h}

\clearpage

\lstinputlisting[language=C,linerange={59-64,73-73,79-81},caption={Simple File System Protocol},captionpos=b,label=lst:simple-filesytem-protocol]{deps/edk2/MdePkg/Include/Protocol/SimpleFileSystem.h}

\clearpage

\lstinputlisting[language=C,linerange={528-528,534-549},caption={File Protocol},captionpos=b,label=lst:file-protocol]{deps/edk2/MdePkg/Include/Protocol/SimpleFileSystem.h}

\clearpage

\lstinputlisting[language=C,linerange={50-58,78-86,98-98,104-107},caption={Disk \ac{I/O} Protocol},captionpos=b,label=lst:disk-io-protocol]{deps/edk2/MdePkg/Include/Protocol/DiskIo.h}

\clearpage

\lstinputlisting[language=C,linerange={69-78,99-108,214-214,220-220,224-230},caption={Block \ac{I/O} Protocol},captionpos=b,label=lst:block-io-protocol]{deps/edk2/MdePkg/Include/Protocol/BlockIo.h}

\clearpage


\end{document}