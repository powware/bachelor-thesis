% !TeX root = ../../thesis.tex

\subsection{Security}

others not discussed further
user identification

PEI
GuidedSection Extraction


\subsubsection{Secure Boot}
% https://learn.microsoft.com/en-us/windows/security/information-protection/secure-the-windows-10-boot-process
% https://edk2-docs.gitbook.io/understanding-the-uefi-secure-boot-chain/secure_boot_chain_in_uefi/uefi_secure_boot
% https://papers.vx-underground.org/papers/Other/Advanced%20Malware/UEFI%20Secure%20Boot%20in%20Modern%20Computer%20Security%20Solutions.pdf

\cite{tianocore-understanding-uefi-secure-boot-chain}

% workings of secure boot
driver signing
executables may be located on un-secured media
system provider can authenticate either origin or integrity

digital signature
data to sign
public/private key pair used to verify integrity

% how it is signed

embedded within PE file
calculating the pe image hash
- hashing the pe header, omitting the file's checksum and the Certificate Table entry in Optional Header Data Directories
- sorting and hasing pe sections
omitting attribute certifacte table and hash remaining data

\cite{microsoft-pe-signature-format}


% key and hash storage

% how, when and where is it verified

guarantees only valid 3rd party firmware code can run in OEM firmware environment
UEFI Secure Boot assumes the system firmware is a trusted entity
any 3rd party firmware code is not trusted
including bootloader/osloader, PCI option ROMs, UEFI shell tool

two parts
verification of the boot image and verification of updates to the image security database
\cite{understanding-uefi-secure-boot-chain}


\subsubsection{Firmware Protection}

% https://eclypsium.com/2019/10/23/protecting-system-firmware-storage/

DXE SMM Ready to Lock Vol4

Capsule Architectural Protocol

provides
CapsuleUpdate()
QueryCapsuleCapabilities()
of the runtime services table

flash device security

\subsubsection{TPM measurements}
% https://tianocore-docs.github.io/edk2-TrustedBootChain/release-1.00/
% https://tianocore-docs.github.io/edk2-TrustedBootChain/release-1.00/3_TCG_Trusted_Boot_Chain_in_EDKII.html
% https://tianocore-docs.github.io/edk2-TrustedBootChain/release-1.00/6_Checklist_for_Platform_Developers.html

% https://learn.microsoft.com/en-us/windows/security/information-protection/tpm/trusted-platform-module-overview
% what is TPM
A \acf{TPM} is a system component which enables trust in computing platforms
helps verify if the Trusted Computing Base has been compromised
securely storing passwords, certificates and encryption keys in separate state to host
only communicating through a well defined interface.
store platform measurements that help ensure that the platform remains trustworthy
authentication
attestation
hardware and software implementations
software special mode shielding TPM resources from normal execution
\cite{tcg-tpm-summary}
\cite{tcg-tpm-library-part1-architecture}

% what is done with the measurements
% https://learn.microsoft.com/en-us/windows/security/information-protection/bitlocker/bitlocker-overview
how are they used
works with bitlocker to protect user data
ensure computer has not been tampered with while offline

% what is measured
statically configured, unchangeable data
not dynamic and changeable across the boot,
\cite{tianocore-trusted-boot-chain}

% when is it measured
\cite{tianocore-trusted-boot-chain}

% where is it measured
TCG2 Protocol
\cite{tcg-efi-protocol}

% secret storage, seal and unseal