% !TeX root = ../../thesis.tex

\section{Windows}



\subsection{Installation}

For us to understand how UEFI threats act towards Windows we need to understand how the layout of the Windows installation integrates into the UEFI environment.
% uefi begins with windows installation
This begins with the installation process and the partitioning of the hard drive Windows is installed onto.
% creates at least four partitions
When the Windows Installer is launched, it creates at least four partitions on the target hard drive.
The \acf{ESP}, a recovery partition, a partition reserved for temporary storage and the boot partition containing the system files.
Two copies of the Windows Boot Manager \lstinline{bootmgrfw.efi} are placed on the \ac{ESP}, one under \lstinline{EFI\\Boot\\bootx64.efi} for the default boot behavior the installed hard drive and one under \lstinline{EFI\\Microsoft\\Boot\\bootmgrfw.efi} alongside boot resources such as the \ac{BCD}. The path of the latter boot manager is saved in a boot load option variable entry \lstinline{Boot####}, which is then added to the \lstinline{BootOrder} list variable. The boot load option contains optional data consiting of a GUID identifying the Windows Boot Manager entry in the \ac{BCD}. The \ac{BCD}, as its name suggests, contains arguments used to configure various steps of the boot process \cite[12. The Windows Boot Manager]{windows-internals-7-part2}. The boot partition is the primary Windows partition and is formatted with the \ac{NTFS} file system containing the Windows installation. This is also the location of the final step of the Windows UEFI boot process, \lstinline{Windload.efi}, the application repsonsible for loading the kernel into memory \cite[12. The Windows OS Loader]{windows-internals-7-part2}.

\subsection{Boot Process}

Now that we established the basic structure of the Windows UEFI boot environment, we can discuss the boot process. The Windows boot process begins after the UEFI Boot Manager launches the Windows Boot Manager, which starts by retrieving its own executable path and the \ac{BCD} entry GUID from the boot load options. Then it loads the \ac{BCD} and access its entry. If not disabled in the \ac{BCD} it loads its own executable into memory for integrity verification \cite[12. The Windows Boot Manager]{windows-internals-7-part2}. Depending on what hibernation status is set within the \ac{BCD} it may launch the \lstinline{Winresume.efi} application, which reads the hibernation file and resumes kernel execution \cite[12. Launching a boot application]{windows-internals-7-part2}. On a full boot it checks the \ac{BCD} for boot entries, if the entry points to a BitLocker encrypted drive, it attempts decryption. If this faile it will show a reocvery prompt, otherwise it proceeds to load the \lstinline{Windload.efi} \ac{OS} loader \cite[12. Launching a boot application]{windows-internals-7-part2}.

% tries to TPM unseal
% url and message are read from BCD
% if succeeds make sure TPM cant be used to unseal further things
% StartImage os loader, only returns in error case
% if so start Windows Recovery Sequence
\TODO{TPM interaction}
\cite[12. Launching a boot application]{windows-internals-7-part2}


\subsection{Registry}

A crucial part to the whole Windows ecosystem is the Registry, it is a system database containing information required to boot, such as what drivers to load, general system wide configuration as well as application configuration.
\cite[1. Registry]{windows-internals-7-part1}
The Registry is a hierachical database containing keys and values, keys can contain other keys or values, forming a tree structure. Values store data through various data types.
It is comparable to a filesystem structure with keys behaving like directories and values like files \cite[10. The registry - Registry data types]{windows-internals-7-part2}. At the top level it has 9 different keys \cite[10. The registry - Registry logical structure]{windows-internals-7-part2}. Normally Windows users are not required to change Registry values directly and instead interact with it through applications providing setting abstractions. Though some more advanced options may not be exposed and can be accessed through the \lstinline{regedit.exe} application which provides a graphical user interface to travere and modify the Registry \cite[10. The registry - Viewing and changing the registry]{windows-internals-7-part2}. It also supports ex- and importing registry keys along their subkeys and contained values. Internally the registry is not a single large file but instead a set of file called hives, each hive contains one tree, that is mapped into the Registry as a whole. There is no one to one mapping of registry root key to hive file, the \ac{BCD} file for example is also a hive file and is mapped into the Registry under \lstinline{HKEY_LOCAL_MACHINE\BCD00000000} \cite[10. The registry - Registry logical structure]{windows-internals-7-part2}. Some hives even reside entirely in memory as a means of offering hardware configuration through the Registry \ac{API}.



% https://learn.microsoft.com/en-us/windows/whats-new/windows-11-overview#security-and-scanning
% \subsection{User Access Control (UAC)}
% https://learn.microsoft.com/en-us/windows/security/identity-protection/user-account-control/how-user-account-control-works

\subsection{Trusted Boot}
% https://learn.microsoft.com/en-us/windows/security/information-protection/secure-the-windows-10-boot-process
% https://learn.microsoft.com/en-us/windows/security/trusted-boot
% https://www.anoopcnair.com/understanding-windows-trusted-boot/
\subsubsection{KMCI}
\subsubsection{HVCI}
% https://learn.microsoft.com/en-us/windows-hardware/design/device-experiences/oem-hvci-enablement
\subsection{BitLocker}

\cite{microsoft-how-windows-uses-the-tpm}


operating system can only protect when it's active
bitlocker protects system files and data against physical access attacks or generally outside of operating system
windows supports Encrypting File System (EFS) but it cant be used for registry hives
\cite[9. BitLocker Drive encryption]{windows-internals-6-part2}

\cite{microsoft-bitlocker-overview}
\cite{microsoft-bitlocker-device-encryption}
BitLocker Drive Encryption (BDE) integrates with operating system
fixed disk or BitLocker To Go
encryption enabled per volume
encrypt os and data drives
supports removable data drives
maximum protection with TPM 1.2 or later
alternatively USB startup key or password, not system integrity verification
optionally PIN or USB startup key required to unlock

- tpm only
no additional user interaction
- tpm with startup key
additional usb
- tpm with PIN
- tpm with startup key and PIN
protects against unauthorized data access
\cite{microsoft-bitlocker-countermeasures}

with tpm ensures integrity of early boot components and boot configuration

% https://learn.microsoft.com/en-us/windows/security/information-protection/tpm/how-windows-uses-the-tpm

system requirement
include support for TCG-specified Static Root of Trust Measurement

\cite{microsoft-bitlocker-device-encryption}

bitlocker device encryption if supported automatically enabled
after clean install encrypted with clear key (bitlocker suspended state)
non domain account -> recovery key uploaded to microsoft account
domain account -> recovery key backed up to active directory domain services (AD DS)
clear key removed

encryption on used disk space only or whole drive
former security risk if turned on after drive was already in use, deleted data accessible with disk recovery tools
latter the following is recommended
% https://learn.microsoft.com/en-us/windows/security/information-protection/encrypted-hard-drive
encrypted hard drive support

% https://github.com/libyal/libbde/blob/main/documentation/BitLocker%20Drive%20Encryption%20(BDE)%20format.asciidoc
% how does it work
two partitions
- operating system partition with os and support files, all system files on the volume, including the paging files and hibernation files, bitlocker encrypted, ntfs
- system partition with windows boot manager and minimal software required for decryption of the os, fat32, unencrypted, files needed

data is encrypted blockwise with Full Volume Encryption Key (FVEK)
- AES 128-bit the key is 128-bit of size
- AES 256-bit the key is 256-bit of size
FVEK encrypted with Volume Master Key (VMK) 256 bit
VMK encrypted by multiple protectors, default configuration:
% https://learn.microsoft.com/en-us/windows/security/information-protection/tpm/tpm-fundamentals
% https://learn.microsoft.com/en-us/windows/security/information-protection/tpm/how-windows-uses-the-tpm#bitlocker-drive-encryption
- TPM, seal operation
- Recovery Key

or
- startup key/external key

\subsubsection{Recovery Key}
% https://github.com/libyal/libbde/blob/main/documentation/BitLocker%20Drive%20Encryption%20(BDE)%20format.asciidoc#24-recovery-key
recovery key 48 digits of 8 blocks
block is converted to a 16-bit value making up a 128-bit key

\subsubsection{Clear Key}
% https://github.com/libyal/libbde/blob/main/documentation/BitLocker%20Drive%20Encryption%20(BDE)%20format.asciidoc#25-clear-key
unprotected 256-bit key stored on the volume t decrypt vmk

\subsubsection{Startup Key}
% https://github.com/libyal/libbde/blob/main/documentation/BitLocker%20Drive%20Encryption%20(BDE)%20format.asciidoc#26-startup-key
stored in a .bek file with GUID name equaling key identifier in bitlocker meta data
multiple possible for a single bitlocked volume

\subsubsection{Startup Key}
% https://github.com/libyal/libbde/blob/main/documentation/BitLocker%20Drive%20Encryption%20(BDE)%20format.asciidoc#27-user-key
password with max 49 characters