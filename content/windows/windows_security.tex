% !TeX root = ../../thesis.tex

\section{Security}


\begin{figure}[htb]
    \centering
    \includegraphics[width=1.0\textwidth]{windows/windows_startup_process.png}
    \caption{Windows startup process \cite{microsoft-secure-the-windows-boot-process}}
    \label{fig:windows-startup-process}
\end{figure}

\subsection{Secure Boot}

Devices shipping with Windows 11 must have Secure Boot enabled by default \cite{microsoft-windows-minimum-hardware-requirements-overview}.
Windows certified devices generally must allow users to enroll custom keys and signature \acp{DB}, to allow non\-/Windows boot loaders, as well as completely disable secure boot.
Windows offers two signature \acp{DB} the \emph{Microsoft Windows Production PCA 2011} required for the Windows boot process and \emph{Microsoft Corporation \ac{UEFI} \ac{CA} 2011}, which is reserved for third party executables signed at Microsoft's discretion after manual review \TODO{better source} \cite{microsoft-uefi-signing}.
Microsoft advises to only allow other third party \ac{UEFI} applications if necessary and even mandates the exclusion of \acp{DB} other than \emph{Microsoft Windows Production PCA 2011} on Secured\-/core \acp{PC} \cite{microsoft-secure-the-windows-boot-process}.

\subsection{Trusted Boot}

Trusted Boot picks up where Secure Boot ends and maintains the code integrity chain through the kernel into the Windows startup process, by verifying the digital signature of every component.
This includes boot drivers, startup files and the \ac{ELAM} driver \cite{microsoft-trusted-boot}.

\subsubsection{KMCI}
\cite{understanding-windows-trusted-boot}
\subsubsection{ELAM}
\cite{understanding-windows-trusted-boot}

\subsubsection{VSB}
Virtualization-based Security {VBS}
formerly Device Guard
\subsubsection{HVCI}
% https://learn.microsoft.com/en-us/windows-hardware/design/device-experiences/oem-hvci-enablement




\subsection{\acf{BDE}}
\label{sec:windows:security:bde}
Windows is only able to enforce security policies when it is active, leaving the system vulnerable when accessed from outside of the \ac{OS} \cite[Section 9]{windows-internals-6-part2}.
Windows uses BitLocker, integrated \ac{FVE}, aimed to protect system files and data from unauthorized accecss while at rest \cite{microsoft-bitlocker-overview}, while also verifying boot integrity when used with a \ac{TPM} \cite[Section 9]{windows-internals-6-part2}.
The en- and decryption of the volume is done by a filter driver beneath the \ac{NTFS} driver as shown in \autoref{fig:bitlocker-volume-access-driver-stack}.
The \ac{NTFS} driver translates file and directory access into block-wise operations on the volume \TODO{CITE}, the filter driver receives these block operations, encrypting blocks on write and decrypting blocks on read, while they pass through.
This leaves the en- and decryption entirely transparent, making the underlying volume appear decrypted to the \ac{NTFS} driver \cite[Section 9]{windows-internals-6-part2}.
The encryption of each block is done using a modified version of the \ac{AES}128 and \ac{AES}256 cypher \cite[Section 9]{windows-internals-6-part2}.
A \ac{FVEK} is used in combination with the block index as input for the algorithm, resulting in an entirely different output for two blocks with identical data \cite[Section 9]{windows-internals-6-part2}.
The \ac{FVEK} is encrypted with a \ac{VMK} which is in turn encrypted with multiple protectors, these encrypted versions of the \ac{VMK} are stored together with the encrypted \ac{FVEK} in an unencrypted meta data portion at the beginning of the volume \cite[Section 9]{windows-internals-6-part2}.
The \ac{VMK} is encrypted by the following protectors:

\begin{itemize}
    \item[Startup key] stored in a \lstinline{.bek} file with a \ac{GUID} name equaling key identifier in bitlocker meta data
        \cite[Section 2.6]{bde-format-spec}
    \item[TPM]
        - tpm only
        no additional user interaction
        - tpm with startup key
        additional usb
        - tpm with PIN
        - tpm with startup key and PIN
        \cite{microsoft-bitlocker-countermeasures}
        with tpm ensures integrity of early boot components and boot configuration
        tpm usage requires \ac{TCG}2 compliant \ac{UEFI} firmware \cite[Section 9]{windows-internals-6-part2}

        tpm is used to \emph{seal} and \emph{unseal} \ac{VMK}
        \TODO{PCR table either here or at TPM section}
        platform validation profile
        % https://learn.microsoft.com/en-us/windows/security/information-protection/bitlocker/bitlocker-group-policy-settings#allow-secure-boot-for-integrity-validation
        defaults are \acp{PCR} \lstinline|{7, 11}| with PCR7 binding  \lstinline|{0, 2, 4, 11}| without PCR7 binding
        11 is required
    \item[Recovery key] recovery key 48 digits of 8 blocks
        block is converted to a 16-bit value making up a 128-bit key
        \cite[Section 2.4]{bde-format-spec}
        % how to obtain
        when enabling manually, save on non encrypted medium
        \cite{microsoft-bitlocker-basic-deployment}

        bitlocker device encryption if supported automatically enabled
        after clean install encrypted with clear key (bitlocker suspended state)
        non domain account -> recovery key uploaded to microsoft account
        domain account -> recovery key backed up to active directory domain services (AD DS)
        clear key removed
        \cite{microsoft-bitlocker-device-encryption}

    \item[User key] password with max 49 characters
        \cite[Section 2.7]{bde-format-spec}
    \item[Clear key] unprotected 256-bit key stored on the volume to decrypt vmk
        \cite[Section 2.5]{bde-format-spec}
        used for suspension


        \TODO{decide if we add this} With Windows 11 and Windows 10, administrators can turn on BitLocker and the TPM from within the Windows Pre-installation Environment \cite{microsoft-bitlocker-device-encryption}

        \begin{figure}[htb]%
            \centering
            \includesvg[width=0.5\textwidth]{bitlocker_volume_access_driver_stack.drawio.svg}
            \caption{BitLocker Volume Access Driver Stack (inspired by \cite[Figure 9-24]{windows-internals-6-part2})}%
            \label{fig:bitlocker-volume-access-driver-stack}%
        \end{figure}

\end{itemize}


% what is done with the measurements
% https://learn.microsoft.com/en-us/windows/security/information-protection/bitlocker/bitlocker-overview
how are they used
works with bitlocker to protect user data
ensure computer has not been tampered with while offline