% !TeX root = ../thesis.tex

\chapter{Conclusion}

Our practical analysis of \ac{UEFI} threats against Windows 11 showed that enabling Secure Boot when using BitLocker comes with a hidden hit in security.
Microsoft missuses Secure Boot in an attempt to provide platform firmware integrity validation, where the \ac{TPM} already offered a perfectly fine solution.
While also having set a dangerous precedence by offering user the immediate ability to override integrity violations in an inherently untrustworthy system.
Trade\-/off


\section{Achieved Goals}

when we are already in the image we can gain full control over the system
system cant be trusted anymore e.g. uefi services
full file access
escalate it to local system level execution
bitlocker has the flaw of allowing to enter criticial information into an inherently untrustable system
on the other hand one could force such a prompt themselves
mere exisitence of a recovery key is a security flaw

\section{Future Work}

tpm and pin
capsule update
exploit in tpm measruement chain that results in not being measured
can exploit the tg2 hook directly to retrieve the vmk
look into memory based rootkit with bitlocker and tpm and also what the hypervisor kernel security does to them
look further into \ac{SMM}

In \autoref{sec:tpm:pcr} we briefly mentioned that the \ac{RTM} is established by either a \ac{H-CRTM} measing the \ac{SRTM} or the \ac{SRTM} measuring.
The latter opens up area for further investigations.