% !TeX root = ../thesis.tex

\chapter{Conclusion}
\label{sec:conclusion}

Our practical analysis of \ac{UEFI} threats against Windows 11 showed that enabling Secure Boot when using BitLocker comes with a hidden forfeiture of security.
Microsoft misuses Secure Boot in an attempt to provide platform firmware integrity validation, where the \ac{TPM} already offers a perfectly fine solution.
With such a misconfigured BitLocker validation profile our rootkit was able to sniff the communication between the Windows Boot Manager and \ac{TPM} without introducing side effects.
Through interception of the \emph{unseal} command we gained access to the unencrypted BitLocker \ac{VMK}, to decrypt the hard drive and deploy further payload in the Windows installation.
By then modifying the Windows registry our payload was executed with privileges of the local system account.

Microsoft also have set a dangerous precedence by offering the user a mechanism to override the security reaction to integrity violations in an inherently untrustworthy system.
In the case of a correctly configured BitLocker validation profile the code of our root- or bootkit is measured into the \ac{TPM}, causing the \emph{unseal} operation to fail and the Windows Boot Manager to trigger a recovery prompt.
The burden of security enforcement is now left to the user and when they decide put further trust into the system and enter the recover key, our \emph{BitLogger} is able to record the performed keystrokes to decrypt the hard drive.

\section*{Future Work}

Investigations into the \ac{RTM} being established by the \ac{SRTM} measuring itself could reveal flaws further up in the implementation of the measurement chain.