% !TeX root = ../../thesis.tex

\section{\acf{UEFI}}

The \ac{UEFI} specification itself is a pure interface specification, describing the programmatic interface for interaction with the \ac{PF}, merely stating what interfaces and structures a \ac{PF} has to offer and what an \ac{OS} may use \cite{beyond-bios}.

during boot system resources are owned by the firmware

protected mode, takes care of memory management and mapping with a one to one mapping of physical to virtual


% what are its concrete goals
- complete solution describing all features and capabilities
- abstract interfaces to support a range of processors without the need for knowledge about underlying hardware for the bootloader
- sharable persistent storage for platform support code
security
\TODO{MEMORY LAYOUT no memory protection, RWE everywhere}

It was designed to replace the legacy \acl{BF} \ac{BIOS} \TODO{which wasnt very standardized}, while also providing backwards compatibility by defining the \acf{CSM} allowing \ac{UEFI} firmware to boot legacy \ac{BIOS} applications.

system table with boot- and runttime service functions for the bootloader and os to call
datatables containing platform-related information

\cite{beyond-bios}

\subsection{\acf{GUID}}

The \ac{UEFI} environment depends on \acp{GUID}, also known \acp{UUID} to uniquely identify a variety of things, such as protocols, files, hard drive paritions.
\acp{GUID} are 128-bit long, statistically unique identifiers and can be generated on demand and without a centralized authority, statistically guaranteeing that there will be no duplicates on a system that combines hard and software from multiple vendors \cite{rfc-4122}.

\subsection{\acf{GPT}}

Partitions allow a disk to be distinctly separated into logical disks, allowing for each to be formatted with a different file systems.
Prior to \ac{UEFI} disks have been paritioned using the \ac{MBR} parition table, supporting up to 4 different partitions.
The \ac{MBR} is stored within the first sector, also optionally containing 424 bytes of bootable code through which the \ac{BIOS} boots \cite[Section 13.3.1]{uefi-spec}.
\ac{UEFI} is still backwards compatible with \ac{MBR} partitioned disks and contained on each disk, but \ac{UEFI} does not execute the boot code.
The \ac{MBR} is used in two different ways by the \ac{UEFI} environment, either as a legacy \ac{MBR} or a protective \ac{MBR}.
With the legacy \ac{MBR}, \ac{UEFI} uses the partitions defined in the \ac{MBR} parition table, where as the protective \ac{MBR} only has one partition spanning the entire disk.
The protective partition is for legacy devices and in reality \ac{GPT} partitioning is used to separate the disk.
For this \ac{UEFI} defines two \ac{OS} types used in \ac{MBR} parition entries.
One identifies the \ac{ESP}, the parition \ac{UEFI} boots from, within the legacy \ac{MBR} partition table and the other indicates that a protective parition is used \cite[Section 5]{uefi-spec}.
\cite[Section 5]{uefi-spec} defines the \ac{GPT} disk layout, with the \ac{GPT} format \ac{LBA} are 64 bit instead of 32 bit, allowing to support drives with up to 9400000000 \ac{TB} of storage, where as \ac{MBR} is limited to 2 \ac{TB}.
This is accompanied by allowing many more than 4 partitions, with Windows supporting up to 128 \cite{microsoft-windows-and-gpt-faq}.
\ac{GUID} are used to identify paritions and parition types, but also offering a human readable parition name.
\ac{GPT} also has a primary and a backup parition table for redundancy pruposes, the primary table follows the \ac{MBR} sector and the backup is at the end of the disk.

\subsection{\acf{ESP}}

The \ac{ESP} can reside any media that is supported by the \ac{UEFI} firmware and has to be \ac{FAT}32 formatted \cite[Section 13.3]{uefi-spec}.
It must contain an \lstinline{EFI} root directory \cite[Section 13.3.1.3]{uefi-spec} and all \ac{UEFI} applications, that are to be launched directly by the \ac{UEFI} firmware have to be located in subdirectories below the \lstinline{EFI} driectory \cite[Section 13.3.1.3]{uefi-spec}. Drivers and indirectly loaded applcations have no storage restrictions. Vendors are to use vendor\-/specifically named subdirectories within the \lstinline{EFI} directory. Fixed disks have no restrictions on the amount of \acp{ESP} present, whereas removable media is only allowed to have one \ac{ESP}, so that boot behavior is deterministic. In general the \ac{ESP} is identified by a specific \ac{GUID}, but implementations are allowed to support accordingly structured \ac{FAT} partitions. Since there is no limitation on the amount of \acp{ESP}, boot applications can share the drive with their \ac{OS}, or can be accumulated in a single system\-/wide \ac{ESP} \cite[Section 13.3.3]{uefi-spec}.


\subsection{\acs{UEFI} Images}

\ac{UEFI} Images are files containting executable code, they use a subset of the \ac{PE32}+ file format with a modified header signature.
The format comes with relocation tables, making it possible for the images to be executed in place or to be loaded at non pre\-/determined memory addresses.
They support multiple CPU architectures such as IA, ARM, RISC-V and x86.
There are three different subtypes of executables: applications, boot and runtime drivers. They mainly differ by their memory type and how it behaves.
Loading and transferring execution are two separate steps, so that security policies can be applied before executing a loaded image \cite[Section 2.1.1]{uefi-spec}.

Applications are always unloaded when they return execution, while drivers are only unloaded when they return an error code. This allows drivers to install their offered functionality upon intial executions and later calls to these functions jump back into the driver's image which is still loaded.
Boot drivers are unloaded when an \ac{OS} loader application transitions to runtime by taking over the memory management through the call of the boot service function \lstinline{ExitBootServices}, while runtime drivers remain loaded and are translated into the virtual memory mapping. \ac{OS} loaders only return execution in error cases.


\subsection{Protocols and Handles}

When \ac{UEFI} binaries are loaded only the entry point is \emph{linked}, the rest of the communication has to be programmatically discovered through protocol interfaces.
Protocols are created dynamically and provide a mechanism to allow extension of firmware capabilities over time \cite[Section 3.6]{tianocore-edk2-driver-writer-s-guide}.
They are C structures and may contain services, in the form of function pointers, or other data structures, they are identified by \acp{GUID} and stored in a single global database implemented by the firmware \cite{beyond-bios}.
This database is called the handle database, handles describe a logical grouping of one or more protocols \cite[Section 3.6]{tianocore-edk2-driver-writer-s-guide}.
Handles are unique per session and should not be saved across reboots \cite{beyond-bios}.
Multiple instances of a protocol identified by the same \ac{GUID} can exist on different handles, offering the same service on different devices.

\begin{figure}[htb]%
    \centering%
    \includegraphics[width=0.7\textwidth]{uefi/handle_types.jpg}%
    \caption{Handle types (taken from \cite[Figure 3]{tianocore-edk2-driver-writer-s-guide})}%
    \label{fig:handle-types}%
\end{figure}

\cite{tianocore-edk2-driver-writer-s-guide} explains the categories of handles that are formed by the type of protocols that are grouped. \autoref{fig:handle-types} shows these categories.

\begin{description}
    \item[Image handles] are handles of \ac{UEFI} images loaded into memory, as they support the \hyperref[lst:loaded-image-protocol]{Loaded Image Protocol}, giving access to information about the image in memory. This includes the image's address, size, memory type, origin and optional load options.
    \item[Driver handles] are handles that group the \ac{UEFI} Driver Model related protocols (Driver Binding Protocol, the two Component Name Protocols and the two Driver Diagnostics Protocols)
    \item[Driver image handles] are \ac{UEFI} Driver Model related protocols installed onto images loaded in memory.
    \item[Agent handles] is a term used in the \ac{UEFI} Driver Model, they describe tracked consumers of other protocols.
    \item[Controller/Device handles] are interchangably used to refer to physical and virtual devices that offer \ac{I/O} abstraction protocols.
        Physical device handles support the Device Path Protocol for generic path/location information \cite[Section 10.2]{uefi-spec}.
    \item[Service handles] are used for generic hardware unrelated abstractions.
\end{description}

\subsection{\acs{UEFI} Driver Model}

\cite[Section 2.5.2]{uefi-spec} describes the \acs{UEFI} Driver Model, it simplifies the design of device drivers by moving implementation of the device mangement and discovery into the firmware, leaving drivers with only the responsibility to offer interfaces for installation and removal.

We will focus on device drivers, these do not add any new device handles but instead offer protocol abstractions build upon already existing \ac{I/O} abstractions offered by bus drivers.
A driver following the \ac{UEFI} Driver Model is not allowed to interact with any hardware in its entry point and is instead required to install an instace of the Driver Binding Protocol on its own image handle.
The Loaded Image Protocol also offers a field where a driver can provide a function to unload itself.
It may also additionally install the confirguration or diagnostic related protocols.
Runtime drivers usually register a notification function that is triggered when an \ac{OS} loader calls \code{ExitBootServices}, this allows them to translate any allocated memory to their virtual addresses.

The firmware will try to connect device drivers to a controller by iterating over all instances of the Driver Binding Protocol in the handle database and calling the \code{Supported} function of the Driver Binding Protocol on a controller. The device driver then checks whether it supports the controller by for example looking for specific \ac{I/O} abstraction protocols, that it will want to laters use and further abstract.
If the driver supports the device the firmware will call the \code{Start} function of the Driver Binding Protocol to have the driver install its offered protocols on the controller handle.
This is done recursively as the newly installed device driver might now fullfill the requirements for another driver.
The firmware can also call the \code{Stop} of the Driver Binding Protocol function if it wants a driver to uninstall its protocol instance from a controller, an example for this would be another device driver wanting to exclusively manage a controller. This is done by tracking agents of protocols, in other words the drivers who consume a protocol.

\subsection{Systemtable}

The UEFI System Table is an important data structure, it provides access to system configuration information, generic boot and runtime services \cite[Section 3.3]{tianocore-edk2-driver-writer-s-guide}.
It also serves as the entrance in to the \ac{UEFI} environment, as a loaded images receives a only pointer to the system table as well as its image handle through its entry point. Although the Loaded Image Protocol provides and interface to hand optional load options to the image \cite{beyond-bios}.

\TODO{during boot boot and runtime services are available}


\subsubsection{Boot Services}

\ac{UEFI} applcations must use boot services functions to access devices and allocate memory. They are available until an \ac{OS} loader takes control over the system via a call to the boot service \code{ExitBootServices()}, from which on only runtime services are available. \cite[Section 7]{uefi-spec} splits the boot services into five categories:

\begin{description}
    \item [Event, Timer, and Task Priority Services] used to create, close, signal, wait for and check events. Setting timers and raising or restoring task priority levels.
    \item [Memory Allocation Services] to allocate and free pools or whole pages of memory, as well as retrieve the \ac{UEFI} managed memory map.
    \item [Protocol Handler Services] used to install, uninstall and retrieve protocol instance as well as abstractions related to the \ac{UEFI} Driver Model.
    \item [Image Services] to load, unload and start images. Images can also use these to transfer execution back to the firmware or with \code{ExitBootServices()} assume control over the system
    \item [Miscellaneous Services] offer basic memory manipulation, checksum calculation, watchdog timers and monotonic counters.
\end{description}

\subsubsection{Runtime Services}

\TODO{me}

\begin{description}
    \item [Variable Services] used to query, get and set \hyperref[sec:uefi-pi:uefi:variables]{variables}.
    \item [Time Services] used to get and set time as well as a system wakeup timer.
    \item [Virtual Memory Services] relate to enabling virtual memory and translating memory addresses.
    \item [Image Services] to load, unload and start images. Images can also use these to transfer execution back to the firmware or with \code{ExitBootServices()} assume control over the system
    \item [Miscellaneous Runtime Services] offer system reset, a monotonic counter and capsule services. Capsules allow the \ac{OS} to pass data to the firmware, this includes firmware managment related data.
\end{description}

\subsection{Variables}
\label{sec:uefi-pi:uefi:variables}

\ac{UEFI} variables are key/value pairs used to store arbitrary data passed between the \ac{UEFI} firmware and \ac{UEFI} applications.
The data type has to be known beforehand and as such is specified for variables defined in \ac{UEFI}.
The Storage implementation is not specified by \ac{UEFI}, but it must support non\-/volatility, to retain after reboots, or temper resistance if demanded.
Variables are defined by a vendor \ac{GUID}, a name and attributes.
Attributes include their scope (boot, runtime, non-volatile), whether writes require authentication or result in appending data instead of overwriting \cite[Section 8.2]{uefi-spec}.
Architecturally defined \ac{UEFI} variables are called Globally Defined Variables where the vendor \ac{GUID} has the value \code{EFI\_GLOBAL\_VARIABLE} \cite[Section 3.3]{uefi-spec}.

\subsection{Boot Manager}
\label{sec:uefi-pi:uefi:boot-manager}

The \ac{UEFI} boot manager is a firmware component executed after the platform is completely initialized, it decides which \ac{UEFI} drivers or applcations are loaded and when.
The boot behavior is configured through architecturally defined \ac{NVRAM} global variables\cite[Section 3.1]{uefi-spec}.
Each load option entry for a driver or application resides in a variable following the naming scheme of \code{Driver\#\#\#\#} or \code{Boot\#\#\#\#} respectively. Where \code{\#} stands for a hexadecimal digit forming a 4 digit number, requiring leading zeros.
If a firmware implementation allows for the creation of new load options they can then be added to the ordered lists \code{DriverOrder} and \code{BootOrder}, they reference load options and dictate the order in which they are processed.
Driver load options are processed before the boot load options, there also exists the \code{BootNext} variable to override the boot options once.
A general depiction of the \ac{UEFI} boot flow can be seen in \autoref{fig:uefi-boot-sequence}.
Implementations usually allow for an interactive menu, where users can modify the order or boot entries manually \cite[Section 3.1.1]{uefi-spec}.
Boot options are generally first attempted to be loaded through the \code{LoadImage} boot service.
If the device path of a boot option only points to a device instead to the file on a device, it attempts to load a default boot application with the \hyperref[lst:simple-file-system-protocol]{Simple File System Protocol}\cite[Section 3.1.2]{uefi-spec}, for x64 it uses the default path \code{\textbackslash EFI\textbackslash BOOT\textbackslash BOOTX64.EFI} \cite[Section 3.5]{uefi-spec}.




\begin{figure}[htb]%
    \centering%
    \includegraphics[width=0.8\textwidth]{uefi_boot_sequence}%
    \caption{Booting Sequence (taken from \cite[Figure 2-1]{uefi-spec})}%
    \label{fig:uefi-boot-sequence}%
\end{figure}


% !TeX root = ../../thesis.tex

\subsection{Security}

\ac{UEFI} offers security mechanisms restricting what is allowed to modify and be modified on a system.
This involves authentication of ownership over the platform.

\subsubsection{Secure Boot}
\label{sec:uefi-pi:uefi:secure-boot}

Secure Boot provides a secure hand-off from the firmware to 3rd party applications used during the boot process, located on insecure media \cite{tianocore-understanding-uefi-secure-boot-chain} \cite[Sections 32.2 and 32.5.1]{uefi-spec}.
It assumes the firmware to be a trusted entity and all 3rd party software to be untrusted, this includes images from hardware vendors in the \ac{PCI} option \acp{ROM}, bootloader from \ac{OS} vendors, and tools such as the \ac{UEFI} shell \cite{tianocore-understanding-uefi-secure-boot-chain}.
Digital signatures, embedded within the \ac{UEFI} images, can be used to authenticate origin and/or integrity \cite[Section 32.2]{uefi-spec}.
This is done through asymmetric signing. Component providers must sign their executables with their private key and publish the public key.
The public keys are stored in a signature \ac{DB} and the signed executable can be verified against the database before execution.
Multiple signatures can be embedded within the same image \cite[Section 32.2.2]{uefi-spec}.
The signatures are created by first calculating a hash over select parts of the executable and then signing it with a private key.
The output of this hashing is called a digest and the algorithm for obtaining the digest is defined in \cite{microsoft-pe-signature-format}.
Secure Boot also disallows legacy booting through the \ac{CSM}.

Secure Boot is managed through three components: a \ac{PK}, one or more \ac{KEK}, and the signature \acp{DB}.

\begin{description}
    \item[\acl{PK}]
        The \ac{PK} establishes a trust relationship between platform owner and firmware, the public half is enrolled into the firmware.
        The private half represents platform ownership, as it can be used to change or delete the \ac{PK} as well as enroll or modify \acp{KEK}.
    \item[\acl{KEK}]
        The \ac{KEK} establishes a trust relationship between \ac{OS} and firmware, as its private half is used to modify the signature \acfp{DB}.
    \item[Signature \aclp{DB}]
        Signature \acp{DB} contain image hashes and certificates, to either allow or deny execution of associated images.
\end{description}

Internally, these are all implemented by authenticated variables, residing in tamper resistant non-volatile storage \cite[Section 32.3]{uefi-spec}.
The \ac{PK} is a simple variable where the \ac{KEK} and \ac{DB} are implemented through signature list data structures \cite[Section 32.4.1]{uefi-spec}.
The variable services can be used to append entries or to read and write the list as a whole \cite[Sections 32.3.5 and 32.5.3]{uefi-spec}.
The variables are part of the \hyperref[sec:uefi-pi:uefi:variables]{Globally Defined Variables}, for each variable there also exists a variant reserved for default entries.
These can be used by an \ac{OEM} to supply platform\-/defined values, used for example during Secure Boot initialization by a user.
Their contents can be copied to their live versions, to then be used during Secure Boot operation.
The current state of Secure Boot is communicated with a secure variable, which the \ac{OS} can probe \cite[Section 3.3]{uefi-spec}.

Users, who are physically present, may disable Secure Boot as well as enroll default or custom keys via an interactive menu \cite[Section 3.3]{uefi-spec}.

\subsubsection{Firmware Management}

The \nameref{lst:firmware-management-protocol} provides a boot abstraction for authenticated updating and management of the \ac{PF} \cite[Section 23]{uefi-spec}.
The runtime services \code{QueryCapsuleCapabilities()} and \code{CapsuleUpdate()} may be used by the \ac{OS} to pass updates to the firmware in a persistent manner, so that they can be processed on subsequent boots \cite[Section 23.3]{uefi-spec}.
\acp{OEM} also often provide their own ways to process firmware updates, for example via dedicated \ac{USB} ports, which allow to process firmware updates upon boot.
As these are entirely dictated by the platform designer, it is not possible to make vendor independent assessments about their security.

\subsubsection{User Identity Policies}

\ac{UEFI} enables a system to have multiple users with varying levels of privileges.
This may restrict their ability to enroll other users or to boot from select drives \cite[Section 36.1.2]{uefi-spec}.
A trusted environment must be maintained for the integrity of the security identification, by restricting which drivers are loaded and securing the storage of drivers \cite[Section 36.1.4]{uefi-spec}.