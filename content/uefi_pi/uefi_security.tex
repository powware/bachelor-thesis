% !TeX root = ../../thesis.tex

\subsection{Security}

\subsubsection{Secure Boot}

Secure Boot provides a secure hand-off from the firmware to 3rd party applications used for during the boot process, located on unsecure media \cite{tianocore-understanding-uefi-secure-boot-chain}\cite[32.2 and 32.5.1]{uefi-spec}.
It assumes the firmware to be a trusted entity and all 3rd party software to be untrusted, this includes images from hardware vendors in \ac{PCI} option \acp{ROM}, bootloader from \ac{OS} vendors and tools such as the \ac{UEFI} shell \cite{tianocore-understanding-uefi-secure-boot-chain}.
Digital signatures, embedded within the \ac{UEFI} images, can be used to authenticate origin and/or integrity \cite[32.2]{uefi-spec}.
This is done through asymmetric signing, component provider must sign their executables with their private key and publish the public key.
The public keys are stored in a signature \ac{DB} and before execution the signed executable can be verified against the database.
Multiple signatures can be embedded within the same image \cite[32.2.2]{uefi-spec}.
The signatures are created by first calculating a hash over select parts of the executable, leaving, for example, the signatures out of the hashed data and then signing it with a private key.
The output of this hashing is called a digest and the algorithm for obtaining the digest is defined in \cite{microsoft-pe-signature-format}.
Secure Boot also disallows legacy booting through the \ac{CSM}.

Secure Boot is managed through three components, a \ac{PK}, one or more \ac{KEK} and the signature \acp{DB}.

\begin{description}
    \item[\ac{PK}]
        The \ac{PK} establishes a trust relationship between platform owner and firmware, the public half is enrolled into the firmware.
        The private half represents platform ownership, as it can be used to change or delete the \ac{PK} as well as enroll or modify \acp{KEK}.
    \item[\ac{KEK}]
        The \ac{KEK} establishes a trust relationship between \ac{OS} and firmware, as its private half is used to modify the signature \acp{DB}.
    \item[Signature \acfp{DB}]
        \ac{DB} \ac{DBX} \ac{DBR} \ac{DBT}
        signature database consists of zero or more signature lists
        \TODO{WHICH DATA BASES DO WHAT}
\end{description}

Internally these are all implemented through authenticated variables, residing in tamper resistant non-volatile storage \cite[32.3]{uefi-spec}.
The \ac{PK} is a simple variable where the \ac{KEK} and \ac{DB} are implemented through signature list data structures \cite[32.4.1]{uefi-spec}, the variable services can be used to append entries or to read and write the list as a whole \cite[32.3.5 and 32.5.3]{uefi-spec}.
The variables are part of the \hyperref[sec:uefi-pi:uefi:variables]{Globally Defined Variables}, for each variavble also exist a variant reserved for default entries. These can be used by an \ac{OEM} to supply platform\-/defined values, used during Secure Boot initialization.
Their contents can be copied to their live versions, used during Secure Boot operation.
The current state of Secure Boot is also reflected within a secure variable \cite[3.3]{uefi-spec}.

% https://papers.vx-underground.org/papers/Other/Advanced%20Malware/UEFI%20Secure%20Boot%20in%20Modern%20Computer%20Security%20Solutions.pdf
Users, who are physically present, may disable Secure Boot, enroll default or custom keys via the \ac{BIOS}\TODO{find a good cite}

% The firmware may approve UEFI images for other reasons than those specified here. For example:
% whether the image is in the system flash, whether the device providing the UEFI image is secured
% (in a case, etc.) or whether the image contains another type of platform-supported digital signa-
% ture
\cite[32.5.3.2]{uefi-spec}

\TODO{secure boot authorization process}

% 1. Reset. This is when the platform begins initialization during boot.
% 2. Key Store Initialization. During the firmware initialization and before any signed UEFI images
% are initialized, the platform firmware must validate the signature database.
% 3. UEFI Image Validation Succeeded? During initialization of an UEFI image, the UEFI Boot
% Manager decides whether or not the UEFI image should be initialized. By comparing the
% calculated UEFI image signature against that in one of the signature databases, the firmware
% can determine if there is a match.
% The security database db must either contain an entry with a hash value of the image (with a
% supported hash type), or it must contain an entry with a certificate against which an entry in
% the image’s certificate table can be verified. In either case verification must not succeed if the
% security database dbx contains any record with:
% A. Any entry with SignatureListType of EFI_CERT_SHA256_GUID with any
% SignatureData containing the SHA-256 hash of the binary.
% B. Any entry with SignatureListType of EFI_CERT_X509_SHA256,
% EFI_CERT_X509_SHA384, or EFI_CERT_X509_SHA512, with any SignatureData
% which reflects the To-Be-Signed hash included in any certificate in the signing chain of
% the signature being verified.
% C. Any entry with SignatureListType of EFI_CERT_X509_GUID, with SignatureData
% which contains a certificate with the same Issuer, Serial Number, and To-Be-Signed hash
% included in any certificate in the signing chain of the signature being verified.
% Multiple signatures are allowed to exist in the binary’s certificate table (as per PE/COFF Section
% “Attribute Certificate Table”). Only one hash or signature is required to be present in db
% in order to pass validation, so long as neither the SHA-256 hash of the binary nor any
% present signature is reflected in dbx.
% Then, based on this match or its own policy, the firmware can decide whether or not to launch
% the UEFI image.
% 4. Start UEFI Image. If the UEFI Image is approved, then it is launched normally.
% 5. UEFI Image Not Approved. If the UEFI image was not approved the platform firmware may use
% other methods to discover if the UEFI image is authorized, such as consult a disk-based catalog
% or ask an authorized user. The result can be one of three responses: Yes, No or Defer.
% 6. UEFI Image Signature Added To Signature Database. If the user approves of the UEFI image,
% then the UEFI image’s signature is saved in the firmware’s signature database. If user approval
% is supported, then the firmware be able to update of the Signature Database. For more
% information, see Signature Database Update.
% 7. Go To Next Boot Option. If an UEFI image is rejected, then the next boot option is selected
% normally and go to step 3. This is in the case where the image is listed as a boot option.
% 8. UEFI Image Signature Passed In System Configuration Table. If user defers, then the UEFI image
% signature is copied into the Image Execution Information Table in the EFI System Configuration
% Table which is available to the operating system.
% 9. OS Application Validates UEFI Image. An OS application determines whether the image is valid.
% 10. UEFI Image Signature Added To Signature Database. For more information, see Signature
% Database Update.
% 11. End.