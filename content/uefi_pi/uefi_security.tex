% !TeX root = ../../thesis.tex

\subsection{Security}

\ac{UEFI} offers security mechanisms restricting what is allowed to modify and be modified on a system.
This involves authentication of ownership over the platform.

\subsubsection{Secure Boot}
\label{sec:uefi-pi:uefi:secure-boot}

Secure Boot provides a secure hand-off from the firmware to third\-/party applications used during the boot process, located on insecure media \cite{tianocore-understanding-uefi-secure-boot-chain} \cite[Sections 32.2 and 32.5.1]{uefi-spec}.
It assumes the firmware to be a trusted entity and all third\-/party software to be untrusted, this includes images from hardware vendors located in \acsu{PCI} option\-/\acsp{ROM}\acused{ROM}, bootloaders from \ac{OS} vendors, and tools such as the \ac{UEFI} shell \cite{tianocore-understanding-uefi-secure-boot-chain}.
Digital signatures, embedded within the \ac{UEFI} images, can be used to authenticate the origin and/or integrity \cite[Section 32.2]{uefi-spec}.
This is done through asymmetric signing.
Component providers must sign their executables with their private keys and publish the public keys.
The public keys are stored in a signature \ac{DB} and the signed executable can be verified against the database before execution.
Multiple signatures can be embedded within the same image \cite[Section 32.2.2]{uefi-spec}.
The signatures are created by first calculating a hash over select parts of the executable and then signing it with a private key.
The output of this hashing is called a digest and the algorithm for obtaining the digest is defined in \cite{microsoft-pe-signature-format}.
Secure Boot also disallows legacy booting through the \ac{CSM}.

Secure Boot is managed through three components: a \ac{PK}, one or more \ac{KEK}, and the signature \acp{DB}.

\begin{description}
    \item[\acl{PK}]
        The \ac{PK} establishes a trust relationship between the platform owner and firmware, the public half is enrolled into the firmware.
        The private half represents platform ownership, as it can be used to change or delete the \ac{PK} as well as enroll or modify \acp{KEK}.
    \item[\acl{KEK}]
        The \ac{KEK} establishes a trust relationship between \ac{OS} and firmware, as its private half is used to modify the signature \acfp{DB}.
    \item[Signature \aclp{DB}]
        Signature \acp{DB} contain image hashes and certificates, to either allow or deny execution of associated images.
\end{description}

Internally, these are all implemented by authenticated variables, residing in tamper\-/resistant non-volatile storage \cite[Section 32.3]{uefi-spec}.
The \ac{PK} is a simple variable where the \ac{KEK} and \ac{DB} are implemented through signature list data structures \cite[Section 32.4.1]{uefi-spec}.
The variable services can be used to append entries or to read and write the list as a whole \cite[Sections 32.3.5 and 32.5.3]{uefi-spec}.
The variables are part of the \hyperref[sec:uefi-pi:uefi:variables]{Globally Defined Variables}, for each variable, there also exists a variant reserved for default entries.
These can be used by an \ac{OEM} to supply platform\-/defined values, used during Secure Boot initialization.
Their contents can be copied to their live versions, to then be used during Secure Boot operation.
The current state of Secure Boot is communicated with a secure variable, which the \ac{OS} can probe \cite[Section 3.3]{uefi-spec}.

Users, who are physically present, may disable Secure Boot as well as enroll default or custom keys via an interactive menu \cite[Section 3.3]{uefi-spec}.

\subsubsection{Firmware Management}

The \nameref{lst:firmware-management-protocol} provides a boot abstraction for authenticated updating and management of the \ac{PF} \cite[Section 23]{uefi-spec}.
The runtime services \code{QueryCapsuleCapabilities()} and \code{CapsuleUpdate()} may be used by the \ac{OS} to pass updates to the firmware in a persistent manner, so that they can be processed on subsequent boots \cite[Section 23.3]{uefi-spec}.
\acp{OEM} also often provide their own ways to process firmware updates, for example via dedicated \acsu{USB} ports, which allow processing of firmware updates upon boot.
As these are entirely dictated by the platform designer, it is not possible to make vendor\-/independent assessments about their security.

\subsubsection{User Identity Policies}

\ac{UEFI} enables a system to have multiple users with varying levels of privileges.
This may restrict their ability to enroll other users or to boot from select drives \cite[Section 36.1.2]{uefi-spec}.
A trusted environment must be maintained for the integrity of the security identification, by restricting which drivers are loaded and securing the storage of drivers \cite[Section 36.1.4]{uefi-spec}.