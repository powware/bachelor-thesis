% !TeX root = ../../thesis.tex

\subsection{Security}

The \ac{PI} specification does not attempt to define a coherent security system and instead only provides optional services and mechanisms for a system designer to use.

\subsubsection{Hardware Validated Boot}

The \ac{PI} specification defines \acp{PPI} and \ac{DXE} protocols which can be used to validate loaded images before exeuting them.
During the \ac{PEI} phase the \emph{\ac{PEI} Guided Section Extraction \ac{PPI}} can be used for the authentication of file section, while the \emph{Security \ac{PPI}} implements the policy response to the authentication result.
The \ac{DXE} phase has similar counter parts in the form of \emph{Guided Section Extraction Protocol}, \emph{Security Architectural Protocol} and \emph{Security2 Architectural Protocol}
These allows the platform designer to implement policy responses following image authentication, ranging from deferral of execution to locking the system flash.


Secure Boot relies for the firmware as its root of trust, hardware validated boot shifts this trusts out of the firmware image into hardware.
amd
% https://ebrary.net/24869/computer_science/secure_technology
% https://www.amd.com/system/files/documents/amd-security-white-paper.pdf
intel
% https://edk2-docs.gitbook.io/understanding-the-uefi-secure-boot-chain/secure_boot_chain_in_uefi/intel_boot_guard

\subsubsection{Firmware Protection}

The \ac{PI} speficiation defines a \emph{End of Dxe Event}, it indicates the introduction of third party software execution to the platform.
Up until this point it is assumed that the entire system software is under the control of the platform manufacturer.
Drivers may react to this event by locking critical system resources, by using the \ac{SMM} services \cite[Vol. 2, 5.1.2.1]{pi-spec}.
The \ac{SMM} is a secure execution environment, achieved by isolation from the rest of the system, through the \ac{CPU} \cite[Vol. 4, Section 1.3]{pi-spec}.


% https://papers.vx-underground.org/papers/Other/Advanced%20Malware/UEFI%20Secure%20Boot%20in%20Modern%20Computer%20Security%20Solutions.pdf
% NIST 800-147 BIOS Protection Guidelines [15]
% NIST 800-147B BIOS Protection Guidelines for Servers [16]
% NIST 800-155 BIOS Integrity Measurement Guidelines [17]


% https://eclypsium.com/2019/10/23/protecting-system-firmware-storage/

