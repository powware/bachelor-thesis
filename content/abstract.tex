% !TEX root = ../thesis.tex

% http://williamstallings.com/Extras/Abstract.html
% https://www.enago.com/academy/abstract-versus-introduction-difference/

\pdfbookmark[0]{Abstract}{Abstract}
\chapter*{Abstract}
\label{sec:abstract}
\thispagestyle{empty}

% motivation
In Computer Security firmware attacks are one of the most feared security threats, executing during the boot process, they can already have full control over the system before an operating system and accompanying antivirus programs are even loaded.
With widespread adaption of standardized \acs{UEFI} firmware these threats have become less machine dependent, and able to target a host of systems at once.
% problem statement
Their appearances in the wild are rare as they are stealthy by nature. We categorize past analyses of \acs{UEFI} threats (against Windows) by their attack vector and perform our own.
% approach
With a deep-dive into the \acs{UEFI} environment we learn hands on about encountered security mechanisms targeting pre-boot attacks, setting our focus on Secure Boot and \acs{TPM}-assisted BitLocker.
% results
We were able to achieve system level privileged execution on Windows 11 by exploiting unrestricted hard drive access to deploy our payload and modify the Windows Registry. With BitLocker enabled, our \emph{BitLogger} was able to decrypt and mount the drive using a keylogged Recovery Key, or when part of the chain of trust using a \acs{VMK} sniffed from \acs{TPM} communication.
% conclusions
\acs{UEFI} threats are very powerful and discredit all system integrity, making it impossible to put any further trust into the system.

\acresetall

\pdfbookmark[0]{Zusammenfassung}{Zusammenfassung}
\chapter*{Zusammenfassung}
\thispagestyle{empty}
\label{sec:zusammenfassung}

\blindtext

\acresetall