% !TEX root = ../thesis.tex

% http://williamstallings.com/Extras/Abstract.html
% https://www.enago.com/academy/abstract-versus-introduction-difference/

\pdfbookmark[0]{Abstract}{Abstract}
\addchap*{Abstract}
\label{sec:abstract}

% motivation
In Computer Security one of the most feared threats is a bootkit, executing at the beginning of a computers boot chain, before the operating system and accompanying antivirus programs. With the wide spread adaption of standardized UEFI firmware these threats have become less machine dependent and can now target a host of systems at once.
% problem statement
Past analyses about bootkits have been case studies of their appearences in the wild, this thesis instead aims to be a more practical approach by developing a bootkit and analysing the challenges doing so.
% approach
We restrict our analysis by assuming an attacker has already gained read and write access to the BIOS image and is thus only facing security mechanisms involved during and with execution of the bootkit.
% results
Our bootkit was able to achieve elevated execution on Windows 11 by exploiting unrestricted hard drive access to edit Windows Registries, this was also possible on Bitlocker encrypted hard drives by keylogging the Recovery Key.
% conclusions
UEFI makes it very easy for an attacker who has gained access to the System Firmware to leverage it's powers and gain full control over the system.

\vspace*{20mm}

{\usekomafont{chapter}Abstract (different language)}
\label{sec:abstract-diff}

\blindtext
