% !TEX root = ../thesis.tex

% http://williamstallings.com/Extras/Abstract.html
% https://www.enago.com/academy/abstract-versus-introduction-difference/

\pdfbookmark[0]{Abstract}{Abstract}
\addchap*{Abstract}
\label{sec:abstract}

\TODO{past tense and rewrite}
% motivation
In Computer Security malicious firmware is one of the most feared security threats, executing at the beginning of a computers boot chain, before the operating system and accompanying antivirus programs. With the widespread adaption of standardized \acs{UEFI} firmware these threats have become less machine dependent and can now target a host of systems at once.
% problem statement
Their appearances in the wild are rare as they are very hard to detect by nature, we categorize analyses of past UEFI threats by their attack vector and perform our own attacks against Windows 11.
% approach
By performing a deep-dive into the UEFI environment we learn hands on about security mechanisms targeting pre-boot attacks, with the focus on Secure Boot and TPM-assisted BitLocker.
% results
We were able to achieve system level privileged execution on Windows 11 by exploiting unrestricted hard drive access to deploy our payload and edit the Windows Registry. With BitLocker enabled, our \emph{BitLogger} was able to decrypt and mount the drive using a keylogged Recovery Key.
% conclusions
UEFI threats are very powerful and can take away all system integrity, making it impossible to put further trust on the system.

\vspace*{20mm}

{\usekomafont{chapter}Abstract (deutsch)}
\label{sec:abstract-diff}

\blindtext
