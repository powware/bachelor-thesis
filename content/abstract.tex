% !TEX root = ../thesis.tex

% http://williamstallings.com/Extras/Abstract.html
% https://www.enago.com/academy/abstract-versus-introduction-difference/

\pdfbookmark[0]{Abstract}{Abstract}
\chapter*{Abstract}
\label{sec:abstract}
\thispagestyle{empty}

% motivation
Firmware attacks are one of the most feared threats in Computer Security.
Executing during the boot process, they can already have full control over the system before an operating system and any accompanying antivirus programs are even loaded.
With the widespread adaption of standardized \acf{UEFI} firmware, these threats have become less machine\-/dependent, and able to target multiple different systems at once.
% problem statement
Their appearances in the wild are still rare as they are stealthy by nature.
Through analyses of past threats, we can categorize their attack vectors against Windows and perform our practical analysis.
% approach
With a deep\-/dive into the \ac{UEFI} environment, we learn hands\-/on about encountered security mechanisms targeting pre\-/boot attacks, setting our focus on Secure Boot and \acf{TPM}\-/assisted BitLocker.
% results
We were able to achieve system\-/level privileged execution on Windows 11 by exploiting unrestricted hard drive access to deploy our payload and modify the Windows Registry.
With BitLocker enabled, our \emph{BitLogger} was able to decrypt and mount the drive using a keylogged Recovery Key.
When BitLocker is misconfigured or our code is part of the trusted measurements, we could sniff the \acs{TPM} communication to retrieve the \acs{VMK}.
% conclusions
Our thesis shows, that \acs{UEFI} threats are very powerful, to a point where they discredit all system integrity, making it impossible to put any further trust into the system.

\acresetall

\pdfbookmark[0]{Zusammenfassung}{Zusammenfassung}
\chapter*{Zusammenfassung}
\thispagestyle{empty}
\label{sec:zusammenfassung}

Firmware Attacken sind eine der gef\"urchtesten Sicherheitsrisiken in der Computersicherheit.
Sie werden w\"ahrend des Bootvorgangs ausgef\"uhrt und k\"onnen Kontrolle \"uber das gesamte System erlangen, bevor das Betriebssystem und seine begleitenden Antivirusprogramme geladen worden sind.
Mit der zunehmenden Verbreitung von standardisierter \acf{UEFI} Firmware sind diese Bedrohungen weniger ger\"ateabh\"anig geworden und k\"onnen nun eine Vielzahl von Systemen gleichzeitig adressieren.
Das Vorkommnis solcher Attacken ist weiterhin recht selten, da sie von Natur aus schwierig aufzufinden sind.
Durch Analysen bisheriger Funde k\"onnen wir ihre Angriffsvektoren kategorisieren und unsere praktische Analyse durchf\"uhren.
Wir tauchen in die \ac{UEFI}\-/Welt ein, um durch die Implementierung eigener Attacken die Sicherheitsmechanismen des Bootvorgangs besser zu verstehen.
Dabei setzen wir unseren Fokus auf Secure Boot und BitLocker unter Verwendung des \acf{TPM}.
Als Resultat unserer Attacken erreichten wir Codeausf\"uhrung mit erh\"ohten Privilegien unter Windows 11.
Mittels unbeschr\"ankten Zugriffs auf die Festplatte konnten wir weitere Benutzerebnenprogramme installieren und die Windows Registry modifizieren.
Wenn BitLocker aktiv ist, erm\"oglichte unser \emph{BitLogger} das Entschl\"usseln der Festplatten durch einen mitgeloggten Recovery Key.
Bei einer Fehlkonfiguration BitLockers oder, wenn unser Schadcode Teil der vertrauten Messungen ist, k\"onnen wir die \ac{TPM}\-/Kommunikation mitlesen, um an den \acf{VMK} zu gelangen.
Unsere Arbeit zeigt, dass \ac{UEFI}\-/Bedrohungen sehr m\"achtig sind und durch sie jegliche Systemintegrit\"at so diskreditiert wird, dass es unm\"oglich ist dem System weiter zu vertrauen.

\acresetall