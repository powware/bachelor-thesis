% !TEX root = ../thesis.tex

% http://williamstallings.com/Extras/Abstract.html
% https://www.enago.com/academy/abstract-versus-introduction-difference/

\pdfbookmark[0]{Abstract}{Abstract}
\chapter*{Abstract}
\label{sec:abstract}
\thispagestyle{empty}

% motivation
In Computer Security firmware attacks are one of the most feared security threats, executing during the boot process, they can already have full control over the system before an operating system and accompanying antivirus programs are even loaded.
With widespread adaption of standardized \acs{UEFI} firmware these threats have become less machine dependent, and able to target a host of systems at once.
% problem statement
Their appearances in the wild are still rare as they are stealthy by nature.
Through past analyses, we are able to categorize \acs{UEFI} threats (against Windows) by their attack vector and perform our own.
% approach
With a deep\-/dive into the \acs{UEFI} environment we learn hands on about encountered security mechanisms targeting pre\-/boot attacks, setting our focus on Secure Boot and \acs{TPM}\-/assisted BitLocker.
% results
We were able to achieve system level privileged execution on Windows 11 by exploiting unrestricted hard drive access to deploy our payload and modify the Windows Registry.
With BitLocker enabled, our \emph{BitLogger} was able to decrypt and mount the drive using a keylogged Recovery Key.
When either BitLocker is misconfigured or we are part of the chain of trust of measurement, we could sniff the \acs{TPM} communication to retrieve the \acs{VMK}.
% conclusions
\acs{UEFI} threats are very powerful and discredit all system integrity, making it impossible to put any further trust into the system.

\acresetall

\pdfbookmark[0]{Zusammenfassung}{Zusammenfassung}
\chapter*{Zusammenfassung}
\thispagestyle{empty}
\label{sec:zusammenfassung}

Firmware Attacken sind eine der gef\"urchtesten Sicherheitsrisiken in der Computersicherheit.
Sie werden w\"ahrend des Bootvorgangs ausgef\"uhrt, wodurch sie Kontrolle \"uber das gesamte System erlangen k\"onnen, bevor das Betriebssystem und seine begleitenden Antivirusprogramme geladen worden sind.
Mit der zunehmenden Etablierung der standardisierten \acs{UEFI}\-/Firmware wurden diese Bedrohungen unabh\"aniger von der zugrunde liegenden Machinenarchitektur und k\"oennen nun verschiedene Arten von Systemen gleichzeitig adressieren.
Das Vorkommnis solcher Attacken ist weiterhin recht selten, da sie von Natur aus schwierig auffindig zu machen sind.
Durch Analysen bisheriger Funde k\"oennen wir ihre Angriffsvektoren kategorisieren und unsere eigene Attacke implementieren.
Wir tauchen tief in die \acs{UEFI} Welt ein und lernen durch die praktische Durchf\"uhrung \"uber die Sicherheitsmechanismen, welche den Bootvorgang besch\"utzen.
Dabei setzen wir unseren Fokus auf Secure Boot und BitLocker unter Verwendung des \acs{TPM}.
Als Resultat unserer Attacken erreichten wir Codeausf\"uhrung mit erh\"ohten Privilegien unter Windows 11.
Mittels unbeschr\"ankten Zugriffs auf die Festplatte konnten wir weitere Benutzerebnenprogramme installieren und die Windows Registry modifizieren.
Mit BitLocker angeschaltet, erm\"oglichte unser \emph{BitLogger} das Entschl\"usseln der Festplatten durch einen mitgelauschten Recovery Key.
Bei einer Fehlkonfiguration BitLockers oder im Fall, dass unser Schadcode Teil der Vertrauenskette ist, war es uns m\"oglich die Kommunikation zum \acs{TPM} mitzuschneiden, um an den \acs{VMK} zu gelangen.
\acs{UEFI} Attacken sind sehr m\"achtig und diskreditieren die Systemintegrit\"at, wodurch kein weiteres Vertrauen in das System m\"oglich ist.

\acresetall