% !TeX root = ../thesis.tex

\chapter{Results}

We were able to implement \ac{UEFI} attacks in the form of a \ac{UEFI} firmware rootkit and a \ac{UEFI} bootloader rootkit (bootkit), with both being able to deploy Windows level payload from within the \ac{UEFI} environment using an \ac{NTFS} drivers.
Through our \ac{UEFI} port of the \lstinline{reged} utility we were able to modify the Windows registry, so that our Windows payload is executed with the privileges of the built-in local system account.
The execution is done by the Task Scheduler at system boot.
With Secure Boot enabled we showed that our bootkit was denied execution, while the execution of our rootkit is left unaffected.
Although restrictions are applied \ac{SPI} flash access.
When BitLocker is used with a \ac{TPM} and the default validation profile \lstinline{0, 2, 4, 11} our root- and bootkit trigger the BitLocker recovery prompt, from which we were able to retrieve the recovery key to use it with our \ac{UEFI} port of Dislocker to mount the encrypted drive, allowing us to perform the same attack as on unencrypted drives.
When our payload is part of the \ac{PCR} values used to encrypt the \ac{VMK} or when a misconfigured validation profile is used, we were able to sniff the communication between the \ac{TPM} and the Windows Boot Manager to retrieve the unencrypted \ac{VMK} for use with Dislocker.
We showed that Secure Boot, that Microsoft labels as the preferred configuration, forcing a default validation profile of \lstinline{7, 11} left the system vulnerable to \ac{TPM} sniffing through our rootkit.