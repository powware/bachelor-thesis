% !TeX root = ../thesis.tex

\chapter{Results}

We were able to implement \ac{UEFI} attacks in the form of a \ac{UEFI} firmware rootkit and a \ac{UEFI} bootloader rootkit (bootkit).
Both were able to deploy Windows level payload from within the \ac{UEFI} environment using an \ac{NTFS} drivers.
By porting the Linux utility \program{reged} to \ac{UEFI} we were able to modify the Windows Registry.
Through the manipulation of a Task Scheduler master registry key we were able to have our Windows payload executed with the privileges of the built-in local system account.
The execution happens at system boot before login.

We showed that the attack using our rootkit is unaffected by Secure Boot, as Secure Boot relies on the \ac{PF} as its root of trust.
This leaves Secure Boot unable to verify the behavior of the \ac{PI} when it includes our rootkit.
As Secure Boot's threat model consists of the hand\-/over process from the plaftorm firmware to \ac{UEFI} images, the attack with our bootkit was successfully mitigated.

We showed that a BitLocker is used with a \ac{TPM} and the default validation profile \code{\{0, 2, 4, 11\}} our root- and bootkit trigger the BitLocker recovery prompt, from which we our \emph{BitLogger} was able to log entered keystrokes to obtain the recovery key.
We were then able to use the recovery key with our \ac{UEFI} port of Dislocker to mount the encrypted drive, allowing us to repeat our initial attack of deploying payload and modifying the registry.
In the case of our \ac{UEFI} payload being part of the \ac{TPM} measurements used to encrypt the \ac{VMK} or when a validation profile is used that does not include \ac{PCR}0, we were able to sniff the communication between the \ac{TPM} and the Windows Boot Manager to retrieve the unencrypted \ac{VMK} for use with Dislocker.
We showed that this is the case when using a Secure Boot configuration that uses only Microsoft's signature \ac{DB} required to boot Windows.
This forces a default validation profile of \code{\{7, 11\}}, leaving out \ac{PCR}0.
Our rootkit attack was able to gain access to this type of system without requiring any prior knowledge or additional user input.