\section{Secure Boot and Bitlocker}
% https://edk2-docs.gitbook.io/edk-ii-uefi-driver-writer-s-guide/3_foundation/36_protocols_and_handles/365_tag_guid

assumptions:
secure boot or not
bitlocker enabled with TPM auto decryption, no PIN, no startup key

observation:
boot execution differs from executing rootkit
tpm values different
bitlocker auto decryption fails
recovery key prompt

what is the reaction of the average user
(ask admin for recovery password)
type in recovery password
alternative would be to remove drive and insert into safe device

% ref to background os loader
prompt is done by the OS Loader
ergo still during transient system load phase
required to use protocol services
therefor uses uefi services for IO
such as SimpleTextInputEx Protocol
go over the two different input protocols
find out which one is used

explain more in depth how protocols are returned to the end user
one instance per controller/handle

explain basic hooking
explain how we retain information of the hook in question
map protocol pointer to hook information
keylog recovery key
key input advancment is weird and makes tracking tricky

alternatively screen shot
still need hook to find when enter is pressed
explain how screenshotting works
some basic compression
wait for recovery key
send recovery key on enter press

on real hardware
network stack wasn't installed onto handles when boot over ip was disabled
compared loaded dxe drivers between both configurations with efi shell
Realtek Family driver not loaded
load manually
reinstall all handle to controllers to enable network stack regardless

sending key out is only good for physical access attack vector
dislocker linux utility
\cite{dislocker}
mount encrypted drive with decryption mean
read and write access
dual boot in vm
enter recovery key and it works
port to uefi
bitlocker encrypts block-wise
uefi protocol stack
hook block io
again hook data mapping
dislocker validate block
solves recovery key advancement issue

hook ExitBootServices
enable hook
write payload
import registry key
disable hook

next boot would require to input tpm values again
% https://learn.microsoft.com/en-us/windows/security/information-protection/bitlocker/bitlocker-use-bitlocker-drive-encryption-tools-to-manage-bitlocker
update tpm values in payload
caveat pin? look into this

% reference to rootkit definition
persistence when part of root of trust
fresh install / tpm update values
% paper von betreuern
hook Trusted Copmuting Group 2 (TCG2) Protocol
TPM communication
\cite[6.7.3]{tcg-efi-platform-spec}
% \cite[12.7 TPM_Unseal]{tcg-tpm-library-part3-commands}
receive bitlocker vmk key and send to dislocker

% https://labs.withsecure.com/publications/sniff-there-leaks-my-bitlocker-key