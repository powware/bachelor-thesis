% !TeX root = ../../thesis.tex

\section{Secure Boot}
% how does one enable it
mostly comes with default keys OEM

\subsection{Bootkit}

\subsubsection{Infection}

booting into usb installer doesnt work without disabling secure boot, if you already have access to the device and have to change the boot order to include removal media, you could very well also disable secure boot
have to assume that bios is password protected
what happens when removable media is still before windows

windows installer executable should still work without any difference

\subsubsection{Approach}

expectation:
not to boot
observation:
doesnt boot

\subsection{Rootkit}

\subsubsection{Infection}

add DXE Drivers to the DXE Volume.
% how to achieve assumption/requirement
This can be achieved by having read/write access to the SPI flash or using the Signed Capsule Update. Gaining read/write access to the SPI Flash is possible either through physical access to the device by using an SPI clamp on the chip itself or through exploits like for example the
% see SMM multi threaded exploit
. Signed Capsule Updates can be leveraged with access to private vendor information by signing the payload to make it appear legitimate or by intercepting the distribution process and employing infected firmware.
% ref lenovo vantage for distributed example
% network boot maybe

\subsubsection{Approach}

no difference
% ref to paper default policy
secure boot default policy snippet
% https://github.com/tianocore/edk2/blob/f3da13461cbed699e54b1d7ef3fba5144cc3b3b4/SecurityPkg/Library/DxeImageVerificationLib/DxeImageVerificationLib.c#L1704
option roms and bootloader
% ref background firmware update
instead relies on Signed Capsule Updates
assumes integrity

\cite[32.3]{uefi-spec}

policy defined
\cite[32.5.3.3 Authorization Process]{uefi-spec}
