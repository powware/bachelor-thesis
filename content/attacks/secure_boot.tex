% !TeX root = ../../thesis.tex

\section{Secure Boot}
\label{sec:attacks:secure-boot}

Our second attack is against systems with Secure Boot enabled.

\subsection{Bootkit}

For the installation via removable media we have to assume that the \ac{BIOS} menu is password protected, as we otherwise could simply turn off Secure Boot.
This makes the likelihood of an infection via this method smaller since we now solely rely on the boot order/firmware policy to prefer removable media.
Even given this assumption we promptly see that Secure Boot already denies execution of the installer when trying to boot it.
The same denial is observable for the bootkit itself, when using our Windows installer.
The Windows Boot Manager boot option pointing to our bootkit is now denied execution, if we were to have overwritten the standard boot entry of the hard drive \lstinline{EFI\Boot\bootx64.efi}, a copy of the Windows Boot Manager, Windows would now be rendered unbootable.

\subsection{Rootkit}

When installing our rootkit on \autoref{sec:test-setup:asrock}, we observe that Secure Boot is not applying its verification to our \ac{DXE} drivers, as they are still being executed without any restrictions.
When we look at the reference implementation in \ac{EDK} II, we can see why.
\autoref{lst:dxe-image-verification-handler} shows, that the image origin dictates which is applied.
Standard policy for images from a \acf{FV} (\lstinline{IMAGE_FROM_FV}) is to always allow execution. This aligns with what the \ac{UEFI} specification says on Secure Boot Firmware Policy:
\textcquote[32.5.3.2]{uefi-spec}{The firmware may approve \ac{UEFI} images for other reasons than those specified here.
    For example: whether the image is in the system flash \textelp{}}. This behavior was reproducible on all our hardware setups, likely in order to prevent accidentally entirely unbootable firmware or to reduce boot time.

\TODO{MEEE}