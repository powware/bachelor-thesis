% !TEX root = ../thesis.tex
%
\chapter{Background}

general introduction to UEFI
replace bios
security

\section{UEFI}
\subsection{Boot Sequence}

\subsubsection{1. Security (SEC)}
establishment of root of trust
\subsubsection{2. Pre-EFI (PEI)}
\subsubsection{3. Driver Execution Environment (DXE)}
dxe core
dxe dispatcher
dxe drivers
\subsubsection{4. Boot Device Selection (BDS)}
\subsubsection{5. Transient System Load (TSL)}
boottime and runtime services/driver
bootloader
ExitBootServices()
\subsubsection{6. Runtime (RT)}
runtime services/driver
\subsubsection{7. Afterlife (AL)}
hibernation
sleep


\subsection{UEFI/PI Firmware Images}
flash device
flash volume
flash file system

\subsection{Executables}
PE32 file format
fixed and dynamic address loading
relocatable
application vs driver
boot and runtime memory

\subsection{Programming}
boot and runtime services
boot service table
guids
handles and protocols
protocols

\subsection{edk2}
build system

\subsection{Security}
% https://edk2-docs.gitbook.io/understanding-the-uefi-secure-boot-chain/
\subsubsection{Secure Boot}
% https://edk2-docs.gitbook.io/understanding-the-uefi-secure-boot-chain/secure_boot_chain_in_uefi/uefi_secure_boot
\subsubsection{Signed Capsule Update}

\section{Windows}
\subsection{User Access Control (UAC)}
\subsection{Signing}
\subsection{Bitlocker}
how does it work
explain TPM



