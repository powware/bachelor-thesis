% !TeX root = ../thesis.tex

% https://www.enago.com/academy/abstract-versus-introduction-difference/
% https://www.student.unsw.edu.au/introductions
% https://www.scribbr.com/dissertation/introduction-structure/

\chapter{Introduction}


% what is UEFI
As the first piece of software that is run on your computer, UEFI holds an immense amount of responsibility during system initialization, attacks targeting your operating system from this environment are executed long before

what does it different than bios
this helps write platform independent code
uefi threats:
% definition of rootkit and bootkit
A rootkit is a collection of software designed to grant a threat actor control over a system, typically with malicious intend.
Rootkits set up a backdoor exploit and may deliver additional malware while leveraging their privileges to remain hidden.
There are different types of rootkits such as User Mode, Kernel Mode, Bootkits (bootloader rootkits), Hypervisor and Firmware rootkits.
\cite{microsoft-secure-the-windows-boot-process}
\cite{crowdstrike, techtarget}
\TODO{consult abstract for similar definition, how easy uefi makes it to write hardware independent payload}
Firmware rootkits targets the software running during the boot process, which is responsible for the system initialization.
This is done before the operating system is executed making them particularly hard to find, they are also persistent across operating system installation or hard drive replacements.
\cite{crowdstrike}


look at UEFI + threats against windows
danger of uefi infection
in recent years root and bootkits have popped up in the wild and been analysed
differences of root-/bootkits
reason about infection scenarios
we will discuss their commonalities
attack vectors:
- storage based
- memory based
implement a storage based ourselves
analyse security mechanism to prevent these attacks by attempting an attack itself
discuss security mechanisms we encounter
increasing security mechanisms
add onto past threats by attacking bitlocker
reflect their weaknesses
how to potentially evade them
- analyse countermeasures against UEFI threats
- Trusted Boot: KMCI from windows
- Secure Boot
- TPM
- Bitlocker
- firmware lock + signed capsule update
-


\section*{Overview}

We start off in Chapter 2 by introducing all necessary knowledge about the \ac{UEFI} environment, defined by the \ac{UEFI} and \ac{PI} specifications, listing the interface and its implementation.
This allows us to go over Windows~11's \ac{UEFI} installation and boot process as well as relevant security mechanisms in Chapter 3.
With this knowledge we then look at analyses of previously discovered \ac{UEFI} threats in Chapter 4, categorizing them by their attack vector and threat model.
In Chapter 5 we discuss the test setups, we performed our attacks on, consisting of emulation and hardware.
We then lay out our practical approach of implementing our own \ac{UEFI} attacks in Chapter 6, analyzing security mechanism faced when attempting attacks from the UEFI environment.
Afterwards we discuss the impact of our findings, the restrictions that apply, as well as potential mitigation techniques in Chapter 7.
Chapter 8 concludes the thesis by summarizing the achievements of our attacks and lays out potential future topics.