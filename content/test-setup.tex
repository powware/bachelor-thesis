% !TeX root = ../thesis.tex

\chapter{Test Setup}
\label{sec:test-setup}

We perform our attacks against Windows 11 on three different setups, as even though all three \ac{UEFI} firmwares used, are \cite{pi-spec} compliant, there still are many things left up to the \acp{OEM} to decide, when implementing a firmware image.

\section{\acs{QEMU}}
\label{sec:test-setup:qemu}

Our main development setup is an emulated environment using the emulator \ac{QEMU} \cite{qemu} together with the \ac{OVMF} image, from \ac{EDK} II (\lstinline{edk2-stable202208}).
For Secure Boot we generate our own \ac{PK} and use the \emph{Microsoft Corporation \acs{KEK} \acsu{CA} 2011} as \ac{KEK} and the two signature \acp{DB} \emph{Microsoft Windows Production PCA 2011} and \emph{Microsoft Corporation UEFI CA 2011} from Microsoft. The former required for their \ac{UEFI} executables used during the Windows boot process \cite{microsoft-secure-boot-guidance} and the latter reserved for third party executables signed at Microsoft's discretion after manual review \TODO{better source} \cite{microsoft-uefi-signing}.
In the attacks against BitLocker we use \emph{swtpm} for the emulation of a software \ac{TPM} \cite{swtpm}. Accessing the firmware image with this setup is just done through simple file access.

\section{Lenovo Ideapad 5 Pro-16ACH6}
\label{sec:test-setup:lenovo}

Lenovo Ideapad 5 Pro-16ACH6

% https://www.lenovo.com/de/de/laptops/ideapad/500-series/IdeaPad-5-Pro-16ACH6/p/88IPS501619?orgRef=https%253A%252F%252Fwww.google.com%252F

microsoft device guard

secure boot default keys


This can be done by using a spi flash programmer and clamping the physical chip. \TODO{FLASHROM}

\section{ASRock A520M-HVS}
\label{sec:test-setup:asrock}

\TODO{describe test setup}


secure boot und bitlocker

% https://www.asrock.com/mb/AMD/A520M-HVS/index.asp#BIOS

A520M-HVS 2.30 latest firmware at time of writing
Ryzen 5 5600X Zen 3

secure boot default keys


flashrom -p internal
SPI chip emulator. \TODO{EM100}
