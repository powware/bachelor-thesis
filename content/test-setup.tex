% !TeX root = ../thesis.tex

\chapter{Test Setup}
\label{sec:test-setup}

fresh Windows 11 installation

\section{\ac{QEMU}}
\label{sec:test-setup:qemu}

qemu + swtpm
% for qemu we can build it ourselves
If we want to use emulation we can build the \ac{OVMF} Package from EDK II which is a firmware image for virtual machines.
% https://projectacrn.github.io/latest/tutorials/waag-secure-boot.html
later found that only microsoft production Db should be used
% https://learn.microsoft.com/en-us/troubleshoot/windows-server/deployment/pcr7-configuration-binding-not-possible

\section{Lenovo Ideapad 5 Pro-16ACH6}
\label{sec:test-setup:lenovo}

Lenovo Ideapad 5 Pro-16ACH6

% https://www.lenovo.com/de/de/laptops/ideapad/500-series/IdeaPad-5-Pro-16ACH6/p/88IPS501619?orgRef=https%253A%252F%252Fwww.google.com%252F

microsoft device guard

secure boot default keys

\section{Flash access}

This can be done by using a spi flash programmer and clamping the physical chip. \TODO{FLASHROM}

\section{ASRock A520M-HVS}
\label{sec:test-setup:asrock}

\TODO{describe test setup}


secure boot und bitlocker

% https://www.asrock.com/mb/AMD/A520M-HVS/index.asp#BIOS

A520M-HVS 2.30 latest firmware at time of writing
Ryzen 5 5600X Zen 3

secure boot default keys

\section{Flash access}

flashrom -p internal
SPI chip emulator. \TODO{EM100}
